\documentclass[11pt]{article}
\usepackage[margin=0.8in]{geometry}

\title{Appunti corso di Reti}

\begin{document}
\begin{center}
    \large\textbf{Domande Reti}
\end{center}
\subsubsection*{Spiegare brevemente la rete internet e il concetto di protocollo e standard, facendo riferimento alla temporizzazione.}
Internet è una rete di reti, formata da Internet Service Providers (ISPs) interconnessi, ed una rete di calcolatori, che
connette miliardi di dispositivi, detti end systems. Gli end systems sono connessi tra di loro da una rete di link di 
comunicazione, come fibra ottica, rame e onde radio, e packet switches, come router e switches. Dal punto di vista dei 
servizi, Internet è un'infrastruttura che fornisce servizi alle applicazioni, come il Web, streaming o email, dette applicazioni 
distribuite; inoltre fornisce un'interfaccia di programmazione per le applicazioni distribuite: queste infatti usano Internet 
come un servizio di trasporto.\\
Un protocollo definisce il formato e l'ordine dei messaggi scambiati tra due o più entità in comunicazione tra loro, e definisce 
le azioni da compiere all'invio e/o alla ricezione di un messaggio o altri eventi. 
Un protocollo è standard quando tutte le parti coinvolte concordano con la sua implementazione.
La temporizzazione è un aspetto cruciale nei protocolli di rete in quanto questi impongono rigidi vincoli di tempo per 
la ricezione ed invio dei messaggi.
\subsubsection*{Elencare le varie tipologie di accesso alla rete esistenti e spiegare brevemente il concetto di capacità trasmissiva.} 
La rete d'accesso è ciò che collega fisicamente un end system ad un edge router. Le tipologie di reti d'accesso sono divise 
in tre principali categorie: domestica, aziendale e mobile.
I due tipi di accesso domestici più diffusi sono DSL e cavo. L'accesso DSL è di solito ottenuto dalla stessa compagnia 
telefonica che fornisce servizio telefonico: ogni modem DSL usa la linea telefonica esistente per scambiare dati con il 
DSL multiplexer presente nella centrale telefonica. L'accesso via cavo utilizza l'infrastruttura televisiva. Esistono inoltre 
la connessione FTTH e 5G fixed wireless.\\
In contesti aziendali, un local area network (LAN) viene usato per connettere end systems all'edge router. La tecnologia 
prevalente è Ethernet, che sfrutta un doppino in rame per connettere un sistema ad uno switch Ethernet, quest'ultimo connesso 
ad Internet. Sta diventando sempre più prevalente l'accesso wireless: in una LAN wireless, i sistemi trasmettono e ricevono 
pacchetti da un punto d'accesso.\\
I dispositivi mobili possono connettersi ad Internet tramite reti d'accesso mobili, come 3G, 4G e 5G.\\
La capacità trasmissiva si riferisce alla quantità di dati che una connessione può trasmettere in un determinato lasso di 
tempo ed è solitamente è misurata in byte/secondo. Viene anche chiamata larghezza di banda.
\subsubsection*{Descrivere la movimentazione dei dati nella rete con il concetto di commutazione a pacchetto e commutazione 
a circuito con rispettivi vantaggi e svantaggi.}
La commutazione a circuito è usata prevalentemente nelle reti telefoniche tradizionali ed offre una connessione end-to-end 
dedicata, nella quale tutte le risorse necessarie per la comunicazione (buffers, larghezza di banda) sono riservate fino 
al termine della comunicazione stessa. Data questa garanzia, il mittente può trasferire dati al destinatario a velocità 
costante e senza accodamenti. Un circuito può essere implementato tramite FDM o TDM. Con FDM, le frequenze (elettromagnetiche 
o ottiche) sono divise in bande, una per ogni connessione. Con TDM, il tempo è diviso in frame, ed ogni frame è diviso in 
slot; quando la rete stabilisce una connessione, dedica uno slot di tempo in ogni frame a questa connessione. I principali 
svantaggi sono: spreco delle risorse di rete durante i periodi in cui non vi è comunicazione tra gli end sytem e minor numero 
di utenti connessi.\\
La commutazione a pacchetto non offre una connessione dedicata, in quanto non vengono preallocate tutte le risorse necessarie 
a una connessione, e ciò può causare ritardi dovuti ad accodamenti dei pacchetti nei router. I principali vantaggi sono: 
facilità di implementazione, maggior numero di utenti connessi contemporaneamente.
\subsubsection*{Elencare e descrivere in modo accurato i ritardi che sono presenti all'interno della rete.}
I tipi di ritardo presenti all'interno della rete sono quattro: trasmissione, accodamento, processamento e propagazione.\\
Il ritardo di trasmissione è pari a lunghezza pacchetto/velocità di trasmissione: questo è il tempo necesssario per trasmettere 
tutti i bit di un pacchetto su un collegamento.\\
Il ritardo di accodamento si verifica quando i pacchetti attendono di essere trasmessi su un collegamento, e dipende dalla 
congestione del router (quanti pacchetti sono contenuti nei buffer).\\
Il ritardo di processamento è dato dal tempo richiesto per esaminare l'header di un pacchetto e determinare dove debba 
essere inviato.\\
Il ritardo di propagazione è il tempo impegato da un bit per propagarsi da un router ad un altro, e dipende dal medium 
fisico utilizzato.\\
Il ritardo nodale è dato dalla somma di questi quattro ritardi; il ritardo end-to-end è dato dalla somma dei ritardi nodali.
\subsubsection*{Descrivere la struttura ISO e OSI con tutte le loro caratteristiche e specificare quale delle due sia meglio.}
La struttura ISO è un modello di riferimento a 5 livelli: applicazione, trasporto, rete, collegamento e fisico.
Il livello applicazione fornisce supporto per le applicazioni di rete (HTTP, SMTP, ...); le applicazioni si scambiano messaggi. 
Il livello trasporto è adibito al trasferimento dati tra gli endpoint delle applicazioni di rete; i protocolli del livello 
trasporto incapsulano i messaggi in un segmento. Il livello rete trsferisce il segmento de un host ad un altro; i segmenti 
sono incapsulati in un datagramma. Il livello collegamento trasferisce il datagramma da un host ad un host vicino ad esso 
(routing da router a router); il datagramma è incapsulato in una trama (frame). Il lavoro dell livello fisico è quello di 
muovere i bit in un frame da un nodo ad un altro del percorso.\\
La struttura OSI è un modello di riferimento a 7 livelli. I due livelli aggiuntivi rispetto al modello ISO si trovano tra 
il livello applicazione ed il livello trasporto e sono il livello prestazione ed il livello sessione. Quest'ultimo è il 
migliore in quanto pià esaustivo ed accurato.
\subsubsection*{Elencare le architetture che possono essere presenti all'interno delle reti, accennare anche alla versione Bit-torrent per quanto riguarda P2P.}
I due tipi di architetture presenti all'interno delle reti sono l'architettura client-server e l'architettura P2P.
Nell'architettura client-server, il server è un host sempre online, con un indirizzo IP fisso, mentre il client è un host 
non sempre online che contatta e comunica con il server. I client non comunicano direttamente tra loro, comunicano tramite 
un server.\\
Nell'architettura P2P, gli host comunicano direttamente tra loro, fornendosi reciprocamente servizi. Quest'architettura 
è scalabile, ossia nuovi host (detti anche peers) aumentano la capacità di servizio.\\
Bit-Torrent è il protocollo P2P più diffuso. Una gruppo di peer che partecipano nella distribuzione di un file è detta 
torrent; i peer in un torrent scaricano/caricano chunk (di solito 256 kb) di un file tra loro. Ogni torrent ha un nodo 
detto tracker che viene contattato dai peer per informarlo della loro presenza nel torrent. Periodicamente, un peer chiede 
ad oogni peer la lista dei chunk che posseggono e ne richiesde uno che gli manca, scegliendo per primo il più raro.
\subsubsection*{TCP, quale servizi offre? E' un protocollo definito affidabile e orientato alla connessione, spiegare cosa significa, descriverlo e dire quando un protocollo può essere definito ideale.}
TCP è un protocollo che offre trasferimento dati affidabile e controllo della congestione. Con trasferimento affidabile 
si intende avere certezza che tutti i dati inviati da una applicazione su un host vengano ricevuti dall'applicazione presente 
su un altro host, e che questi vengano ricevuti in ordine e senza errori. TCP assicura trasporto affidabile tramite l'invio 
di conferme di ricezione (ACKs) da parte del destinatario e ritrasmissione in caso i segmenti vengano persi o danneggiati
durante il trasporto.\\
Con orientato alla connessione si intende un 
protocollo che si assicuri di avere una connessione stabile ad un host (in questo caso, da client a server) prima di iniziare 
l'invio dei dati. TCP assicura una connessione tramite un three-way handshake: prima di inviare dati, il client invia un 
segnale di sincronizzazione al server, il quale (eventualmente) riscontra il messaggio di sincronizzazione; ricevuto il 
riscontro, il client invia a sua volta un riscontro al server e la connessione viene così stabilita.
\end{document}

\documentclass[11pt]{book}
\usepackage[margin=.8in]{geometry}
\usepackage[italian]{babel}

\title{Appunti PISSIR}

\begin{document}
\chapter{Link Layer e LAN}
\section{Il link layer}
I dispositivi che supportano un protocollo link-layer sono detti \textit{nodi}. I canali di comunicazione che connettono
nodi adiacenti sono detti \textit{collegamenti}. Un nodo incapsula il datagramma ricevuto dal network layer sovrastante 
in un \textit{link-layer frame} e lo trasmettono sul collegamento.
\subsection{I servizi forniti dal link layer}
\subsubsection{Incapsulazione}
Quasi tutti i protocolli link-layer incapsulano i datagrammi ricevuti dal network layer prima di trasmetterli sul collegamento.
Il frame è composto da un campo dati, dove viene inseito il datagramma, e degli header.
\subsubsection{Accesso al collegamento}
Un protocollo di medium access control (MAC) specifica come il frame deve essere trasmesso sul collegamento. 
\subsubsection{Trasporto affidabile}
Un protocollo di trasferimento affidabile garantisce che ogni frame raggiunga la sua destinazione senza errori.
\subsubsection{Individuazione e correzione degli errori}
Il nodo mittente fornisce un meccanismo per individuare gli errori, che verranno poi corretti dal destinatario.
\subsection{Implementazione del link layer}
Le funzionalità Ethernet sono integrate nella scheda madre o in un chip Ethernet. Il link layer è implementato su un chip 
detto \textit{network adapter} o \textit{NIC}. 
\section{Individuazione e correzione degli errori}
\subsection{Controlli di parità}
La forma più semplice di error detection è l'utilizzo di un bit di parità. Gli schemi di parità possono essere pari o dispari.
Con uno schema di parità \textit{bidimensionale}, dove i bit sono disposti a matrice, è possibile identificare il bit 
corrotto e correggerlo. Questo schema non può correggere due errori in un singolo pacchetto, ma li può individuare.
\subsection{CRC}
I codici CRC (\textit{cyclic redundancy check}) sono anche detti codici polinomiali in quanto è possibile considerare la 
stringa da inviare come un polinomio i quali coefficienti sono i valori 0 e 1 della stringa di bit. I CRC possono individuare 
$resto\leq$ bit errati consecutivi.
\section{Multiple access link}
Esistono due tipi di collegamento. Il collegamento \textit{point to point} consiste in un solo mittente e un solo destinatario.
Il collegamento \textit{broadcast} può avere più nodi mittenti e destinatari connessi allo stesso canale di broadcast. 
Per coordinare la trasmissione di pacchetti all'interno di una rete broadcast, si utilizzano \textit{protcolli di accesso multiplo}.
Quando due o più nodi trasmettono frame allo stesso momento, i frame \textit{collidono} e vengono persi. Esistono tre tipi
di protocolli ad accesso multiplo.
\subsection{Protocolli a partizionamento di canale}
\begin{itemize}
    \item \textit{TDMA}: l'accesso al canale avviene in "round", e ogni nodo ha uno slot di tempo in cui può trasmettere 
    in ognuno di questi round. Il tasso di trasmissione è $R/N$, e i canali inutilizzati vengono sprecati.
    \item \textit{FDMA}: divide il canale da $R$ bps in multiple frequenze e assegna ciascuna di queste frequenze ai nodi.
    \item \textit{CDMA}: assegna un codice a ogni nodo, e ogni nodo usa quel codice per codificare i bit da inviare.
\end{itemize}
\subsection{Protocolli ad accesso casuale}
Con questi protocolli, il nodo mittente trasmette alla velocità concessa dal canale. Quando avviene una collisione, i nodi
affetti ritrasmettono il pacchetto finchè non viene ricevuto, attendendo prima di ritrasmettere il pacchetto. Ogni nodo 
sceglie indipendentemente l'intervallo di ritrasmissione. 
\subsubsection{Slotted ALOHA}
Assumiamo che i frame abbiano tutti la stessa dimensione $L$, il tempo sul canale sia diviso in slot di dimensione $L/R$,
i nodi siano sincronizzati e se due frame collidono in un slot, tutti i nodi rilevano la collisione in quello slot. 
Sia $p$ una probabilità tra 0 e 1.
\begin{itemize}
    \item Quando un nodo ha un frame da inviare, attende fino al prossimo slot e lo trasmette.
    \item Se non si verifica una collisione, il nodo non ritrasmette.
    \item Altrimenti, il nodo rileva una collisione e ritrasmette il suo frame negli slot successivi con probabilità $p$ 
    finchè non riesce a ritrasmettere senza collisione
\end{itemize}
La probabilità introduce casualità, che rende più efficiente il protocollo.
Slotted ALOHA permette a ogni nodo di trasmettere a full rate ed è decentralizzato, ma nelle collisioni gli slot vengono
persi, possono esserci slot inutilizzati ed è necessario un meccanismo di sincronizzazione. Inoltre, non è molto efficiente:
con un gran numero di nodi, solo il 37\% degli slot viene effettivamente utilizzato.
\subsubsection{CSMA}
I protocolli CSMA e CSMA/CD seguono due regole: 
\begin{itemize}
    \item \textit{carrier sensing}: un nodo ascolta sul suo canale prima di trasmettere; se il canale è occupato, attende
    \item \textit{collision detection}: se un nodo rileva un altro nodo che sta trasmettendo sul canale, annulla la trasmissione
\end{itemize}
Nonostante il carrier sensing, le collisioni possono comunque avvenire a causa del ritardo di propagazione. Quando avviene
una collisione, tutto il tempo impiegato a inviare un pacchetto viene sprecato. Il CSMA semplice non effettua collision 
detection.
\subsubsection{CSMA/CD}
CSMA/CD effettua collision detection, annullando una trasmissione se nota che il canale è occupato. Prima di ritrasmettere, 
il nodo attende una quantità di tempo casuale. Algoritmo CSMA/CD di Ethernet:
\begin{enumerate}
    \item Ethernet riceve il datagramma dal livello rete e lo incapsula in un frame.
    \item Se il canale è libero, trasmette, altrimenti attende.
    \item Se non vengono rilevate collisioni, la trasmissione è andata a buon fine.
    \item Altrimenti, la trasmissione viene annullata e viene inviato un segnale.
    \item Ethernet entra in fase di \textit{binary exponential backoff}: dopo la $m$-esima collisione, sceglie un numero 
    $K$ con $0\leq 2^{m-1}$. Attende $512K$ bit volte e ritorna al punto 2, ripetendo fino al completamento della trasmissione.
\end{enumerate}
\subsection{Protocolli a turni}
I protocolli a turni permettono di ottenere buone prestazioni sia con carichi leggeri che con carichi pesanti. 
\begin{itemize}
    \item \textit{Polling protocol}. Un nodo viene scelto come nodo master. Questo interpella ognuno dei nodi con metodo 
    round-robin. In questo modo, vengono eliminate le collisioni e non ci sono tempi morti. I principali svantaggi sono 
    l'introduzione di un ritardo di polling, la latenza e il potenziale fallimento del nodo master.
    \item \textit{Token-passing protocol}. Un frame apposito, detto token, viene passato in ordine tra i nodi. Quando un 
    nodo riceve un token, lo tiene finchè ha dei frame da trasmettere. I principali svantaggi sono l'overhead introdotto 
    dal token, la latenza e il fatto che il token rappresenta un potenziale punto di rottura.
\end{itemize}
\section{Switched Local Area Networks}
Gli switch utilizzano indirizzi propri al link layer per inoltrare frame.
\subsection{Indirizzamento nel link layer e ARP}
\subsubsection{Indirizzi MAC}
Le interfacce di rete degli host e dei router hanno indirizzi link layer, ma non gli switch. Un indirizzo link layer viene 
chiamato \textit{indirizzo MAC}. Gli indirizzi MAC sono di solito lunghi 6 bytes (48 bit) e sono espressi in motazione 
esadecimale. Ogni interfaccia nella LAN ha un unico indirizzo MAC e un (localmente) unico indirizzo IP. Ogni interfaccia 
possiede un unico indirizzo MAC perché questi sono controllati dalla IEEE. 

Quando un'interfaccia vuole inviare un frame, inserisce l'indirizzo MAC di destinazione nel frame e lo inoltra sulla LAN.
Se un mittente vuole inviare un frame a tutte le interfacce presenti sulla LAN, questo inserisce un indirizzo speciale,
l'\textit{indirizzo broadcast} nel frame.
\subsubsection{ARP}
ARP è il protocollo che si occupa della traduzione tra indirizzi IP e MAC, ma solo per interfacce sulla stessa sottorete.

Ogni host e router possiedono una \textit{tabella ARP}, che contiene mappature tra indirizzi IP e MAC, e un valore di 
time-to-live che indica la durata di quella voce nella tabella.  

Per ottenere l'indirizzo MAC del destinatario, un mittente invia sull'indirizzo MAC di broadcast un \textit{pacchetto ARP},
attendendo la risposta del nodo con l'indirizzo IP corrispondente. Una volta ricevuta la risposta, il mittente aggiorna 
la sua tabella ARP.
\subsubsection{Inviare datagrammi in un'altra sottorete}
Come inviare un datagramma da un host $A$ a un host $B$? Supponiamo $A$ conosca l'indirizzo IP di $B$, l'indirizzo IP del
router e l'indirizzo MAC del router (via ARP). Allora:
\begin{itemize}
    \item $A$ crea un datagramma con destinazione l'indirizzo IP $B$ (non può conoscere il suo MAC)
    \item $A$ incapsula il datagramma in un frame indirizzato al router
    \item il frame è ricevuto dal router e passato a IP
    \item il router determina la giusta interfaccia sulla quale inoltrare il datagramma 
    \item il router incapsula il datagramma con destinazione indirizzo MAC di $B$
\end{itemize}
\subsection{Ethernet}
La LAN Ethernet originalmente utilizzava un bus coassiale per connettere i nodi, e i frame erano trasmessi in broadcast.
Successivamente, il bus è stato rimpiazzato con uno switch, che a differenza dei router opera esclusivamente sul link layer.
\subsubsection{Struttura di un frame Ethernet} 
\begin{itemize}
    \item \textit{Preambolo}. Serve a sincronizzare le interfacce.
    \item \textit{Indirizzo di destinazione}. Contiene l'indirizzo MAC di destinazione.
    \item \textit{Indirizzo di provenienza}. Contiene l'indirizzo MAC di provenienza.
    \item \textit{Tipo}. Serve all'interfaccia di destinazione per indirizzare il frame al giusto protocollo di livello rete.
    \item \textit{Campo dati}. Contiene il datagramma IP.
    \item \textit{CRC}.
\end{itemize}
Ethernet è connectionless e non affidabile. 
\subsection{Switch}
Gli switch sono dispositivi operanti nel link layer. Sono trasparenti gli host e ai router nella sottorete. Gli switch, 
come i router, hanno dei buffer per contenere i frame in eccesso.
\subsubsection{Forwarding e filtering}
Il \textit{filtering} è la funzionalità degli switch che determina se un frame deve essere inoltrato o scartato. Il 
\textit{forwarding} è la funzionalità che determina le interfacce alle quali il frame va inoltrato. Queste funzioni sono 
svolte tramite una \textit{tabella di switching}. Questa tabella contiene informazioni su alcuni dispositivi presenti in 
LAN. Ogni voce contiene un indirizzo MAC, il numero di interfaccia dello switch che porta a quell'indirizzo, e quando 
quell'indirizzo è stato aggiunto alla tabella.

Supponiamo che un frame con indirizzo di destinazione DD-DD-DD-DD-DD-DD giunga a uno switch sull'interfaccia $x$. Lo switch 
consulta la sua tabella. Si possono verificare tre casi.
\begin{itemize}
    \item Nessuna voce nella tabella corrisponde all'indirizzo DD-DD-DD-DD-DD-DD. Lo switch inoltra copie del frame a tutte 
    le interfacce eccetto l'interfaccia $x$.
    \item È presente una voce corrispondente nella tabella, che associa l'indirizzo DD-DD-DD-DD-DD-DD con l'interfaccia 
    $x$. Non serve imoltrare il frame ad altre interfacce, quindi viene scartato (filtering).
    \item È presente una voce corrispondente nella tabella, che associa l'indirizzo DD-DD-DD-DD-DD-DD con l'interfaccia 
    $y\neq x$. Il frame viene inoltrato sul segmento della LAN corrispondente all'interfaccia $y$ (forwarding).
\end{itemize}
\subsubsection{Self-Learning}
La tabella di uno switch viene riempita automaticamente, dinamicamente e autonomamente.
\begin{enumerate}
    \item La tabella è inizialmente vuota.
    \item Per ogni frame in arrivo, lo switch memorizza nella sua tabella l'indirizzo MAC del mittente del frame, l'interfaccia 
    dal quale è arrivato e quando. 
    \item Lo switch elimina dalla tabella un indirizzo se nessun frame con quell'indirizzo viene ricevuto in un certo lasso 
    di tempo.
\end{enumerate}
\subsubsection{Proprietà}
\begin{itemize}
    \item \textit{Eliminazione delle collisioni.} In una LAN gestita tramite switch non ci sono collisioni. Gli switch 
    bufferizzano i frame e non trasmettono più di un frame su uno stesso segmento. 
    \item \textit{Collegamenti eterogenei.} Poiché uno switch isola segmenti di LAN l'uno dall'altro, questi possono avere 
    velocità differenti o operare su architetture differenti.
    \item \textit{Gestione.} Uno switch semplifica la gestione della rete. Se un'interfaccia di un dispositivo si guasta 
    e continua a inviare frame in broadcast, uno switch può rilevare il problema e disconnettersi dall'interfaccia malfunzionante.
\end{itemize}
\subsection{VLAN}
Le reti LAN tradizionali sono affette da tre problemi:
\begin{itemize}
    \item \textit{Nessuna isolazione del traffico}. 
    \item \textit{Uso non efficiente degli switch}.
    \item \textit{Gestione degli utenti}.
\end{itemize}
\end{document}
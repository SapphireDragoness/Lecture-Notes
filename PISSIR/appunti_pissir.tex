\documentclass[11pt]{book}
\usepackage[margin=.8in]{geometry}
\usepackage[italian]{babel}

\title{Appunti Probabilità e Statistica}

\begin{document}
\chapter*{Link Layer e LAN}
\section*{Il link layer}
I dispositivi che supportano un protocollo link-layer sono detti \textit{nodi}. I canali di comunicazione che connettono
nodi adiacenti sono detti \textit{collegamenti}. Un nodo incapsula il datagramma ricevuto dal network layer sovrastante 
in un \textit{link-layer frame} e lo trasmettono sul collegamento.
\subsection*{I servizi forniti dal link layer}
\subsubsection*{Incapsulazione}
Quasi tutti i protocolli link-layer incapsulano i datagrammi ricevuti dal network layer prima di trasmetterli sul collegamento.
Il frame è composto da un campo dati, dove viene inseito il datagramma, e degli header.
\subsubsection*{Accesso al collegamento}
Un protocollo di medium access control (MAC) specifica come il frame deve essere trasmesso sul collegamento. 
\subsubsection*{Trasporto affidabile}
Un protocollo di trasferimento affidabile garantisce che ogni frame raggiunga la sua destinazione senza errori.
\subsubsection*{Individuazione e correzione degli errori}
Il nodo mittente fornisce un meccanismo per individuare gli errori, che verranno poi corretti dal destinatario.
\subsection*{Implementazione del link layer}
Le funzionalità Ethernet sono integrate nella scheda madre o in un chip Ethernet. Il link layer è implementato su un chip 
detto \textit{network adapter} o \textit{NIC}. 
\section*{Individuazione e correzione degli errori}
\subsection*{Controlli di parità}
La forma più semplice di error detection è l'utilizzo di un bit di parità. Gli schemi di parità possono essere pari o dispari.
Con uno schema di parità \textit{bidimensionale}, dove i bit sono disposti a matrice, è possibile identificare il bit 
corrotto e correggerlo. Questo schema non può correggere due errori in un singolo pacchetto, ma li può individuare.
\subsection*{CRC}
I codici CRC (\textit{cyclic redundancy check}) possono essere visualizzati come polinomi i quali coefficienti sono i valori 
della stringa di bit da inviare.
\end{document}
\documentclass[11pt]{article}
\usepackage[margin=.8in]{geometry}
\usepackage[italian]{babel}

\title{Domande esame Basi}

\begin{document}
\subsubsection*{Dare la definizione di chiave e superchiave.}
Una chiave è un insieme di attributi che identificano le tuple di una relazione. Un insieme $K$ di attributi è superchiave 
di una relazione $r$ se $r$ non contiene due tuple distinte $t_1$ e $t_2$ con $t_1[K]=t_2[K]$. $K$ è chiave per $r$ se è
una superchiave minimale per $r$, cioè non esiste un'altra superchiave $K'$ per $r$ che sia contenuta in $K$ come sottoinsieme 
proprio.
\subsubsection*{Dare la definizione di vincolo d'integrità referenziale e descriverne il significato.}
Un vincolo d'integrità è una proprietà che deve essere soddisfatta dalle istanze che rappresentano informazioni corrette 
per l'applicazione; il vincolo d'integrità referenziale è una di queste. Un vincolo d'integrità referenziale fra un insieme 
di attributi $X$ di una relazione $R_1$ e un'altra relazione $R_2$ è soddisfatto se i valori su $X$ di ciascuna tupla 
dell'istanza $R_1$ compaiono come valori di chiave dell'istanza di $R_2$.
\subsubsection*{Cosa significa che esiste una dipendenza funzionale tra un insieme di attributi $Y$ e $Z$ sullo schema di
relazione $R(X)$?}
Data una relazione $r$ su uno schema $R(X)$ e due sottoinsiemi di attributi non vuoti $Y$ e $Z$, si dice che esiste una
dipendenza funzionale tra $Y$ e $Z$ se, per ogni coppia di tuple $t_1$ e $t_2$ di $r$ aventi gli stessi valori sugli 
attributi $Y$, risulta che $t_1$ e $t_2$ hanno gli stessi valori anche sugli attributi $Z$. 
\subsubsection*{Dare la definizione di 3FN.}
Una relazione è in 3FN se, per ogni dipendenza funzionale $X\rightarrow A$ definita su di essa, $X$ contiene una chiave 
di $r$ oppure $A$ appartiene ad almeno una chiave di $r$.
\subsubsection*{Dare la definizione di BCNF.}
Una relazione è in BCNF se, per ogni dipendenza funzionale $X\rightarrow A$ definita su di essa, $X$ contiene una chiave 
$K$ di $r$, cioè $X$ è superchiave per $r$.
\subsubsection*{Cosa significa che una relazione si decompone senza perdita?}
Una relazione $r$ si decompone senza perdita su due relazioni se l'insieme degli attributi comuni alle due relazioni è 
chiave per almeno una delle relazioni composte.
\end{document}
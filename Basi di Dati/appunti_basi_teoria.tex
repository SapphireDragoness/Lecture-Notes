\documentclass[11pt]{book}
\usepackage[margin=1in]{geometry}
\usepackage[italian]{babel}
\usepackage{amsmath}
\usepackage{amssymb}

\title{Appunti Basi di Dati - Teoria}

\begin{document}
\chapter{Introduzione alle Basi di Dati}
\section{Sistemi informativi, informazioni e dati}
Ogni organizzazione è dotata di un \textit{sistema informativo}, che organizza e gestisce le informazioni necessarie per 
perseguire gli scopi dell'organizzazione stessa.

Per indicare la porzione automatizzata del sistema informativo di solito viene utilizzato il termina \textit{sistema informatico}.
Nei sistemi informatici le informazioni vengono rappresentate per mezzo di \textit{dati}.

Una \textit{base di dati} è una collezione di dati, utilizzati per rappressentare le informazioni di interesse per un 
sistema informativo.
\section{Basi di dati e sistemi di gestione di basi di dati}
Un \textit{sistema di gestione di basi di dati} (DBMS) è un sistema software in grado di gestire collezioni di dati che 
siano 
\begin{itemize}
    \item \textit{grandi}: in termini di occupazione di memoria
    \item \textit{condivise}: applicazioni e utenti diversi devono poter accedervi
    \item \textit{persistenti}: persistono anche dopo l'esecuzione del programma che le utilizza
\end{itemize}
assicurando la loro
\begin{itemize}
    \item \textit{affidabilità}: mantengono intatti i dati 
    \item \textit{privatezza}: mantengono sicuri e privati i dati
\end{itemize}
ed essendo
\begin{itemize}
    \item \textit{efficiente}: le operazioni vengono svolte rapidamente
    \item \textit{efficace}: rendono produttive le attività dei loro utenti
\end{itemize}
\section{Modelli dei dati}
Un \textit{modello di dati} è un insieme di concetti utilizzati per organizzare i dati di interesse e descriverne la 
struttura in modo che essa risulti comprensibile a un elaboratore.

Il \textit{modello relazionale} dei dati permette di definire tipi per mezzo del costruttore \textit{relazione}, che 
consente di organizzare i dati in insiemi di record a struttura fissa.

I \textit{modelli concettuali} vengono utilizzati per descrivere i dati in maniera indipendente dal modello logico. Un 
tipo di modello concettuale è il modello \textit{Entità-Relazione}.
\subsection{Schemi e istanze}
Nelle basi di dati esiste una parte sostanzialmente invariante nel tempo, detta \textit{schema} della base di dati, 
costituita dalle caratteristiche dei dati, e una parte variabile nel tempo, detta \textit{istanza} o \textit{stato} della 
base di dati, costituita dai valori effettivi.

Lo schema di una relazione è costituito dalla sua intestazioen, cioè dal nome della relazione seguito dai nomi dei suoi 
attributi, ad esempio:
\begin{center}
    \textbf{Docenza}(Corso,NomeDocente)
\end{center}
L'\textit{istanza di una relazione} è costituita dall'insieme, variante nel tempo, delle sue righe.
\chapter{Il modello relazionale}
\section{Il modello relazionale: strutture}
\subsection{Relazioni e tabelle}
Dati due insiemi $D_1$ e $D_2$, si chiama \textit{prodotto cartesiano} di $D_1$ e $D_2$ l'insieme di coppie ordinate 
$(v_1,v_2)$ tali che $v_1$ è un elemento di $D_1$ e $v_2$ è un elemento di $D_2$.
Il numero $n$ delle componenti del prodotto cartesiano viene detto $grado$ del prodotto cartesiano e della relazione. Il 
numero degli elementi ($n$-uple) della relazione viene chiamato \textit{cardinalità} della relazione.
\subsection{Relazioni con attributi}
Nelle basi di dati, ciascuna $n$-upla contiene dati fra loro collegati. Inoltre, una relazione è un insieme, quindi:
\begin{itemize}
    \item non è definito alcun ordinamento fra le $n$-uple
    \item le $n$-uple di una relazione sono distinte l'una dall'altra, in quanto tra gli elementi di un insieme non ce ne 
    possono essere presenti due uguali tra loro 
\end{itemize} 
Ciascuna $n$-upla è, al proprio interno, ordinata: l'$i$-esimo valore di ciascuna proviene dall'$i$-esimo dominio.

Indichiamo con $D$ l'insieme dei domini e specifichiamo la corrispondeza tra attributi e domini per mezzo della funzione 
$dom:X\rightarrow D$, che associa a ciascun attributo $A\in X$ un dominio $dom(A)\in D$. Diciamo che una \textit{tupla} 
su un insieme di attributi $X$ è una funzione $t$ che associa a ciascun attributo $A\in R$ un valore del dominio $dom(A)$.
Una \textit{relazione} su $X$ è un insieme di tuple su $X$.
\subsection{Relazioni e basi di dati}
Uno \textit{schema di relazione} è costituito da un simbolo $R$, detto \textit{nome della relazione}, e da un insieme di 
\textit{attributi} $X=\{A_1,A_2,\dots A_n\}$, indicato con $R(X)$. A ciascun attributo è associato un dominio.

Uno \textit{schema di base di dati} è un insieme di schemi di relazione con nomi diversi:
\begin{equation*}
    R=\{R_1(X_1),R_2(X_2),\dots,R_n(X_n)\}
\end{equation*}
I nomi di relazione hanno come scopo principale quello di distinguere le varie relazioni nella base di dati.

Un \textit{istanza di relazione} su uno schema $R(X)$ è un insieme $r$ di tuple su $X$.

Un \textit{istanza di base di dati} su uno schema $R=\{R_1(X_1),R_2(X_2),\dots,R_n(X_n)\}$ è un insieme di relazioni dove 
ogni relazione è una relazione sullo schema $R_i(X_i)$.
\section{Vincoli d'integrità}
Il \textit{vincolo d'integrità} è una proprietà che deve essere soddisfatta dalle istanze che rappresentano informazioni 
corrette per l'applicazione. Ogni vincolo può essere visto come un \textit{predicato} che associa a ogni istanza il valore 
\textit{vero} o \textit{falso}. Se il predicato assume il valore vero, allora diciamo che l'istanza \textit{soddisfa} il 
vincolo. Sono presenti deu categorie di vincoli:
\begin{itemize}
    \item Un vincolo è \textit{intrarelazionale} se il suo soddisfacimento è definito rispetto a singole relazioni della 
    base di dati 
    \begin{itemize}
        \item un \textit{vincolo di tupla} è un vincolo che può essere valutato su ciascuna tupla indipendentemente dalle 
        altre
        \item un vincolo definito con riferimento a singoli valori viene detto \textit{vincolo su valori} o \textit{vincolo 
        di dominio}
    \end{itemize}
    \item Un vincolo è \textit{interrelazionale} se coinvolge più relazioni
\end{itemize}
\subsection{Vincoli di tupla}
I vincoli di tupla esprimono condizioni sui valori di ciascuna tupla, indipendentemente dalle altre tuple.
\subsection{Chiavi}
Una chiave è un insieme di attributi utilizzato per identificare univocamente le tuple di una relazione.
Formalmente:
\begin{itemize}
    \item un insieme $K$ di attributi è \textit{superchiave} di una relazione $r$ se $r$ non contiene due tuple distinte 
    $t_1$ e $t_2$ con $t_1[K]=t_2[K]$
    \item $K$ è \textit{chiave} di $r$ se è una superchiave minimale di $r$, cioè non esiste un'altra superchiave $K'$ di 
    $r$ che sia contenuta in $K$ come sottoinsieme proprio
\end{itemize}
Ciascuna relazione e ciascuno schema di relazione hanno sempre una chiave. Una relazione è un insieme e quindi è costituita
da elementi fra loro diversi; di conseguenza, per ogni relazione $r(X)$, l'insieme $X$ di tutti gli attributi su cui è
definita è senz'altro una superchiave per essa. O tale insieme è anche chiave, nel qual caso si conferma l'esistenza della 
chiave stessa, oppure non è chiave, perchè esiste un'altra superchiave in esso contenuta.

Il fatto che su ciascuno schema di relazione possa essere definita almeno una chiave garantisce l'accessibilità a tutti 
i valori di una base di dati e la loro univoca identificabilità.
\subsection{Chiavi e valori nulli}
Su una delle chiavi, detta \textit{chiave primaria} si vieta la presenza di valori nulli; sulle altre, i valori nulli sono 
generalmente ammessi.
\subsection{Vincoli d'integrità referenziale}
Un \textit{vincolo d'integrità referenziale} fra un insieme di attributi $X$ di una relazione $R_1$ e un'altra relazione 
$R_2$ è soddisfatto se i valori su $X$ di ciascuna tupla dell'istanza di $R_1$ compaiono come valori della chiave (primaria)
dell'istanza di $R_2$.

Se la chiave di $R_2$ è unica e composta da un solo attributo $B$, il vincolo di integrità referenziale fra l'attributo 
$A$ di $R_1$ e la relazione $R_2$ è soddisfatto se, per ogni tupla $t_1$ in $R_1$ per cui $t_1[A]$ non è nullo, esiste 
una tupla $t_2$ in $R_2$ tale che $t_1[A_i]=t_2[B]$.

Nel caso generale, bisogna prestare attenzione al fatto che ciscuno degli attributi in $X$ deve corrispondere a un preciso 
attributo della chiave primaria $K$ di $R_2$. Allo scopo, è necessario specificare un ordinamento sia nell'insieme $X$ sia 
in $K$. Indicando gli attributi in ordine, $X=A_1 A_2 \dots A_p$ e $K=B_1 B_2 \dots B_p$, il vincolo è soddisfatto se per 
ogni tupla $t_1$ in $R_1$ senza nulli su $X$ esiste una tupla $t_2$ in $R_2$ con $t_1[A_i]=t_2[B_i]$, per ogni $i$ compreso
fra 1 e $p$.
\chapter{Algebra e calcolo relazionale}
\section{Algebra relazionale}
L'algebra relazionale è un linguaggio procedurale, basato su concetti di tipo algebrico. Esso è costituito da un insieme 
di operatori, definiti su relazioni e che producono ancora relazioni come risultati
\subsection{Unione, intersezione, differenza}
Le relazioni sono insiemi, quindi ha senso definire su di esse gli operatori insiemistici tradizionali di unione, differenza 
e intersezione:
\begin{itemize}
    \item l'\textit{unione} di due relazioni $r_1$ e $r_2$ definite sullo stesso insieme di attributi $X$ è indicata con 
    $r_1\cup r_2$ ed è una relazione ancora su $X$ contenente le tuple che appartengono a $r_1$ oppure a $r_2$, oppure 
    ad entrambe 
    \item l'\textit{intersezione} di $r_1(X)$ e $r_2(X)$ è indicata con $r_1\cap r_2$ ed è una relazione su $X$ contenente 
    le tuple che appartengono sia a $r_1$ sia a $r_2$
    \item la \textit{differenza} di $r_1(X)$ e $r_2(X)$ è indicata con $r_1 - r_2$ ed è una relazione su $X$ contenente 
    le tuple che appartengono a $r_1$ e non appartengono ad $r_2$
\end{itemize}
\subsection{Ridenominazione}
L'operatore di \textit{ridenominazione} cambia il nome degli attributi lasciando invariato il contenuto delle relazioni.

Sia $r$ una relazione definita sull'insieme di attributi $X$ e sia $Y$ un altro insieme di attributi con la stessa cardinalità.
Inoltre, siano $A_1 A_2 \dots A_k$ e $B_1 B_2 \dots B_k$ rispettivamente un ordinamento per gli attributi in $X$ e un 
ordinamento per quello in $Y$. Allora la ridenominazione:
\begin{equation*}
    \rho_{B_1 B_2 \dots B_k\leftarrow A_1 A_2 \dots A_k}(r)
\end{equation*}
contiene una tupla $t'$ per ciascuna tupla $t$ in $r$, definita come segue: $t'$ è una tupla su $Y$ e $t'[B_i]=t[A_i]$, 
per $i=1,\dots ,k$.
\subsection{Selezione}
La selezione produce un sottoinsieme di tuple su tutti gli attributi ("decomposizione orizzontale").

L'operatore è denotato dal simbolo $\sigma$, al pedice del quale viene indicata la "condizione di selezione". Il risultato 
contiene le tuple dell'operando che soddisfano la condizione di selezione.
Le condizioni di selezione possono prevedere confronti fra attributi e confronti fra attributi e costanti, ottenute 
combinando condizioni semplici con i connettivi logici $\wedge$, $\vee$ e $\lnot$.
\subsection{Proiezione}
La proiezione produce un risultato al quale contribuiscono tutte le tuple, ma su un sottoinsieme degli attributi ("decomposizione 
orizzontale").
\subsection{Join}
Esistono due versioni del join: il join naturale e il theta-join.
\subsubsection{Join naturale}
Il \textit{join naturale} è un operatore che correla dati in relazioni diverse, sulla base di valori uguali in attributi 
con lo stesso nome. Il risultato del join è costituito da una relazione sull'unione degli insiemi di attributi degli 
operandi e le sue tuple sono ottenute combinando le tuple degli operandi con valori uguali sugli attributi comuni.

In generale, un join naturale $r_1\Join  r_2$ di $r_1(X_1)$ e $r_2(X_2)$ è una relazione definita su $X_1 X_2$ come segue:
\begin{equation*}
    r_1\Join r_2=\{t\rightarrow X_1X_2|t[X_1]\in r_1\wedge t[X_2]\in r_2\}
\end{equation*}
\subsubsection{Join completi ed incompleti}
Il join di $r_1$ e $r_2$ contiene un numero di tuple compreso fra 0 e $|r_1|\times|r_2|$. Inoltre:
\begin{itemize}
    \item se il join di $r_1$ e $r_2$ è completo, allora contiene almeno un numero di tuple pari al massimo tra $|r_1|$ 
    e $|r_2|$
    \item se $X_1\cap X_2$ contiene una chiave per $r_2$, allora il join di $r_1(X_1)$ e $r_2(X_2)$ contiene al più $|r_1|$
    tuple 
    \item se $X_1\cap X_2$ coincide con una chiave per $r_2$ e sussiste il vincolo di riferimento fra $X_1\cap X_2$ in 
    $r_1$ e la chiave di $r_2$, allora il join $r_1(X_1)$ e $r_2(X_2)$ contiene esattamente $|r_1|$ tuple
\end{itemize}
\subsubsection{Join esterni}
Questa variante del join prevede che tutte le tuple diano un contributo al risultato, eventualmente estese con valori 
nulli ove non vi siano controparti opportune. Esistono tre varianti dell'operatore: il join esterno \textit{sinistro}, 
che estende solo le tuple del primo operando, quello \textit{destro}, che estende solo le tuple del secondo operando, e 
quello \textit{completo} che le estende tutte.
\subsubsection{Semijoin}
Questo operatore restituisce le tuple di una relazione che partecipano al join naturale di tale relazione con un'altra.
L'operatore può essere espresso per mezzo del join e della proiezione.
\subsubsection{Join n-ario, intersezione e prodotto cartesiano}
Il join naturale è associativo e commutativo. Se $X_1=X_2$, il join coincide con l'intersezione. Se i set sono disgiunti, 
il risultato del join sarà in \textit{prodotto cartesiano} delle due relazioni.
\subsubsection{Theta-join e equi-join}
Il \textit{theta-join} è un prodotto cartesiano seguito da una selezione.
\subsection{Interrogazioni in algebra relazionale}
Dato uno schema $R$ di base di dati, un'interrogazione è una funzione che, per ogni istanza $r$ di $R$, produce una relazione 
su un dato insieme di attributi $X$.

Prendendo come esempio lo schema:
\begin{center}
    \textbf{Impiegati}(\underline{Matr}, Nome, Età, Stipendio)\\
    \textbf{Supervisore}(Capo, \underline{Impiegato})
\end{center}
si possono formulare le seguenti interrogazioni:
\begin{itemize}
    \item trovare matricola, nome ed età degli impiegati che guadagnano più di 40 mila euro
    \begin{equation*}
        \pi_{\textnormal{Matr,Nome,Età}}(\sigma_{\textnormal{Stipendio$>$40}}(\textnormal{Impiegati}))
    \end{equation*}
    \item trovare matricola e nome dei capi i cui impiegati guadagnano più di 40 mila euro 
    \begin{align*}
        \pi_{\textnormal{Matr,Nome}}((\textnormal{Impiegati}\Join_{\textnormal{Matr=Capo}}))(\pi_{\textnormal{Capo}}
        (\textnormal{Supervisione}-\pi_{\textnormal{Capo}}(\textnormal{Supervisione}\Join_{\textnormal{Imp=Matr}}
        \\\sigma_{\textnormal{Stip$\leq$40}}(\textnormal{Impiegati}))))
    \end{align*}
\end{itemize}
\subsection{Viste}
Possono esistere due tipi di relazioni derivate:
\begin{itemize}
    \item \textit{viste materializzate}: relazioni derivate effettivamente memorizzate nella base di dati 
    \item \textit{relazioni virtuali}: relazioni definite per mezzo di funzioni, non memorizzate nella base di dati, ma 
    utilizzabili nelle interrogazioni come se lo fossero
\end{itemize}
\section{Calcolo relazionale}
Con il termine \textit{calcolo relazionale} si fa riferimento a una famiglia di linguaggi d'interrogazione, basati sul 
calcolo dei predicati del primo ordine, che hanno la caratteristica di essere dichiarativi, cioè di specificare le proprietà 
del risultato delle interrogazioni.
\subsection{Calcolo di tuple con dichiarazioni di range}
Le espressioni hanno la forma 
\begin{equation*}
    \{T|L|f\}
\end{equation*}
dove:
\begin{itemize}
    \item $T$ è la \textit{target list}, con elementi del tipo $x.Z$, con $x$ variabile e $Z$ attributo
    \item $L$ è la \textit{range list}, che elenca le variabili libere della formula $f$ con i relativi range
    \item $f$ è una formula 
\end{itemize}

\end{document}
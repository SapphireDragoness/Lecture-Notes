\documentclass[12pt]{article}
\usepackage[margin=0.8in]{geometry}
\usepackage{xcolor}

\title{Esercizi Sistemi Operativi 2}

\begin{document}
\section{Input/Output}
\subsection{MTTF}
Dati 4 dischi, calcolare MTTF orario di un'organizzazione RAID livello 0 e quello di
un'organizzazione RAID livello 1, sapendo che ogni disco i ha un tasso annuale di fallimento $\lambda_i$,
dove:
\begin{itemize}
    \item $\lambda_1$ = 0.05
    \item $\lambda_2$ = 0.075
    \item $\lambda_3$ = 0.25
    \item $\lambda_4$ = 0.025
\end{itemize}
Non è previsto l'utilizzo di dischi hot-spare. Inoltre, una volta che un disco si guasta non viene
sostituito da uno funzionante.\\\\
\textcolor{blue}{L'aggregato dei tassi di fallimento è $\lambda_{tot}$ = 0.05 + 0.075 + 0.25 + 0.025 = 0.4; il $MTTF_{tot}$ orario è quindi 24*365/0.4 = 21900 ore: dato un set di k dischi, 
la possibilità che un disco su quei k fallisca è almeno uguale alla possibilità che il disco con minor $MTTF_H$ fallisca.
Quindi, il MTTF orario di un'organizzazione RAID livello 0 è 21900 ore.\\
In un'organizzazione RAID livello 1 i dischi sono mirrored (quindi sono 8): in questo caso il $MTTF_{tot}$ orario è 24*365/0.8 = 10950 ore.}
\subsection{RAID}
Un cliente vi commissiona un sistema organizzato a RAID (non specificando il livello) specificando i seguenti requisiti:
\begin{itemize}
    \item Massimizzare capacità e prestazioni
    \item Affidabilità non necessaria
    \item Budget: €2000
\end{itemize}
Considerando solo i costi relativi all'acquisto dei dischi (escludendo quindi quelli relativi all'acquisto di un controller RAID) e supponendo di adottare dischi da 1 TiB, ciascuno venduto al
prezzo di €400, che organizzazione RAID suggerireste al cliente tra quelle standard (livello 0 fino a livello 6)? Motivare le risposte.\\\\
\textcolor{blue}{Dato il budget di €2000, è possibile acquistare un massimo di 5 dischi. Essendo l'affidabilità un requisito non necessario, si possono escludere tutte le organizzazioni che prevedano
qualche forma di ridondanza o error checking, quindi RAID0 risulta essere la scelta migliore.
Questa organizzazione RAID garantisce di massimizzare la capacità e le prestazioni (sia in scrittura, sia in lettura), in quanto non presenta alcuna forma di ridondanza, arrivando ad ottenere
5 TiB di spazio totale.}
\subsection{RAID}
Un cliente vi commissiona un sistema organizzato a RAID (non specificando il livello) specificando i seguenti requisiti:
\begin{itemize}
    \item Massimizzare affidabilità
    \item Nessun interesse su capacità e prestazioni
    \item Budget: €2000
\end{itemize}
Considerando solo i costi relativi all'acquisto dei dischi (escludendo quindi quelli relativi all'acquisto di un controller RAID) e supponendo di adottare dischi da 1 TiB, ciascuno venduto al
prezzo di €400, che organizzazione RAID suggerireste al cliente tra quelle standard (livello 0 fino a livello 6)? Motivare le risposte.\\\\
\textcolor{blue}{Dati i requisiti, si possono escludere RAID0 (non offre alcuna affidabilità), RAID1(non è possibile fare mirroring su 5 dischi), RAID 3,4,5 (sono limitati ad un solo disk failure possibile).
L'organizzazione RAID ideale è quella di livello 6, che ammette almeno due disk fault (sui 5 acquistati). Con questa soluzione si ottiene capacità utile di 3 TiB.}
\subsection{RAID}
Un cliente vi commissiona un sistema organizzato a RAID (non specificando il livello) specificando i seguenti requisiti:
\begin{itemize}
    \item Buon compromesso tra affidabilità e prestazioni (tolleranza ad almeno 1 disco rotto)
    \item Capacità utile di almeno 4 TiB
    \item Budget: €2000
\end{itemize}
Considerando solo i costi relativi all'acquisto dei dischi (escludendo quindi quelli relativi all'acquisto di un controller RAID) e supponendo di adottare dischi da 1 TiB, ciascuno venduto al
prezzo di €400, che organizzazione RAID suggerireste al cliente tra quelle standard (livello 0 fino a livello 6)? Motivare le risposte.\\\\
\textcolor{blue}{Dati i requisiti, si possono escludere RAID0 (non offre alcuna affidabilità), RAID 1,2,6 (non permettono di raggiungere capacità utile di 4 TiB).
Tra le organizzazioni RAID 3,4,5, quella ottimale risulta essere RAID 5 in quanto tra queste è quella che offre le prestazioni in lettura e scrittura migliori, ammettendo un disk fault.}
\subsection{DMA}
Si supponga che un sistema utilizzi il DMA per il trasferimento di dati dal controller del disco alla memoria principale del computer. Si assuma inoltre che l'acquisizione dell'accesso al bus richieda
t1 nsec in media e il trasferimento di una parola sul bus richieda t2 nsec (con t1 $\gg$ t2). Dopo che la CPU ha programmato il controller DMA, quanto tempo sarà necessario per trasferire 1000 parole
dal controller del disco alla memoria principale, se (a) è utilizzata la modalità cycle stealing (una parola alla volta) (b) è usata la modalità burst? Si assuma che l'invio di comandi al controller del
disco richieda l'acquisizione del bus per spedire una parola e che il riscontro richieda anch'esso l'acquisizione del bus per spedire una parola.\\\\
\textcolor{blue}{Nel caso venga usata la modalità \textit{cycle stealing}, sarà necessario un tempo pari a (t1+t2)*1000 nsec, in quanto si può trasferire una sola parola per volta, dovendo richiedere ogni volta l'accesso
al bus (richiede tempo t1). Nel caso venga usata la modalità \textit{burst}, sarà necessario un tempo pari a t1+(t2*1000), in quanto è necessario acquisire il bus una sola volta, ed effettuare il 
trasferimento delle parole in batch.}
\subsection{Dischi}
Un disco ruota a 7200 RPM. Nel suo cilindro più esterno ci sono 500 settori di 512 bytes.
\begin{itemize}
    \item Supponendo che la testina sia già posizionata sul cilindro più esterno e all'inizio di un settore, quanto ci s'impiega a leggere un singolo settore da quel cilindro?
    \item Qual è il massimo tasso di trasferimento dati (in byte/sec)?\\
\end{itemize}
\textcolor{blue}{Il tempo di trasferimento medio per un settore è pari a $massima$ $latenza$ $rotazionale / settori$ $per$ $traccia$; la massima latenza rotazionale è pari a $1/vel.$ $rotazione$. Quindi, per leggere
un settore ci s'impiega $T_T = 1/vel.$ $rotazione$ * $1/settori$ $per$ $traccia$ = $\frac{1}{7200/60}$ * 1000 * $\frac{1}{500}$ = 8 * 0.002 = 0.016 msec.\\
Il massimo tasso di trasferimento dati è pari a $byte$ $settore/T_T$ = $\frac{512}{0.016/1000}$ = 32000000 byte = 32 megabyte/sec.}
\subsection{Dischi}
Si consideri un disco magnetico con 16 testine e 400 cilindri. Il disco ha 4 zone da 100 cilindri ciascuna, dove i cilindri (e quindi le tracce) di zone differenti contengono 160, 200, 240, e 280
settori, rispettivamente. Si assuma che ogni settore contenga 512 byte, che il tempo medio di seek tra cilindri adiacenti sia 1 msec, e che il disco ruoti a 7200 RPM. Calcolare:
\begin{itemize}
    \item la capacità del disco
    \item il cylinder skew ottimale di ogni zona
    \item il massimo tasso di trasferimento dati (in byte/sec) raggiungibile\\
\end{itemize}
\textcolor{blue}{La capacità del disco si ottiene moltiplicando il numero di settori di ogni cilindro per il numero di cilindri, sommando i risultati, moltiplicando per 512 e per il numero di superfici (numero testine):
(16000 + 20000 + 24000 + 28000) *  512 * 16 = 720896000 byte = circa 720 megabyte.\\
Il cylinder skew si calcola tramite la seguente formula $\lceil \frac{tempo \quad di \quad seek \quad traccia-traccia}{intervallo \quad settori} \rceil$. 
Per la prima zona: $\lceil \frac{1}{8 / 160} \rceil$ = 20 settori.
Per la seconda zona: $\lceil \frac{1}{8 / 200} \rceil$ = 25 settori.
Per la terza zona: $\lceil \frac{1}{8 / 240} \rceil$ = 30 settori.
Per la quarta zona: $\lceil \frac{1}{8 / 280} \rceil$ = 35 settori.
Il massimo tasso di trasferimento dati si calcola nella zona con più settori ed è pari a $byte$ $settore/T_T$ = $\frac{512}{\frac{1}{7200/60} * 1000 * \frac{1}{280}}$ = circa 17 kilobyte/sec.}
\subsection{RAID}
Si supponga di effettuare le seguenti operazioni in sequenza su un sistema RAID 5 costituito da 5 dischi identici (inizialmente vuoti) e con blocchi (strip) da 1 byte:
\begin{enumerate}
    \item Scrittura della sequenza di byte: 01000101, 00000110, 10110100, 11101101, 11000111, 10000101, 01110111, 01010101
    \item Lettura secondo e settimo dyte di dati
    \item Modifica del terzo byte di dati da 10110100 a 01001011
\end{enumerate}
Per ogni punto, s'illustrino le operazioni compiute dal sistema, evidenziando quante READ e quante WRITE vengono effettuate, e quante di queste sono fatte in parallelo.\\
NOTA:
\begin{itemize}
    \item Per ognuno dei suddetti punti, il controller RAID riceve i comandi di scrittura/lettura di byte come un'unica richiesta.
    \item L'ordine di scrittura sui dischi è da sinistra verso destra, dall'alto verso il basso.
    \item Racchiudere ogni blocco di parità tra parentesi tonde.
    \item Il sistema RAID non è dotato di dischi hot-spare.\\
\end{itemize}
\begin{enumerate}
    \color{blue}
    \item Essendo i dischi vuoti, vengono effettuate solo operazioni di scrittura (1 per ogni byte + 2 per blocchi di parità = 10). I blocchi di parità sono settati a (00011010) e (01100000).\\
    \begin{tabular}{|c|c|c|c|c|c|}
        \hline
        Stripe & Disco 0 & Disco 1 & Disco 2 & Disco 3 & Disco 4 \\
        \hline
        0 &  01000101 & 00000110 & 10110100 & 11101101 & (00011010) \\
        \hline
        1 & 11000111 & 10000101 & 01110111 & (01100000) & 01010101 \\
        \hline
    \end{tabular}
    \item Il secondo e settimo byte sono letti in parallelo in quanto si trovano su dischi diversi. Sono effettuate 2 letture e 0 scritture.
    \item Per modificare il terzo bit, vengono effettuate 2 letture (blocco da modificare e blocco di parità) e due scritture (blocco modificato e nuovo bit di parità) in parallelo: in questo caso infatti può essere usata la parità sottrattiva.\\
    \begin{tabular}{|c|c|c|c|c|c|}
        \hline
        Stripe & Disco 0 & Disco 1 & Disco 2 & Disco 3 & Disco 4 \\
        \hline
        0 &  01000101 & 00000110 & 01001011 & 11101101 & (11100101) \\
        \hline
        1 & 11000111 & 10000101 & 01110111 & (01100000) & 01010101 \\
        \hline
    \end{tabular}
\end{enumerate}
\subsection{RAID}
Si supponga di effettuare le seguenti operazioni in sequenza su un sistema RAID costituito da 4
dischi identici (inizialmente vuoti), assumendo che il sistema operativo utilizzi blocchi di
dimensione 2 byte:
\begin{itemize}
    \item Scrittura delle sequenza di blocchi: 0100010100000110, 1011010011101101, 0111000111000011, 1000111110101100
    \item Lettura del secondo e del quarto blocco
    \item Modifica del primo e del terzo blocco in: 1011101001001011, 1000111000111100
\end{itemize}
Spiegare come vengono effettuate le suddette operazioni utilizzando le seguenti organizzazioni
RAID:
\begin{itemize}
    \item RAID di livello 0 con strip da 1 byte
    \item RAID di livello 1 con blocchi da 1 byte
\end{itemize}
Per ogni punto, s'illustrino le operazioni compiute dal sistema, evidenziando quante READ e quante
WRITE vengono effettuate, e quante di queste sono fatte in parallelo.
NOTA:
\begin{itemize}
    \item Per ognuno dei suddetti punti, il controller RAID riceve i comandi di scrittura/lettura di byte come un'unica richiesta.
    \item L'ordine di scrittura sui dischi è da sinistra verso destra, dall'alto verso il basso.
    \item Racchiudere ogni blocco di parità tra parentesi tonde.
    \item Il sistema RAID non è dotato di dischi hot-spare.\\
\end{itemize}
\textcolor{blue}{Con organizzazione RAID 0:}
\begin{enumerate}
    \color{blue}
    \item Vengono effettuate 8 scritture, delle quali 4 in parallelo (per due volte)\\
    \begin{tabular}{|c|c|c|c|c|c|}
        \hline
        Stripe & Disco 0 & Disco 1 & Disco 2 & Disco 3\\
        \hline
        0 & 01000101 & 00000110 & 10110100 & 11101101 \\
        \hline
        1 & 01110001 & 11000011 & 10001111 & 10101100 \\
        \hline
    \end{tabular}
    \item Le letture del secondo e quarto blocco avvengono sequenzialmente (4 letture, 0 scritture)
    \item Il primo e il terzo blocco sono scritti sequenzialmente (0 letture, 4 scritture)
\end{enumerate}
\textcolor{blue}{Con organizzazione RAID 1:}
\begin{enumerate}
    \color{blue}
    \item Vengono effettuate 16 scritture, delle quali 4 in parallelo (per 4 volte)\\
    \begin{tabular}{|c|c|c|c|c|c|}
        \hline
        Stripe & Disco 0 & Disco 1 & Disco 2 & Disco 3\\
        \hline
        0 & 01000101 & 00000110 & 01000101 & 00000110 \\
        \hline
        1 & 10110100 & 11101101 & 10110100 & 11101101 \\
        \hline
        2 & 01110001 & 11000011 & 01110001 & 11000011 \\
        \hline
        3 & 10001111 & 10101100 & 10001111 & 10101100 \\
        \hline
    \end{tabular}
    \item Le letture del secondo e quarto blocco avvengono in parallelo, dato che possono essere letti anche i dischi clone (4 letture, 0 scritture)
    \item Le scritture del primo e terzo blocco avvengono sequenzialmente (0 letture, 8 scritture)
\end{enumerate}
\subsection{RAID}
Si consideri un sistema RAID di livello 4 composto da N dischi (dove N comprende il disco di parità e quindi N $>$ 1) a cui arriva una richiesta di scrittura per B blocchi che interessa una sola stripe
(quindi B $<$ N). Determinare il valore di B per cui risulti più efficiente (in termini di numero totale di operazioni di I/O su disco) usare il metodo della parità additiva piuttosto che quello della parità sottrattiva.\\\\
\textcolor{blue}{La soglia per cui risulta più efficiente utilizzare il metodo della parità sottrattiva è B = $\lfloor \frac{n}{2}-1 \rfloor$.\\
Prendendo per esempio un sistema RAID4 composto da 6 dischi (5 + 1 di parità):
Parità sottrattiva (i numeri dell'elenco sono il numero di blocchi coivolti):}
\begin{enumerate}
    \color{blue}
    \item 2 letture, 2 scritture
    \item 3 letture, 3 scritture
    \item 4 letture, 4 scritture
    \item 5 letture, 5 scritture
    \item 6 letture, 6 scritture
\end{enumerate}
\textcolor{blue}{Parità addittiva:}
\begin{enumerate}
    \color{blue}
    \item 4 letture, 2 scritture
    \item 3 letture, 3 scritture
    \item 2 letture, 4 scritture
    \item 1 lettura, 5 scritture
    \item 0 letture, 6 scritture
\end{enumerate}
\subsection{Seek}
Al driver del disco arrivano, nell'ordine specificato, le richieste di cilindri 11, 23, 21, 3, 41, 7, e 39.
Ogni operazione di seek tra due cilindri consecutivi (track-to-track seek time) impiega 6 msec.
Specificare l'ordine di visita dei vari cilindri e il tempo totale di seek che si ottengono utilizzando i
seguenti algoritmi:
\begin{enumerate}
    \item First-Come, First-Served (FCFS)
    \item Shortest-Seek First (SSF)
    \item LOOK
    \item C-LOOK
\end{enumerate}
In tutti i casi, il braccio del disco è inizialmente posizionato sul cilindro 31. Per gli algoritmi LOOK
e varianti, la direzione iniziale è DOWN.\\\\
\begin{enumerate}
    \color{blue}
    \item Ordine di visita: 11, 23, 21, 3, 41, 7, 39.\\ Tempo totale di seek: (20 + 12 + 2 + 18 + 38 + 34 + 32) * 6 =  936 msec
    \item Ordine di visita: 23, 21, 11, 7, 3, 39, 41 oppure 39, 41, 23, 21, 11, 7, 3. \\Tempo totale di seek: (8 + 2 + 10 + 4 + 4 + 36 + 2) * 6 = 396 msec oppure (8 + 2 + 18 + 2 + 10 + 4 + 4) * 6 = 288 msec
    \item Ordine di visita: 23, 21, 11, 7, 3, 39, 41. \\Tempo totale di seek: (8 + 2 + 10 + 4 + 4 + 36 + 2) * 6 = 396 msec
    \item Ordine di visita: 23, 21, 11, 7, 3, 41, 39. \\Tempo totale di seek: (8 + 2 + 10 + 4 + 4 + 38 + 2) * 6 = 408 msec
\end{enumerate}
\subsection{Risparmio energetico}
Un processo richiede l'utilizzo della CPU ogni 0.5 sec. Si supponga che la CPU risponda
immediatamente alla richiesta e impieghi 100 msec per processarla. Inoltre, sia P W il consumo
elettrico della CPU e si supponga che il consumo elettrico della CPU in stato idle sia nullo (cioè, il
consumo elettrico statico è zero). Rispondere alle seguenti domande:
\begin{enumerate}
    \item Se il voltaggio V della CPU è ridotto a V/n, qual è il valore ottimale di n?
    \item Qual è il risparmio energetico che si ottiene in 1 sec, riducendo il voltaggio della CPU come al punto (1) rispetto a non ridurlo?
\end{enumerate}
\begin{enumerate}
    \color{blue}
    \item Con n = 5, la CPU è ancora in grado di processare la richiesta (ora richiede 0.5 secondi).
    \item Quando il voltaggio non è ridotto, la CPU consuma $E_{nocut} = P * 0.1 sec + 0 = 0.1 P$\\ Quando è ridotto ($P/n^2$), $E_{cut} = \frac{P}{25} * 0.5 sec + 0 = 0.02P$.\\
    Il risparmio energetico è pari a $\Delta E = E_{nocut} - E_{cut} = 0.1 - 0.02 / E_{nocut }= 0.08 / 0.1 = 80 \%$
\end{enumerate}
\subsection{Risparmio energetico}
Un processo richiede l'utilizzo della CPU ogni 0.5 sec. Si supponga che la CPU risponda
immediatamente alla richiesta e impieghi 100 msec per processarla. Inoltre, si supponga che il
consumo elettrico statico della CPU sia PS = 30 W e quello dinamico sia PD = 10 W. Rispondere alle
seguenti domande:
\begin{enumerate}
    \item Se il voltaggio V della CPU è ridotto a V/n, qual è il valore ottimale di n?
    \item Qual è il risparmio energetico che si ottiene in 1 sec, riducendo il voltaggio della CPU come al punto (1) rispetto a non ridurlo?
\end{enumerate}
\begin{enumerate}
    \color{blue}
    \item Con n = 5, la CPU è ancora in grado di processare la richiesta (ora richiede 0.5 secondi).
    \item Quando il voltaggio non è ridotto, la CPU consuma $E_{nocut} = 10 * 0.2 sec + 30*1 sec = 32 W$\\ Quando è ridotto ($P/n^2$), $E_{cut} = \frac{10}{25} * 0.5 sec + 30 = 30.2 W$.\\
    Il risparmio energetico è pari a $\Delta E = E_{nocut} - E_{cut} = 32 - 30.2 / 32 = 1.8 / 32 = 5.6 \%$
\end{enumerate}
\subsection{Risparmio energetico}
Si supponga di voler attuare una politica di risparmio del consumo energetico basata sull'utilizzo
del disco tale per cui in base a una stima $T_{est}$ dell'arrivo della prossima richiesta di utilizzo del
disco si decide se lasciare i suoi piatti in movimento (stato “active”) o arrestarne la rotazione (stato
“sleep”).
In particolare, si consideri un disco con le seguenti caratteristiche:
\begin{itemize}
    \item consumo elettrico nello stato “active” Pw=6 Watt
    \item consumo elettrico nello stato “sleep” Ps=2 Watt
    \item tempo per passare da “active” a “sleep” Tsd=10 sec
    \item energia elettrica consumata nel passaggio da “active” a “sleep” Esd=26 Joule
    \item tempo per passare da “sleep” a “active” Twu=5 sec
    \item energia elettrica consumata nel passaggio da “sleep” a “active” Ewu=60 Joule
\end{itemize}
Calcolare il valore $T_d$ tale per cui per $T_{est} > T_d$ diventi vantaggioso porre il disco da “active” a
“sleep”.\\\\
\textcolor{blue}{$T_d$ è il break-even point, ed è pari a $T_d = (E_{sd} + E_{wu} - P_s * (T_{sd} + T_{wu})) / P_w - P_s = (26+60-2*(10+5))/6-2 = (26+60-8*15)/4 = 14 sec$.\\
Quindi, conviene porre disco da "active" a "sleep" se l'intervallo tra le richieste è più di 14 secondi.}
\subsection{Interrupt}
Si considerino le seguenti situazioni e indicare quali rappresentano interrupt precisi e quali interrupt
imprecisi. Il simbolo PC denota il Program Counter. L'ordine delle istruzioni va dal basso verso
l'alto. Motivare le risposte.\\\\
\textcolor{blue}{(Riferirsi al pdf degli esercizi per le immagini)}\\
\textcolor{blue}{Le quattro proprietà dell'interrupt preciso:}
\begin{enumerate}
    \color{blue}
    \item Il PC è salvato in un luogo ben definito
    \item Tutte le istruzioni prima di quella puntata dal PC sono state completate
    \item Nessuna istruzione dopo quella puntata dal PC è stata completata
    \item Si conosce lo stato d'esecuzione dell'istruzione puntata dal PC
\end{enumerate}
\textcolor{blue}{Quindi:}
\begin{itemize}
    \color{blue}
    \item [a.] L'immagine rappresenta interrupt preciso, poichè l'istruzione prima di quella puntata dal PC è stata completata, mentre quelle successive devono ancora essere completate.
    \item [b.] L'immagine rappresenta interrupt preciso, l'istruzione puntata dal PC può essere sia già eseguita, sia in fase di esecuzione.
    \item [c.] L'immagine rappresenta interrupt impreciso, poichè l'istruzione prima di quella puntata dal PC non è stata ancora completata.
    \item [d.] L'immagine rappresenta interrupt impreciso, poichè le istruzioni successive al PC sono già state completate.
\end{itemize}
\subsection{CHS}
Si consideri uno schema di indirizzamento CHS in cui sono utilizzati 14 bit per il numero di
cilindri, 5 bit per il numero di testine, e 11 bit per il numero di settori.
Si converta l'indirizzo LBA 248209144 in notazione CHS (C,H,S).\\\\
\textcolor{blue}{Formule per tradurre da LBA a CHS (\textit{n.d.r: div è la divisione intera, mod il resto della divisione intera}):}
\begin{itemize}
    \color{blue}
    \item $C = LBA$ div $(N_h * N_s)$
    \item $H = (LBA$ div $N_s)$ mod $N_h$
    \item $S = (LBA$ mod $N_s) + 1$
\end{itemize}
\textcolor{blue}{I cilindri sono $2^{14} = 16384$, le testine sono $2^5 = 32$, i settori sono $2^{11} = 2048$. Quindi:}
\begin{itemize}
    \color{blue}
    \item $C = 248209144$ div $(32 * 2048) = 3787$
    \item $H = (248209144$ div $2048)$ mod $32 = 11$
    \item $S = (248209144$ mod $2048) + 1 = 1785$
\end{itemize}
\textcolor{blue}{Indirizzo in notazione CHS: (3787, 11, 1785).}
\subsection{Dischi}
Si consideri uno disco caratterizzato da una velocità rotazionale pari a 10000 RPM, un tempo medio
di seek pari a 5 msec, un numero medio di settori per traccia pari a 500.
Determinare:
\begin{enumerate}
    \item Latenza rotazionale media
    \item Tempo medio di trasferimento
    \item Tempo medio di posizionamento
    \item Tempo medio di accesso a un settore
\end{enumerate}
I risultati devono essere espressi in msec.\\\\
\begin{enumerate}
    \color{blue}
    \item Latenza Rotazionale media = latenza rotazionale massima / 2 = (1 / velocità di rotazione) / 2 = $\frac{1}{10000} * 60 * 1000 * \frac{1}{2}$ = 3 msec
    \item Tempo medio di trasferimento = latenza rotazionale massima / media settori per traccia = $\frac{1}{10000} * 60 * 1000 * \frac{1}{500}$ = 0.012 msec
    \item Tempo medio di posizionamento = tempo medio di seek + latenza rotazionale media = 5 + 3 = 8 msec
    \item Tempo medio di accesso a un settore = tempo medio di seek + latenza rotazionale media + tempo medio trasferimento = 5 + 3 + 0.012 = 8.012 msec
\end{enumerate}
\subsection{Sector interleaving}
Si calcoli il valore del sector interleaving per un disco caratterizzato da una velocità rotazionale pari
a 10000 RPM, un numero di settori per traccia pari a 1500, e un tempo di trasferimento di un settore
in memoria (incluso il tempo di controllo dell'ECC) pari a 40 $\mu$sec.\\\\
\textcolor{blue}{Formula per il calcolo del sector interleaving: $\lceil \frac{tempo \quad trasferimento \quad settore}{intervallo \quad settori} \rceil$.\\
Tempo intervallo settori = tempo medio trasferimento = latenza rotazionale massima / media settori per traccia = $\frac{1}{10000} * 60 * 1000 * \frac{1}{1500}$ = 0.004 msec = 4 $\mu$sec\\
Quindi: $\lceil \frac{40}{4} \rceil$ = 10 settori}
\section{File Systems}
\subsection{Path}
(Riferirsi al pdf degli esercizi per l'immagine)\\
Si supponga che si voglia accedere al file foo.txt e che la directory corrente sia “/a/d”
Dati i seguenti path:
\begin{itemize}
    \item [a.] /a/b/c/foo.txt
    \item [b.] ./foo.txt
    \item [c.] /b/../a/c/foo.txt
    \item [d.] /a/c/foo.txt
    \item [e.] /a/c/./foo.txt
    \item [f.] ../c/foo.txt
    \item [g.] /b/../a/../b/../a/d/../c/foo.txt
    \item [h.] ../../a/c/foo.txt
\end{itemize}
Rispondere alle seguenti domande:
\begin{enumerate}
    \item Quali dei suddetti path sono validi?
    \item Quali tra i path validi sono assoluti, relativi, o canonici?\\
\end{enumerate}
\textcolor{blue}{I path validi sono:}
\begin{itemize}
    \color{blue}
    \item [c.] assoluto
    \item [d.] canonico
    \item [e.] assoluto
    \item [f.] relativo
    \item [g.] assoluto
    \item [h.] relativo
\end{itemize}
\subsection{Allocazione}
Si consideri un file system installato su un disco da 1TiB. Si supponga che il file system utilizzi
blocchi di 2 KiB e indirizzi a blocchi a 64 bit, e si consideri un file “foo.txt” cui sono stati allocati i
seguenti blocchi di dati (elencati in ordine di allocazione): 1080, 1081, 1082, 7400, 7401, 7407,
12510, 22418, 22419, 22420, 23710, 1079.
\begin{itemize}
    \item [a.] Si disegni la struttura dati utilizzata per tenere traccia di tali blocchi nei seguenti casi:
    \begin{enumerate}
        \item Allocazione basata su lista concatenata di blocchi.
        \item Allocazione basata su lista concatenata con File Allocation Table (FAT; si usi -1 come codice di terminazione di lista).
        \item Allocazione indicizzata basata su i-node con 4 blocchi diretti, 1 puntatore indiretto singolo e 1 puntatore indiretto doppio (si usi -1 per indicare un campo della struttura non
        utilizzato). Il blocco contenente l'i-node ha indirizzo 512.
    \end{enumerate}
    \item [b.] Si dica qual è la dimensione massima di un file nei tre casi del punto (a).
    \item [c.] Si dica quali informazioni occorre memorizzare nella entry di directory del file nei tre casi del punto (a)\\
\end{itemize}
\begin{itemize}
    \color{blue}
    \item [b.]
    \begin{enumerate}
        \item Con lista concatenata è possibile sfruttare l'intero disco, quindi la dimensione è limitata dalla capacità 
        del disco/partizione su cui è installato il file system. Tuttavia non è possibile sfruttare l'intero spazio a 
        disposizione in quanto ogni blocco ha un overhead dovuto al puntatore al nodo successivo della lista.\\
        $S_{totale}-S_{puntatori}=2^{40}-\lfloor2^{40}/2^{11}\rfloor*8$
        \item Il limite della dimensione del file è dato dalla capacità del disco/partizione dove è installato il file 
        system. Non è possibile sfruttare l'intero spazio a disposizione in quanto parte dello spazio è riservato per la 
        FAT.\\
        $S_{totale}-S_{FAT}=2^{40}-\lfloor2^{40}/2^{11}\rfloor*8$
        \item Il limite è dato da:\\
        $N_{pDiretti}+N_{pIndiretti}+N_{pIndiretti}^2*D_{blocco}=4+\lfloor2^{11}/2^3\rfloor+\lfloor2^{11}/2^3\rfloor^2*2=131592KiB$
    \end{enumerate}
\end{itemize}
\begin{itemize}
    \color{blue}
    \item [c.] Per gli schemi di allocazione basati su lista basta memorizzare, oltre al nome del file, il puntatore al 
    primo blocco.\\
    Per lo schema indicizzato con i-node, è sufficiente memorizzare, oltre al nome del file, il puntatore al blocco contenente 
    l'i-node.
\end{itemize}
\subsection{\textit{fsck}}
Si consideri la procedura fsck per la verifica della consistenza nei file system Unix e si esamini il contenuto delle 
seguenti tabelle:\\
\begin{center}
    \begin{tabular}{c c|c|c|c|c|c|c|c|c|c }
        & 0 & 1 & 2 & 3 & 4 & 5 & 6 & 7 & 8 & 9 \\
       \hline
       Blocchi in uso & 1 & 0 & 1 & 0 & 1 & 3 & 0 & 0 & 1 & 0 \\
       \hline
       Blocchi liberi & 0 & 2 & 0 & 1 & 0 & 0 & 1 & 1 & 0 & 0 \\
       \hline
   \end{tabular}
\end{center}
Si indichi quali sono le situazioni d'inconsistenza (se ce ne sono) e, per ognuna di esse, si dica come
fcsk riporta il file system in uno stato consistente.
Si assuma che il file system tenga traccia dei blocchi liberi tramite una lista concatenata.\\\\
\textcolor{blue}{I tipi di inconsistenza sono:}
\begin{enumerate}
    \color{blue}
    \item blocco con contatore a 0 in entrambe le tabelle, \textit{fsck} lo aggiunge alla tabella dei blocchi liberi
    \item blocco con contatore a 1 in entrambe le tabelle, \textit{fsck} può rimuoverlo da uno delle dei tabelle
    \item blocco con contatore $>$ 1 nella tabella dei blocchi liberi, \textit{fsck} ricostruisce la tabella dei blocchi
    liberi per eliminare l'inconsistenza
    \item blocco con contatore $>$ 1 nella tabella dei blocchi in uso, \textit{fsck} alloca un blocco libero, copia il
    contenuto del blocco duplicato in quel blocco e inserisce la copia in uno dei file
\end{enumerate}
\textcolor{blue}{In questo caso:} 
\begin{enumerate}
    \color{blue}
    \item la voce ad indice 9
    \item (nessuna inconsistenza di questo tipo)
    \item la voce ad indice 1
    \item la voce ad indice 5
\end{enumerate}
\subsection{Cache}
Si consideri la cache di un file system che utilizza la tecnica di hashing per velocizzare il
reperimento dei blocchi in essa presenti, e la politica LRU per il rimpiazzamento dei blocchi. Si
ipotizzi inoltre che tale cache preveda l'uso di 8 “slot” per la tabella di hash, indirizzati da 0 a 7, e
che la funzione di hashing utilizzata sia h(B) = B mod 8, dove B è l'indirizzo del blocco da
memorizzare nella cache.
\begin{itemize}
    \item Si disegni lo stato della cache dopo l'inserimento dei seguenti blocchi: 4, 20, 44, 244,
    9,17,41, 57, 159, 231.
    \item Partendo dal punto precedente, si disegni lo stato della cache dopo l'inserimento del blocco
    347.
\end{itemize}
Si supponga che:
\begin{itemize}
    \item la dimensione massima della cache sia di 10 blocchi,
    \item l'ordine di arrivo dei blocchi sia da sinistra verso destra (ad es., nel punto (a), il primo
    blocco ad arrivare è il blocco 4),
    \item la cache sia inizialmente vuota.
\end{itemize}
Nei diagrammi, indicare quale elemento rappresenta l'elemento LRU e quale quello MRU.\\\\
\textcolor{blue}{Valori hash: 4, 4, 4, 4, 1, 1, 1, 1, 7, 7}
\subsection{I-node}
Si consideri un disco di 512 GiB e un file system installato su di esso che impieghi gli i-node come
metodo di allocazione di blocchi ai file e una bitmap per la gestione dei blocchi liberi.
Supponendo di utilizzare blocchi di 4 KiB ed indirizzi a 32 bit, si risponda alle seguenti domande:
\begin{itemize}
    \item Qual è la dimensione massima di un file in tale file system se gli i-node includono 16
    puntatori diretti, un puntatore indiretto singolo, 2 puntatori indiretti doppi e 2 puntatori
    indiretti tripli?
    \item Qual è la dimensione della bitmap per la gestione dei blocchi liberi? Per semplicità, si
    assuma che l'intero disco possa essere utilizzato per memorizzare blocchi di dati.
\end{itemize}
\textcolor{blue}{
    Dimensione disco = $2^{39}$\\
    Dimensione blocco = $2^{12}$\\
    Numero blocchi = $\lfloor2^{39}/2^{12}\rfloor=2^{27}$\\
    Numero indirizzi per blocco = $\lfloor2^{12}/2^2\rfloor=2^{10}=1024$\\
    Numero massimo blocchi indirizzabili = $N_{pDiretti}+N_{pIndiretti}+N_{pIndiretti}^2+N_{pIndiretti}^3=16+1024+1024^2*2+1024^3*2$\\
    Dimensione bitmap = $\lfloor2^{39}/2^{12}\rfloor=2^{27}bit$
}
\subsection{Formattazione}
Si vuole formattare un disco da 512 GiB con un particolare file system usando blocchi da 4 KiB.
Determinare quali delle seguenti scelte progettuali sono valide per rappresentare tutti i blocchi del
disco:
\begin{itemize}
    \item [a.]Indirizzi di blocchi a 64 bit
    \item [b.]Indirizzi di blocchi a 16 bit
\end{itemize}
Limitandosi alle scelte progettuali corrette e supponendo di dover memorizzare un file composto da
32 KiB di dati, determinare overhead e spazio sprecato nel caso si utilizzi:
\begin{itemize}
    \item [a.]l'allocazione basata su FAT,
    \item [b.]l'allocazione indicizzata con blocco indice da 4 KiB,
    \item [c.]l'allocazione basata su lista concatenata con cluster da 16 KiB.
\end{itemize}
\textcolor{blue}{
    Il numero di blocchi totale è $2^{27}$, quindi non è possibile rappresentarli con indirizzi a 16 bit, servono indirizzi 
    a 64 bit.
}
\begin{itemize}
    \color{blue}
    \item [a.]Overhead = $(D_{FAT}/D_{disco})\%=(2^{30}/2^{39})\%=0.19\%$, spazio sprecato = $(1-(D_{file}/(N_{blocchi}*D_{blocco})))=0\%$
    \item [b.]Overhead = $(N_{blocchiIndexBlock}/N_{blocchiFile})\%=(1/9)\%=11\%$, spazio sprecato = $(1-(2^3*2^3+2^{15})/(9*2^{12}))\%=11\%$
    \item [c.]Overhead = $(2^3/(2^2*2^{12}))\%=0.05\%$, spazio sprecato = $(1-(2*2^3+2^{12})/(2^{15}*2^{12}))\%=0\%$
\end{itemize}
\end{document}

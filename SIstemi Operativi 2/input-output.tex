\documentclass[12pt]{article}
\usepackage[margin=0.8in]{geometry}
\usepackage{xcolor}

\title{Input/Output}

\begin{document}
\begin{center}
    \huge\textbf{File System}
\end{center}
I dispositivi di I/O sono componenti hardware mirati a fornire input e direzionare output; il SO deve gestire e controllare 
tutti i dispositivi e le loro operazioni.\\
I dispositivi possono essere divisi in due categorie: \textbf{device a blocchi} e \textbf{device a caratteri}. I primi 
operano su blocchi di dati di dimensione prefissata e l'accesso è random; i secondi ricevono o inviano flussi di byte e 
l'accesso e sequenziale. I dispositivi di rete non rientrano in nessuna delle due categorie.
\section{Hardware}
Un dispositivo generalmente consiste di due parti: il \textbf{dispositivo} stesso e il \textbf{controllore}. Il dispositivo 
comunica con il computer tramite il suo controller attraverso un punto di connessione, chiamato \textbf{porta}.
\subsection{Controllori}
Il controllore (o adattatore) è un chip o un insieme di chip che controllano fisicamente il dispositivo. Il controllore 
accetta comandi dal SO e li esegue. L'interfaccia tra il controllore e il dispositivo è standard ma è di basso livello; 
il controllore presenta un'interfaccia più semplice nascondendo i dettagli del dispositivo.\\
La CPU e i controllori comunicano tra loro sul bus di sistema. I computer moderni hanno multipli bus che definiscono 
diversi protocolli di comunicazione. Un controllore può essere un singolo chip nella scheda madre del computer oppure 
un circuito stampato che si inserisce nel computer tramite uno slot d'espansione. Alcuni dispositivi hanno i loro controllore
(\textbf{controllori periferici}), che permettono al controllore del computer di inviare comandi ad alto livello al controllore 
periferico.
\subsection{Interazione tra CPU e dispositivi}
Nel modello di riferimento di sistema, CPU, memoria e dispositivi sono connessi su un singolo bus e comunicano tra di loro 
attraverso questo. Nel modello di riferimento del dispositivo, l'interfaccia è l'interfaccia che il controllore del dispositivo 
presenta al resto del sistema; la struttura interna varia da implementazione a implementazione ed implementa l'astrazione 
che il controllore presenta.\\
Il controllore dei dispositivo ha dei \textbf{registri di controllo} che vengono acceduti dalla CPU e che permettono l'interazione 
CPU-dispositivo: i \textbf{data registers} sono usati dalla CPU per controllare i trasferimenti di dati dal e al dispositivo, 
gli \textbf{status registers} sono usati dalla CPU per controllare lo stato del dipositivo, i \textbf{command registers} 
possono essere scritti dalla CPU per controllare cosa il dispositivo farà. Inoltre, molti dispositivi hanno un \textbf{buffer dati}, 
usato per memorizzare dati che non possono essere passati direttamente al SO o dati che stanno per essere scritti sul 
dispositivo.
\subsection{Port-mapped, memory-mapped}
Sono utilizzati due approcci per garantire l'interazione tra CPU e dipositivo: \textbf{I/O mappato a porte} e \textbf{I/O mappato a memoria}.
I due approcci possono essere combinati.\\
COn I/O mappato a porta, ad ogni registro del dispositivo è assegnato un \textbf{numero di porta I/O}, l'insieme dei quali forma l'\textbf{I/O port 
space}, accessibile solo in modalità kernel. Vengono usate istruzioni speciali e privilegiate per comunicare. Con questo 
approccio, non viene consumata memoria, non bisogna gestire il caching e non vanno prese misure speciali in caso di bus 
multipli.\\
Con I/O mappato a memoria, i registri del dispositivo sono mappati nella memoria, e ad ogni registro è assegnato un indirizzo 
di memoria, di solito a tempo di boot. Le istruzioni sono istruzioni di trasferimento dati standard. Con questo approccio, 
lo sviluppo di driver è semplificato (possono essere sviluppati in C), non è richiesto nessun meccanismo di protezione 
addizionale per impedire ai processi utente di eseguire I/O a basso livello ed il design delle CPU può essere semplificato 
(ogni istruzione che può referenziare la memoria può anche referenziare i registri del dispositivo).
\subsubsection{PIO, Interrupt-Driven}
Ci sono tre approcci per eseguire le operazioni I/O: \textbf{I/O programmato}, \textbf{I/O guidato da interrupt} e 
\textbf{I/O tramite DMA}.\\
Con \textbf{I/O programmato} la CPU richiede un'operazione di I/O al controllore e ne attende l'esecuzione (\textbf{busy waiting}).\\
Con \textbf{I/O guidato da interrupt}, la CPU chiede al controllore di eseguire una richiesta I/O, blocca il processo che 
l'ha richiesta ed esegue un'altra operazione. Quando il controllore individua la fine di una richiesta I/O, solleva un 
interrupt per segnalarne il completamento, asserendo un segnale sul bus. Il segnale è gestito dal \textbf{controllore degli interrupt}. 
Il controllore degli interrupt ignora l'interrupt se interrupt con maggiore priorità sono in attesa. Quando è pronto, il 
controllore gestisce l'interrupt mettendo un numero sul bus specificando quale dispositivo richiede attenzione ed inviando 
un interrupt alla CPU. Questo numero è un indice in una tabella chiamata \textbf{interrupt vector}: ogni cella contiene 
l'indirizzo di memoria che punta all'inizio di un interrupt handler. Quando l'interrupt controller prende in gestione 
l'interrupt, la CPU serve la richiesta di interrupt, caricando il nuovo PC dall'interrupt vector e mandando un segnale 
all'interrupt controller. Ora la CPU è pronta a gestire un nuovo interrupt. Quando l'interrupt handler ha finito, la CPU 
riprende l'esecuzione del processo che era stato interrotto.\\
\subsubsection{Interrupt precisi e imprecisi}
Essendo le CPU moderne in grado di eseguire più istruzioni contemporaneamente, quando avviene un interrupt non tutte le 
istruzioni potrebbero essere state completate in tempo. Un \textbf{interrupt preciso} è un interrupt che lascia la macchina 
in uno stato ben definito: tutte le istruzioni prima del PC sono state eseguite e nessuna istruzione dopo il PC è stata 
portata a termine (è permesso che l'istruzione puntata dal PC sia stata eseguita o meno). Se una di queste proprietà viene 
violata, si ha un \textbf{interrupt impreciso}.
\subsubsection{DMA}
Con \textbf{I/O tramite DMA}, la CPU si appoggia ad un processore speciale detto \textbf{controllore DMA}: il controllore 
DMA gestisce i trasferimenti di dati tra memoria e dispositivi usando I/O programmato, inviando un interrupt alla CPU al 
termine delle operazioni. Il DMA contiene registri che possono essere letti e scritti dalla CPU. Lo schema di trasferimetno 
usato dal DMA è detto \textbf{fly-by mode}: i dati non passano dal controllore del DMA, ma sono eseguiti dal controllore 
del dispositivo, il controllore del DMA comunica al controllore del dispositivo quali operazioni deve svolgere. Un approccio 
alternativo è il \textbf{flow-through mode}: i dati passano dal controllore del DMA, dove sono memorizzati in un buffer 
interno al DMA, ma richiede due cicli di bus invece di uno.
\subsection{Cycle-stealing, burst-mode}
Il controllore del DMA e la CPU si contendono l'accesso alla memoria. Ci sono due approcci: \textbf{cycle-stealing} e 
\textbf{burst mode}.\\
Con approccio \textbf{cycle-stealing}, per ogni parola da trasferire il controllore DMA deve prima acquisire il bus e rilasciarlo 
quando il trasferimento è completato: avendo il DMA priorità sul bus superiore alla CPU, il controllore occasionalmente 
ruba cicli di bus alla CPU.\\
Con approccio \textbf{burst mode}, il controllore DMA comunica al dispositivo di ottenere il bus, iniziare una serie di 
trasferimenti e poi rilasciarlo: è un approccio più efficiente ma può bloccare la CPU ed altri dispositivi se il burst è 
molto lungo.\\
I controllori DMA possono controllare più trasferimenti dati allo stesso momento.
\section{Software}
\subsection{Interazione CPU-SO}
Ci sono tre approcci: \textbf{I/O programmato}, \textbf{I/O guidato da interrupt} e \textbf{I/O tramite DMA}.\\
Con \textbf{I/O programmato}: il SO legge/scrive il dispositivo e continua ad interrogarlo per vedere se ha terminato una 
operazione (\textbf{busy waiting}).\\
Con \textbf{I/O guidato da interrupt}: il SO invia una richiesta I/O e blocca il processo che l'ha richiesta, gestendo 
nel mentre altri processi, finchè non riceve un interrupt dal dispositivo. Quando tutte le operazioni sono finite, il SO 
sblocca il processo.\\
Con \textbf{I/O tramite DMA}, il controllore del DMA serve la richiesta senza appoggiarsi alla CPU (è come PIO ma svolta 
dal controllore del DMA al posto della CPU).
\subsection{Obiettivi del software I/O}
Gli obiettivi del software I/O è garantire:
\begin{itemize}
    \item \textbf{indipendenza dei dispositivi}: poter scrivere programmi che possono accedere a qualsiasi dispositivo
    \item \textbf{denominazione uniforme}: il nome di un file o dispositivo non dovrebbe dipendere dal dispositivo 
    \item \textbf{gestione degli errori trasparente}: gli errori dovrebbero essere gestiti vicino all'HW 
    \item \textbf{supporto a più modalità di trasferimento}: la maggior parte dell'I/O fisico è \textbf{asincrono} (guidato 
    da interrupt), ma i programmi sono più facili da sviluppare se le operazioni di scrittura sono \textbf{sincrone} (bloccanti), 
    il SO dovrebbe far sembrare operazioni asincrone come sincrone ai programmi
    \item \textbf{gestione del del buffering}: a volte i dati in uscita dal dispositivo non possono essere gestiti direttamente 
    e vanno memorizzati in un buffer
    \item \textbf{gestione dei dispositivi condivisi e dedicati}
\end{itemize} 
\subsection{Gestori degli interrupt}
Quando avviene un interrupt, il \textbf{gestore degli interrupt} associato deve gestirlo. Per prima cosa, il SO salva il 
contesto del processo corrente; viene impostato il contesto e lo stack del gestore degli interrupt; il SO riscontra il 
controllore degli interrupt ed esegue il gestore degli interrupt finchè questo non esegue l'istruzione di ritorno dall'interrupt; 
infine il SO schedula un altro processo e lo esegue.
\subsection{Driver dei dispositivi}
I \textbf{driver} nascondono i dettagli per il controllo dei dispositivi agli strati superiori. Ogni driver di solito gestisce 
un tipo di dispositivo o classe di dispositivi, e sono di solito sviluppati dal produttore. I driver forniscono i gestori 
degli interrupt e sono eseguiti in modalità kernel: devono accedere i registri del controllore posizionati sotto il resto 
del SO (driver in modalità utente non sono largamente utilizzati). Struttura generale di un driver:
\begin{enumerate}
    \item Accettare la richiesta I/O astratta dagli strati superiori
    \item Controllare la validità dei parametri in input, e ritornare errore se non sono validi 
    \item Controllare se il dipositivo è in uso, iniziando le operazioni se non lo è
    \item Inviare una sequenza di comandi a basso livello al dispositivo
    \item Quando i comandi sono stati inviati: se PIO, attendere il completamento dell'operazione; se guidato da interrupt 
    bloccare fino all'arrivo di un interrupt
    \item Alla fine dell'operazione, controllare se si sono verificati errori 
    \item Ritornare informazioni di stato e errori 
    \item Selezionare la richiesta successiva o attendere 
\end{enumerate}
Un driver deve essere \textbf{rientrante}, ossia deve poter essere richiamato anche dopo essere stato interrotto. Inoltre 
devono essere capaci di gestire rimozione o aggiunta di dispositivi a runtime.
\subsection{Software indipendente dal dispositivo}
Il software deve includere funzioni che siano comuni a tutti i dispositivi e fornire un'interfaccia uniforme per tutti i 
livelli.\\
L'interfaccia uniforme tra il SO e i driver è realizzata tramite una API e uno schema di denominazione uniforme. Per l'API, 
il SO definisce per ogni classe di dispositivi le funzioni che il driver deve fornire. In UNIX, i dispositivi sono acceduti 
tramite nomi nel filesystem (\textbf{file speciali} o \textbf{file dispositivo}): ogni file speciale ha un \textbf{major device number} 
(per selezionare il driver appropriato) ed un \textbf{minor device number} (per selezionare il dispositivo I/O).\\
Il buffering è un problema sia per dispositivi a blocchi sia per dispositivi a caratteri. Senza buffering si hanno troppi 
interrupt; con buffering in spazio utente si hanno problemi legati alla paginazione; con buffering in spazio kernel, il 
buffer viene copiato in spazio utente quando è pieno (operazione atomica), ma a volte un buffer non è abbastanza quindi 
se ne usano 2 oppure buffering circolare.\\
Gli errori sono molto comuni nel contesto dell'I/O. Gli errori sono specifici al dispositivo ma il framework per gestire 
gli errori è indipendente dai dispositivi. Gli \textbf{errori di programmazione} si verificano quando il chiamante richiede 
operazioni o parametri non validi. Gli \textbf{errori di I/O} si verificano quando qualcosa va storto durante la richiesta 
I/O.\\
Alcuni dispositivi sono dedicati (possono essere usati da un solo processo alla volta) e il SO li deve gestire per evitare 
deadlock.
\subsection{I/O nello spazio utente}
La maggior parte del SW di I/O viene eseguito in modalità kernel. Le chiamate di sistema possono essere invocate via funzioni 
di libreria. I programmi sono detti \textbf{daemon} e sono spesso associati ad un \textbf{sistema spooler}.
\section{Dischi}
\subsection{Dischi magnetici}
Ogni \textbf{superficie} di un piatto è divisa logicamente in \textbf{tracce} circolari, che sono suddivise in \textbf{settori}. 
Ogni settore contiene lo stesso numero di bit di dati; i settori sono separati da spazi senza bit di dati, che servono ad 
identificare i settori; l'insieme delle tracce che si trovano sotto un braccio costituisce un \textbf{cilindro}.\\
Il controllore del disco può scrivere/leggere dati solo a multipli di settore e leggere/scrivere tutti i dati in una traccia 
senza muovere il braccio del disco. I controllori permettono \textbf{seek sovrapposti}, ma il trasferimento dei dati tra 
controllore e memoria principale non può sovrapporsi.\\
Nei dischi moderni, la geometria virtuale è diversa dalla geometria fisica: la lunghezza fisica della traccia aumenta 
all'aumentare della distanza dal centro, quindi i dischi contengono più settori nelle zone esterne che interne 
(\textbf{Zone Bit Recording}). Per nascondere questo al SO, i controllori presentano una geometria virtuale (LBA$\leftrightarrow$CHS). 
\subsection{Performance}
La performance del disco è caratterizzata da:
\begin{itemize}
    \item \textbf{tempo di seek}: tempo per posizionare il braccio del disco sul cilindro desiderato
    \item \textbf{latenza rotazionale}: tempo impegato dal settore desiderato per ruotare sotto la testina
    \item \textbf{velocità rotazionale}: la velocità con cui il motore fa ruotare il disco
    \item \textbf{tasso di trasferimento dei dati}: il tasso in bit/sec al quale i dati scorrono dalla superficie del disco 
    al controllore
    \item \textbf{tempo di posizionamento}: somma del tempo di seek e latenza rotazionale
    \item \textbf{tempo d'accesso}: somma del tempo di posizionamento e tempo di trasferimento
\end{itemize}
\subsection{Affidabilità}
In caso di \textbf{fallimento totale del disco}, il disco non può svolgere ulteriori operazioni e deve essere rimpiazzato. 
Il \textbf{tasso di fallimento annuale} è il numero di fallimenti ogni anno e il \textbf{tasso di fallimento orario} è il 
numero di fallimenti ogni ora.\\
La \textbf{disponibilità} è la frazione di tempo il sistema è disponibile a servire le richieste degli utenti.\\
Il tasso di fallimento segue il modello a vasca: i fallimenti sono più frequenti all'inizio ed alla fine del tempo di servizio.
\subsection{Formattazione a basso livello}
Prima che un disco possa essere usato per memorizzare dati, deve ricevere una \textbf{formattazione a basso livello}, che 
definisce la geometria fisica del disco. Ogni settore ha i seguenti campi: \textbf{preambolo}, che identifica l'inizio di 
un settore, \textbf{dati} e \textbf{error correcting codes}. Quando un settore viene scritto, il controllore del disco 
aggioran l'ECC con un valore calcolato da tutti i byte del campo dati; quando un settore viene letto, il controllore ricalcola 
l'ECC e lo compara al valore in memoria.\\
Le tracce non sono numerate partendo dalla stessa posizione, ma da posizioni diverse: questo è chiamato \textbf{cylinder skew}.\\
Dopo la formattazione a basso livello, la capacità del disco viene ridotta. Inoltre, impone un tasso masimo di trasferimento 
dati.\\
Per ovviare al problema della dimensione del buffer, si usa una tecnica chiamata \textbf{sector interleaving}, oppure un 
buffer molto capiente, che possa memorizzare un'intera traccia.\\
Al termine della formattazione a basso livello, il disco viene \textbf{partizionato}. Logicamente, una partizione è come 
un disco separato; il primo settore di ogni partizione e la sua dimensione sono memorizzate nella \textbf{partition table}, 
e la partition table è memorizzata nel \textbf{Master Boot Record} (MBR). La MBR è memorizzata nel primo settore del disco, che 
contiene inoltre il \textbf{boot loader}. La partition table del MBR supporta fino a 4 \textbf{partizioni primarie}: una 
e solo una può essere la \textbf{partizione estesa}, che può contenere più \textbf{partizioni logiche}. Ogni partizione 
logica viene descritta della \textbf{Extended Boot Record} (EBR). MBR sta venendo soppiantato dalla \textbf{GUID Partition Table} 
(GPT).
\subsection{Scheduling del braccio del disco}
Le richieste di I/O in attesa di essere servite sono gestite tramite diversi algoritmi di scheduling:
\begin{itemize}
    \item \textbf{First Come First Served} (FCFS): le richieste sono servite nell'ordine in cui arrivano, non è molto veloce 
    \item \textbf{Shortest Seek First} (SSF): seleziona la richiesta con il tempo di seek minore dalla posizione attuale, 
    ma può indurre starvation
    \item \textbf{LOOK}: il braccio si muove in una direzione e serve tutte le richieste, poi cambia direzione e serve tutte 
    le altre 
    \item \textbf{C-LOOK}: il braccio serve richieste solo in una direzione 
\end{itemize}
\subsection{Gestione degli errori}
I difetti di produzione sono inevitabili e possono introdurre \textbf{bad sectors}, settori che non ritornano i valori che 
gli sono appena stati scritti. Due approcci per gestire i bad sector: a livello del controllore e a livello del SO.\\
Il controllore del disco può usare dei \textbf{settori di riserva} per rimpiazzare bad sectors, tramite le tecniche di 
\textbf{sector forwarding} (il controllore rimappa uno dei settori di riserva) e \textbf{sector slipping} (il controllore 
muove tutti i settori).\\
Non tutti i controllori sono in grado si rimappare in maniera trasparente i settori, quindi il SO ottiene la lista dei bad 
sectors per far in modo che non vengano utilizzati.\\
Possono anche verificarsi errori di seek, causati da problemi meccanici del braccio, tipicamente risolti da una ricalibrazione.
\section{Interfacce utente}
Le interfacce utente includono SW di input e output, che collaborano tra loro.
\subsection{SW della tastiera}
La tastiera contiene un microprocessore che comunica con un chip controllore sulla scheda madre. Un interrupt viene generato 
ogni volta che un tasto viene premuto o rilasciato. Il numero nel registro di I/O è il numero del tasto, detto \textbf{scan code}, 
che consiste di un singolo byte (7 bit per identificare il tasto, 1 per tasto premuto/rilasciato). Il driver della tastiera 
converte lo scan code nel character code del \textbf{character set} usato dal SO.\\
I driver della tastiera possono operare in due modi: \textbf{noncanonical mode} e \textbf{canonical mode}. Il noncanonical 
mode è orientato a caratteri, mentre il canonical mode è orientato a linee.
\subsection{SW del mouse}
I mice sono più semplici delle tastiere. Ogniqualvolta il mouse viene mosso o un bottone viene premuto, viene inviato un 
interrupt al computer. La distanza minima è un \textbf{mickey} (0.1 mm). Il messaggio contiene la variazione della posizione 
sugli assi x e y ed eventuali bottoni premuti.
\section{Gestione dell'energia}
Per ridurre il consumo energetico, il SO può disattivare parti del computer non utilizzate: il SO deve decidere quale dispositivo 
spegnere e quando, per evitare sprechi e ritardi nell'utilizzo.\\
Per quanto riguarda gli hard disk, il break-even point è il lasso di tempo per il quale l'energia consumata per rallentare 
il disco, tenerlo in questo stato e poi acceleralo è uguale all'energia consumata per tenere il disco ad una velocità costante.
Il SO può fare prefetching dei blocchi dal disco in una memoria cache, in modo da non dover riattivare il disco se il blocco 
è in cache.\\
Per ridurre il consumo della CPU di possono ridurre tensione e/o la frequenza (DVFS). Le CPU moderne hanno più \textbf{performance 
states} per il DVFS: ogni P-state è denotato da un Px, dove x è un numero da 0 ad un determinato massimo; valori più elevati 
di x garantiscono un livello di risparmio energetico superiore. Inoltre, le CPU hanno più \textbf{power states}: livelli 
più elevati indicano minore operatività della CPU.
\end{document}
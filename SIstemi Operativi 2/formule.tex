\documentclass[10pt]{article}
\usepackage[margin=0.8in]{geometry}

\title{Formule}

\begin{document}
\begin{center}
    \large\textbf{Formule}
\end{center}
\section{I/O}
\subsection*{Conversione LBA/CHS}
\begin{itemize}
    \item Siano:
    \begin{itemize}
        \item $N_c$ = numero cilindri
        \item $N_h$ = numero testine
        \item $N_s$ = numero settori
    \end{itemize}
    \item $LBA=(C*N_h+H)*N_s+(S-1)$
    \item $C=LBA/(N_h*N_s)$
    \item $H=(LBA/N_s)\%N_h$
    \item $S=(LBA\%N_s)+1$
\end{itemize}
\subsection*{Performance dischi}
\begin{itemize}
    \item Latenza rotazionale massima (in secondi) = 60 sec/RPM
    \item Latenza rotazionale media = latenza rotazionale massima/2
    \item Tempo di trasferimento medio = latenza rotazionale massima/media settori per traccia
    \item Tempo medio d'accesso = tempo medio di seek + latenza rotazionale media + tempo di trasferimento medio
\end{itemize}
\subsection*{Affidabilità dischi}
\begin{itemize}
    \item $MTTF_y=1/AFR$
    \item $MTTF_h=1/HFR=24*365*MTTF_y$
\end{itemize}
\subsection*{Formattazione a basso livello}
Cylinder skew = $\lceil$track-to-track seek time/sector interarrival time$\rceil$\\
Maximum data rate = track size/max rotational latency\\
Sector interleaving = $\lceil$in-memory sector transfer time/sector interarrival time$\rceil$
\subsection*{Hard Disk Power Management}
Break-even condition: $E_{sd}+P_s*(T_d-T_{sd}-T_{wu})+E_{wu}=P_w*T_d$\\
Break-even point: $T_d=(E_{sd}+E_{wu}-P_s*(T_{sd}+T_{wu}))/(P_w-P_s)$
\subsection*{CPU Power Management}
\begin{itemize}
    \item $P=P_{dynamic}+P_{static}$
    \item $P=k*V^2*f+P_{static}$
    \item $E_{nocut}=P_{dynamic}*T_{utilizzo}+P_{static}*T_{totale}$
    \item $E_{cut}=(P_{dynamic}/n^2)*T_{utilizzo}+P_{static}*T_{totale}$
    \item $\Delta E\%=100*\Delta E/E_{nocut}$
\end{itemize}
\section{File system}
\subsection*{Allocazione contigua}
\begin{itemize}
    \item Siano:
    \begin{itemize}
        \item F' = dimensione stimata del file
        \item F = dimensione attuale del file
        \item B = dimensione blocco
    \end{itemize}
    \item Numero dei blocchi allocati = N = $\lceil F'/B \rceil$
    \item Overhead = 0\%
    \item Percentuale spazio sprecato = (1-F/(N*B))\%
\end{itemize}
\subsection*{Allocazione basata su extents}
\begin{itemize}
    \item Siano:
    \begin{itemize}
        \item F = dimensione del file
        \item B = dimensione blocco
        \item P = dimensione puntatore a blocco
        \item E = numero di blocchi dell'extent
    \end{itemize}
    \item Numero degli extent allocati = M = $\lceil F/(E*B) \rceil$
    \item Numero di blocchi per allocare la tavola degli extent = T = $\lceil M*P/B \rceil$
    \item Numero di blocchi allocati = N = M*E + T
    \item Overhead = ((T*B)/(N*B))\% = (T/N)\%
    \item Percentuale spazio sprecato = (1-(M*P+F)/(N*B))\%
\end{itemize}
\subsection*{Allocazione a liste concatenate}
\begin{itemize}
    \item Siano:
    \begin{itemize}
        \item F = dimensione del file
        \item B = dimensione blocco
        \item P = dimensione puntatore a blocco
    \end{itemize}
    \item Numero di blocchi allocati = N = $\lceil F/(B-P) \rceil$
    \item Overhead = ((N*P)/(N*B))\% = (P/B)\%
    \item Percentuale spazio sprecato = (1-(N*P+F)/(N*B))\%
\end{itemize}
\subsection*{Allocazione a cluster}
\begin{itemize}
    \item Siano:
    \begin{itemize}
        \item F = dimensione del file
        \item B = dimensione blocco
        \item P = dimensione puntatore a blocco
        \item C = numero di blocchi per cluster
    \end{itemize}
    \item Numero di cluster allocati = M = $\lceil F/(C*B-P) \rceil$
    \item Numero di blocchi allocati = N = M*C
    \item Overhead = ((M*P)/(N*B))\% = (P/(C*B))\%
    \item Percentuale spazio sprecato = (1-(M*P+F)/(N*B))\%
\end{itemize}
\subsection*{Allocazione con FAT}
\begin{itemize}
    \item Siano:
    \begin{itemize}
        \item D = dimensioni partizione
        \item F = dimensione del file
        \item B = dimensione blocco
        \item P = dimensione puntatore a blocco
    \end{itemize}
    \item Numero entries nella FAT = M = $\lfloor D/B \rfloor$
    \item Dimensione FAT = S = M*P
    \item Numero di blocchi allocati = N = $\lceil F/B \rceil$
    \item Overhead = (S/D)\%
    \item Percentuale spazio sprecato = (1-(F/(N*B)))\%
\end{itemize}
\subsection*{Allocazione con index block}
\begin{itemize}
    \item Siano:
    \begin{itemize}
        \item F = dimensione del file
        \item B = dimensione blocco
        \item P = dimensione puntatore a blocco
        \item I = numero di blocchi allocati per l'index block
    \end{itemize}
    \item Dimensione index block = S = I*B
    \item Numero di blocchi allocati (solo per i dati) = M = $\lceil F/B \rceil$
    \item Numero di blocchi allocati per file = N = I+M
    \item Overhead = (S/(N*B))\% = (I/N)\%
    \item Percentuale spazio sprecato = (1-(P*M+F)/(N*B))\%
\end{itemize}
\subsection*{Allocazione con i-node}
\begin{itemize}
    \item Siano:
    \begin{itemize}
        \item N = puntatori diretti 
        \item M = puntatori indiretti
        \item L = lunghezza index-block
        \item B = dimensioni index block
    \end{itemize}
    \item Numero massimo di blocchi indirizzabili = A = $N+\sum_{i = 1}^{M}L^{i}$  
    \item Dimensione file massima = S = A*B
\end{itemize}
\subsection*{Blocchi liberi}
Con lista collegata:
\begin{itemize}
    \item Numero di puntatori a blocco per nodo = $\lfloor$dimensione blocco/dimensione puntatore$\rfloor-1$
    \item Numero di blocchi nel disco = $\lfloor$dimensione disco/dimensione blocco$\rfloor$
    \item Numero massimo di nodi nella lista = $\lceil$blocchi su disco/numero di puntatori$\rceil$
    \item Dimensione massima lista = numero nodi * dimensione blocco
\end{itemize}
Con bitmap:
\begin{itemize}
    \item Numero di blocchi su disco = $\lfloor$dimensione disco/dimensione blocco$\rfloor$
    \item Dimensione bitmap = numero di blocchi di disco*8
    \item Numero di blocchi per allocare bitmap = $\lceil$dimensione bitmap/dimensione blocco$\rceil$
\end{itemize}
\end{document}
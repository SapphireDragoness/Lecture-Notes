\documentclass[11pt]{book}
\usepackage[margin=.8in]{geometry}
\usepackage[italian]{babel}
\usepackage{tikz}
\usepackage{amsfonts} 
\usepackage{amsthm}
\usepackage{amsmath}
\usepackage{thmtools}
\usepackage{float}
\usepackage{hyperref}

\title{Appunti Analisi Matematica}

\newtheorem*{theorem}{Teorema}
\newtheorem*{proprietà}{Proprietà}
\newtheorem*{lemma}{Lemma}
\newtheorem*{corollary}{Corollario}
\newtheorem*{definition}{Definizione}

\hypersetup{
    colorlinks=true,
    linkcolor=black,
    filecolor=magenta,      
    urlcolor=cyan,
    }

\urlstyle{same}

\begin{document}
\pagenumbering{roman}
\tableofcontents
\newpage
\pagenumbering{arabic}
\chapter{Insiemi}
\section{Concetti di base sugli insiemi}
Un \textit{insieme} è un raggruppamento di oggetti detti \textit{elementi}, che possono essere di natura qualsiasi, Si 
dice che gli elementi di un insieme \textit{appartengono} all'insieme.
\subsubsection{Simboli}
Per indicare gli insiemi si usano solitamente lettere maiuscole, come
\begin{equation*}
    A,B,C\dots
\end{equation*}

Per indicare gli elementi di un insieme si usano solitamente lettere minuscole ($a,b,c\dots$) e per indicare che un elemento 
$x$ appartiene all'insieme $A$ scriviamo
\begin{equation*}
    x\in A \quad \textnormal{oppure} \quad A\ni x
\end{equation*}
\subsubsection{Rappresentazione}
È possibile rappresentare un insieme elencando i suoi elementi, in caso questo sia finito. Ad esempio
\begin{equation*}
    A=\{1,2,5\}
\end{equation*}
significa che l'insieme $A$ ha come elementi i numeri 1,2,5. In questo caso si dice che l'abbiamo definito \textit{per tabulazione}.

Oppure è possibile rappresentare un insieme descrivendolo mediante una proprietà che lo caratterizzi univocamente. Ad esempio
\begin{equation*}
    X=\{n:n \textnormal{ intero pari}\}
\end{equation*}

Un insieme privo di elementi viene detto \textit{insieme vuoto} e viene indicato con il simbolo $\emptyset$.
\subsubsection{Relazioni tra insiemi}
\begin{definition}
    Si dice che un insieme $X$ è un sottoinsieme di un insieme $Y$ se ogni elemento di $X$ appartiene ad $Y$. Si utilizza 
    il simbolo di inclusione (larga) $X\subseteq Y$.
\end{definition}
\end{document}
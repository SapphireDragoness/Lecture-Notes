\documentclass[11pt]{book}
\usepackage[margin=.9in]{geometry}
\usepackage[italian]{babel}
\usepackage{tikz}
\usepackage{amsfonts} 
\usepackage{amsthm}
\usepackage{amsmath}
\usepackage{thmtools}
\usepackage{float}
\usepackage{hyperref}

\title{Appunti Analisi Matematica}

\newtheorem{theorem}{Teorema}[chapter]
\newtheorem{definition}{Definizione}[chapter]

\hypersetup{
    colorlinks=true,
    linkcolor=black,
    filecolor=magenta,      
    urlcolor=cyan,
    }

\urlstyle{same}

\begin{document}
\pagenumbering{roman}
\tableofcontents

\newpage
\pagenumbering{arabic}
\chapter{Insiemi}
\section{Concetti di base sugli insiemi}
Un \textit{insieme} è un raggruppamento di oggetti detti \textit{elementi}, che possono essere di natura qualsiasi, Si 
dice che gli elementi di un insieme \textit{appartengono} all'insieme.
\subsubsection{Simboli}
Per indicare gli insiemi si usano solitamente lettere maiuscole, come
\begin{equation*}
    A,B,C\dots
\end{equation*}

Per indicare gli elementi di un insieme si usano solitamente lettere minuscole ($a,b,c\dots$) e per indicare che un elemento 
$x$ appartiene all'insieme $A$ scriviamo
\begin{equation*}
    x\in A \quad \textnormal{oppure} \quad A\ni x
\end{equation*}
\subsubsection{Rappresentazione}
È possibile rappresentare un insieme elencando i suoi elementi, in caso questo sia finito. Ad esempio
\begin{equation*}
    A=\{1,2,5\}
\end{equation*}
significa che l'insieme $A$ ha come elementi i numeri 1,2,5. In questo caso si dice che l'abbiamo definito \textit{per tabulazione}.

Oppure è possibile rappresentare un insieme descrivendolo mediante una proprietà che lo caratterizzi univocamente. Ad esempio
\begin{equation*}
    X=\{n:n \textnormal{ intero pari}\}
\end{equation*}

Un insieme privo di elementi viene detto \textit{insieme vuoto} e viene indicato con il simbolo $\emptyset$.
\subsubsection{Relazioni tra insiemi: inclusione}
\begin{definition}
    Si dice che un insieme $X$ è un sottoinsieme di un insieme $Y$ se ogni elemento di $X$ appartiene ad $Y$. Si utilizza 
    il simbolo di inclusione (larga) $X\subseteq Y$. Se $X$ non coincide con $Y$, si usa il simbolo di inclusione stretta
    $X\subset Y$.
\end{definition}
\subsubsection*{Relazioni tra insiemi: uguaglianza}
Due insiemi sono uguali quando possiedono gli stessi elementi.
Siano $X$ e $Y$ due insiemi. Allora $X=Y$ se e solo se $X\subseteq Y$ e $Y\subseteq X$.
\subsubsection*{Operazioni insiemistiche}
\begin{itemize}
    \item \textit{Unione}. L'unione di due insiemi $X$ e $Y$ è definita da:
    \begin{equation*}
        X\cup Y=\{x:x\in X \textnormal{ o } x\in Y\}
    \end{equation*}
    È l'insieme degli elementi che appartengono sia al primo sia al secondo insieme.
    \item \textit{Intersezione}. L'intersezione di due insiemi $X$ e $Y$ è definita da:
    \begin{equation*}
        X\cap Y=\{x:x\in X \textnormal{ e } x\in Y\}
    \end{equation*}
    È l'insieme degli elementi che appartengono al primo e al secondo insieme, intendendo la "o" in modo non esclusivo.
    \item \textit{Unione}. La differenza tra due insiemi $X$ e $Y$ è definita da:
    \begin{equation*}
        X\setminus Y=\{x:x\in X \textnormal{ e } x\notin Y\}
    \end{equation*}
    È l'insieme degli elementi che appartengono al primo ma non al secondo insieme.
    \item \textit{Complementare}. È un tipo particolare di differenza. Siano $X$ e $Y$ insiemi con $X\subseteq Y$. Si 
    definisce insieme complementare di $X$ in $Y$ l'insieme $Y\setminus X$. Si indica con il simbolo $X^C$.
\end{itemize}
\subsubsection*{Proprietà delle operazioni insiemistiche}
\begin{itemize}
    \item \textit{Proprietà dell'unione}.
        \begin{itemize}
            \item [] commutativa: $X\cup Y=Y\cup X$
            \item [] associativa: $/^(X\cup Y)\cup Z = X\cup (Y\cup Z)$
            \item [] idempotenza: $X\cup X=X$
        \end{itemize}
    \item \textit{Proprietà dell'intersezione}.
    \begin{itemize}
        \item [] commutativa: $X\cap Y=Y\cap X$
        \item [] associativa: $/^(X\cap Y)\cap Z = X\cap (Y\cap Z)$
        \item [] idempotenza: $X\cap X=X$
    \end{itemize}
    \item \textit{Proprietà distributive}.
    \begin{itemize}
        \item [] $X\cup (Y\cap Z)=(X\cup Y)\cap(X\cup Z)$
        \item [] $X\cap (Y\cup Z)=(X\cap Y)\cup(X\cap Z)$
    \end{itemize}
    \item \textit{Formule di De Morgan}.
    Siano $A,B\subseteq X$ e denotiamo con $A^C$ e $B^C$ i loro insiemi complementari in $X$.
    \begin{itemize}
        \item [] $(A\cup B)^C =A^C \cap B^C$
        \item [] $(A\cap B)^C =A^C \cup B^C$
    \end{itemize}
\end{itemize}
\section{Insieme delle parti e prodotto cartesiano}
\subsubsection*{Insieme delle parti}
Dato un insieme $X$, si dice \textit{insieme delle parti} di $X$ l'insieme di tutti i sottoinsiemi di $X$. Si indica con 
$\mathcal{P}^X$ o con $2^X$. Ad esempio, si consideri l'insieme $X=\{1,4,5\}$:
\begin{equation*}
    \mathcal{P}(X)=\{\emptyset,\{1\},\{4\},\{5\},\{1,4\},\{1,5\},\{4,5\},\{1,4,5\}\}
\end{equation*}
Se $X$ è un insieme finito con $n$ elementi, allora $\mathcal{P}(X)$ è un insieme finito con $2^n$ elementi. Se $X$ è un 
insieme infinito, allora anche $\mathcal{P}(X)$ è un insieme infinito.
\subsubsection*{Prodotto cartesiano}
Si dice \textit{coppia} un insieme ordinato di due elementi. Ad esempio:
\begin{align*}
    \{3,7\}&=\{7,3\}:\textnormal{ insieme non ordinato}\\
    \{3,7\}&\neq\{7,3\}:\textnormal{ insieme ordinato (coppia)}
\end{align*}
Dati due insiemi $X$ e $Y$, il \textit{prodotto cartesiano} di $X$ e $Y$ è l'insieme delle coppie $(x,y)$ in cui $x\in X$
e $y\in Y$.
\begin{equation*}
    X\times Y=\{(x,y):x\in X\textnormal{ e }y\in Y\}
\end{equation*}
L'insieme dei numeri reali è indicato con $\mathbb{R}$, e il prodotto cartesiano $\mathbb{R}\times\mathbb{R}$ viene 
indicato con $\mathbb{R}^2$.
\begin{equation*}
    \mathbb{R}^2=\mathbb{R}\times\mathbb{R}=\{(x,y):x,y\in\mathbb{R}\}
\end{equation*}
\section{Insiemi numerici fondamentali}
\begin{itemize}
    \item [$\mathbb{N}$] insieme dei numeri naturali (interi positivi, zero compreso)
    \begin{equation*}
        \mathbb{N}=\{0,1,2,3,4,5,6,\dots\}
    \end{equation*}
    \item [$\mathbb{Z}$] insieme dei numeri interi
    \begin{equation*}
        \mathbb{Z}=\{0,1,-1,2,-2,3,-3,\dots\}
    \end{equation*}
    \item [$\mathbb{Q}$] insieme dei numeri razionali (sono frazioni di numeri interi)
    \begin{equation*}
        \mathbb{Q}=\{\frac{a}{b}:a\in\mathbb{Z}\textnormal{ e }b\in\mathbb{Z}\setminus 0\}
    \end{equation*}
    \item [$\mathbb{R}$] insieme dei numeri reali
    \item [$\mathbb{I}$] insieme dei numeri irrazionali ($\mathbb{I}=\mathbb{R}\setminus\mathbb{Q}$)
\end{itemize}
\chapter{Funzioni}
\section{Il concetto di funzione}
\begin{definition}
    Siano $X,Y\neq\emptyset$. Si dice funzione da $X$ in $Y$ una legge che fa corrispondere ad ogni elemento di $X$ uno 
    ed un solo elemento di $Y$. L'insieme $X$ viene detto dominio della funzione mentre l'insieme $Y$ viene detto codominio 
    della funzione.
\end{definition}
Viene utilizzata la seguente notazione:
\begin{align*}
    f&:X\rightarrow Y \textnormal{ ($f$ definita da $X$ a $Y$)} & f:x&\mapsto f(x) \textnormal{ ($f$ associa $f(x)$ ad $x$)}
\end{align*}
Per ogni $x\in X$, l'elemento di $Y$ che la funzione $f$ fa corrispondere a $x$ viene detto \textit{immagine} di $x$ 
mediante $f$ e si indica con $f(x)$. La proprietà caratteristica di $f$, affinchè la si possa chiamare funzione, è 
l'\textit{univocità} della corrispondenza: assegnato l'elemento in "ingresso" $a\in A$, l'elemento in "uscita" $b=f(a)$ 
dev'essere univocamente determinato.
\begin{definition}
    Siano $X,Y \neq \emptyset$ e sia $f:X\rightarrow Y$. Si dice grafico di $F$ l'insieme
    \begin{equation*}
        \mathcal{G}(f)=\{(x,y)\in X\times Y:x\in X \textnormal{ e } y=f(x)\}
    \end{equation*}
\end{definition}
Ossia, il grafico di una funzione $f$ è l'insieme dei punti del piano di coordinate $(x,y)$ con $y=f(x)$ e $x$ variabile 
nel dominio $D$.
\begin{definition}
    Siano $X,Y \neq \emptyset$ e sia $f:X\rightarrow Y$. Siano rispettivamente $A\subseteq X$ e $B\subseteq Y$. Si dice 
    immagine di $A$ il sottoinsieme di $Y$ costruito dalle imagini dei singoli elementi di $A$. Tale insieme viene denotato 
    con $f(A)$. In altre parole si ha 
    \begin{equation*}
        f(A)=\{f(x)\in Y:x\in A\}\subseteq Y 
    \end{equation*}
    Si dice immagine inversa di $B$ o controimmagine di $B$ il sottoinsieme di $X$ costituito da quegli elementi di $X$ 
    la cui immagine appartiene a $B$. Tale insieme viene denotato con $f^{-1}(B)$. In altre parole si ha 
    \begin{equation*}
        f^{-1}(B)=\{x\in X:f(x)\in B\}\subseteq X
    \end{equation*}
\end{definition}
\begin{definition}
    Siano $X,Y \neq \emptyset$ e sia $f:X\rightarrow Y$. Sia inoltre $A\subseteq X$ insieme non vuoto. Si dice restrizione 
    di $f$ all'insieme $A$ la funzione 
    \begin{equation*}
        f_{|A}:A\rightarrow Y
    \end{equation*}
\end{definition}
\begin{theorem}
    Siano $X,Y \neq \emptyset$ e sia $f:X\rightarrow Y$. Siano inoltre $A,B\subseteq X$ e $C,D\subseteq Y$. Valgono le 
    seguenti conclusioni:
    \begin{enumerate}
        \item $f(A\cup B)=f(A)\cup f(B)$
        \item $f(A\cap B)=f(A)\cap f(B)$
        \item $f^{-1}(f(A))\supset A$
        \item $f^{-1}(C\cup D)=f^{-1}(C)\cup f^{-1}(D)$
        \item $f^{-1}(C\cap D)=f^{-1}(C)\cap f^{-1}(D)$
        \item $f(f^{-1}(C))\subseteq C$
    \end{enumerate}
\end{theorem}
\begin{definition}
    Siano $X,Y,Z$ insiemi non vuoti. Siano $f:y\rightarrow Z$ e $g:X\rightarrow Y$. Viene detta composizione di $f$ con 
    $g$ la funzione $f\circ g$ definita da
    \begin{equation*}
        f\circ g:X\rightarrow Z
    \end{equation*}
\end{definition}
\begin{definition}
    Siano $X$ e $Y$ insiemi non vuoti e sia $f:X\rightarrow Y$. Si dice che $f$ è iniettiva se elementi distinti del dominio 
    $X$ hanno immagini distinte:
    \begin{equation*}
        x_1,x_2\in X,x_1\neq x_2\Rightarrow f(x_1)\neq f(x_2)
    \end{equation*}
\end{definition}
\begin{definition}
    Siano $X$ e $Y$ insiemi non vuoti e sia $f:X\rightarrow Y$. Si dice che $f$ è suriettiva se ogni elemento del codominio 
    $Y$ è immagine di almeno un elemento del dominio $X$:
    \begin{equation*}
        \forall y\in Y, \exists x\in X:f(x)=y
    \end{equation*}
\end{definition}
\begin{definition}
    Siano $X$ e $Y$ insiemi non vuoti e sia $f:X\rightarrow Y$. Si dice che $f$ è biettiva (biiettiva o biunivoca) se è 
    contemporaneamente iniettiva e suriettiva. In altre parole, si dice che $f$ è biettiva se ogni elemento del codominio
    $Y$ è immagine di esattamente un elemento del dominio $X$:
    \begin{equation*}
        \forall y\in Y,\exists !x\in X:f(x)=y
    \end{equation*}
\end{definition}
Dato un insieme $X$ non vuoto, indicheremo con $Id_X:X\rightarrow X$ la \textit{funzione identità}, cioè quella funzione 
che ad ogni $x\in X$ associa se stesso:
\begin{equation*}
    Id_X(x)=x \forall x\in X
\end{equation*}
\begin{definition}
    Siano $X$ e $Y$ insiemi non vuoti e sia $f:X\rightarrow Y$. Si dice che $f$ è invertibile se esiste una funzione 
    $g:Y\rightarrow X$ tale che $f\circ g=Id_Y$ e $g\circ f=Id_X$. In altre parole si ha che:
    \begin{align*}
        f(g(y))=y\quad&\forall y\in Y &  g(f(x))=x\quad&\forall x\in X
    \end{align*}
    Se $f$ è invertibile, una tale funzione $g$ è unica e viene denominata funzione inversa di $f$. Si indica con il 
    simbolo $f^{-1}:Y\rightarrow X$.
\end{definition}
\begin{theorem}
    Siano $X$ e $Y$ insiemi non vuoti e sia $f:X\rightarrow Y$. Allora $f$ è invertibile se e solo se $f$ è biettiva.
\end{theorem}
\begin{definition}
    Siano $X$ e $Y$ due insiemi non vuoti. Si dice che $X$ e $Y$ hanno la stessa cardinalità se esiste una funzione 
    $f:X\rightarrow Y$ biettiva.
\end{definition}
Non tutti gli insiemi infiniti hanno la stessa cardinalità. Gli insiemi con la stessa cardinalità di $\mathbb{N}$ si 
dicono \textit{numerabili}. Gli insiemi $\mathbb{Z}$ e $\mathbb{Q}$ sono numerabili ed hanno quindi la stessa cardinalità 
di $\mathbb{N}$, nonostante $\mathbb{N}\subset\mathbb{Z}\subset\mathbb{Q}$. L'insieme $\mathbb{R}$ dei numeri reali non 
è numerabile: ha una cardinalità di \textit{ordine superiore} rispetto ad $\mathbb{N}$.
\chapter{Insiemi in $\mathbb{R}$}
\section{Intervalli}
Si dice \textit{intervallo} un sottoinsieme di $\mathbb{R}$ tale per cui ogni numero reale compreso tra due elementi di 
questo sottoinsieme appartiene al sottoinsieme medesimo.

Dati $a,b\in \mathbb{R}$ con $a<b$, si definiscono i seguenti intervalli:
\begin{align*}
    [a,b]&=\{x\in\mathbb{R}:a\leq x\leq b\} & (a,b)&=\{x\in\mathbb{R}:a< x< b\}
\end{align*}
Anche un insieme costituito da un solo numero reale va considerato un intervallo.
\begin{definition}
    Sia $X\subseteq\mathbb{R}$ un insieme nonn vuoto:
    \begin{enumerate}
        \item si dice che $X$ è limitato superiormente se esiste $a\in\mathbb{R}$ tale che
        \begin{equation*}
            \forall x\in X\rightarrow x\leq a
        \end{equation*}
        \item si dice che $X$ è limitato inferiormente se esiste $a\in\mathbb{R}$ tale che
        \begin{equation*}
            \forall x\in X\rightarrow x\geq a
        \end{equation*}
        \item si dice che $X$ è limitato se è contemporaneamente limitato inferiormente e superiormente
    \end{enumerate}
\end{definition}
\begin{definition}
    Sia $X\subseteq\mathbb{R}$ un insieme non vuoto:
    \begin{enumerate}
        \item si dice $a\in\mathbb{R}$ è un maggiorante di $X$ se
        \begin{equation*}
            \forall x\in X\rightarrow x\leq a
        \end{equation*}
        \item si dice $a\in\mathbb{R}$ è un minorante di $X$ se
        \begin{equation*}
            \forall x\in X\rightarrow x\geq a
        \end{equation*}
    \end{enumerate}
\end{definition}
\begin{definition}
    Sia $X\subseteq\mathbb{R}$ un insieme non vuoto:
    \begin{enumerate}
        \item si dice che $a\in\mathbb{R}$ è un massimo ($maxX$) di $X$ se $a$ è un maggiorante di $X$ e $a\in X$
        \item si dice che $a\in\mathbb{R}$ è un minimo ($minX$) di $X$ se $a$ è un minorante di $X$ e $a\in X$
    \end{enumerate}
\end{definition}
Se $X$ ammette un massimo, tale massimo è unico. Analogamente, se $X$ ammette un minimo, tale minimo è unico.
\begin{definition}
    Sia $X\subseteq\mathbb{R}$ un insieme non vuoto:
    \begin{enumerate}
        \item si dice che $a\in\mathbb{R}$ è l'estremo superiore ($supX$) di $X$ se $a$ è il minimo dell'insieme dei maggioranti di 
        $X$
        \item si dice che $a\in\mathbb{R}$ è l'estremo inferiore ($infX$) di $X$ se $a$ è il massimo dell'insieme dei minoranti di 
        $X$
    \end{enumerate}
\end{definition}
\begin{theorem}[Completezza di $\mathbb{R}$]
    Sia $X\subseteq\mathbb{R}$ un insieme non vuoto:
    \begin{enumerate}
        \item se $X$ è limitato superiormente, allora $X$ ammette un estremo superiore
        \item se $X$ è limitato inferiormente, allora $X$ ammette un estremo inferiore
    \end{enumerate}
\end{theorem}
\chapter{Limiti}
\section{Definizione di limite}
\begin{definition}
    Sia $x_0\in\mathbb{R}$ e sia $R>0$. Si dice intorno di $x_0$ di raggio $R$ l'intervallo $(x_0-R,x_0+R)$.
\end{definition}
Si ha che $x$ appartiene all'intorno di raggio $R$ se e soltanto se $|x-x_0|<R$.
\begin{definition}
    Sia $X\subseteq\mathbb{R},X\neq\emptyset$. Sia $f:X\rightarrow\mathbb{R}$. Sia $x_0\in\mathbb{R}$ tale per cui esista 
    $R>0$ tale che
    \begin{equation*}
        X\supseteq(x_0-R,x_0+R)\setminus\{x_0\}
    \end{equation*}
    Sia $\ell\in\mathbb{R}$. Si dice che $\ell$ è il limite di $f$ per $x$ che tende a $x_0$ se
    \begin{equation*}
        \forall\varepsilon>0,\exists\delta>0:|x-x_0|<\delta\implies|f(x)-\ell|<\varepsilon
    \end{equation*}
\end{definition}
$x_0$ non deve necessariamente far parte del dominio della funzione.
\begin{definition}
    Sia $X\subset\mathbb{R},X\neq\emptyset$. Sia $f:X\rightarrow\mathbb{R}$. Sia $x_0\in\mathbb{R}$ tale per cui esista 
    $R>0$ tale che 
    \begin{equation*}
        X\supseteq(x_0-R,x_0+R)\setminus\{x_0\}
    \end{equation*}
    \begin{enumerate}
        \item si dice che $+\infty$ è il limite di $f$ per $x$ che tende a $x_0$ se 
        \begin{equation*}
            \forall M>0,\exists\delta>0:|x-x_0|<\delta\implies f(x)>M
        \end{equation*}
        \item si dice che $-\infty$ è il limite di $f$ per $x$ che tende a $x_0$ se 
        \begin{equation*}
            \forall M>0,\exists\delta>0:|x-x_0|<\delta\implies f(x)<-M
        \end{equation*}
    \end{enumerate}
\end{definition}
\begin{definition}
    Sia $X\subset\mathbb{R},X\neq\emptyset$.
    \begin{enumerate}
        \item Supponiamo che esista $a\in\mathbb{R}$ per cui si abbia
        \begin{equation*}
            X\supseteq(a,+\infty)
        \end{equation*}
        Sia $f:X\rightarrow\mathbb{R}$. Sia $\ell\in\mathbb{R}$. Si dice che $\ell$ è il limite di $f$ per $x$ che tende
        a $+\infty$ se 
        \begin{equation*}
            \forall\varepsilon>0,\exists N>0:x>N\implies|f(x)-\ell|<\varepsilon
        \end{equation*}
        \item Supponiamo che esista $a\in\mathbb{R}$ per cui si abbia
        \begin{equation*}
            X\supseteq(-\infty,a)
        \end{equation*}
        Sia $f:X\rightarrow\mathbb{R}$. Sia $\ell\in\mathbb{R}$. Si dice che $\ell$ è il limite di $f$ per $x$ che tende
        a $-\infty$ se 
        \begin{equation*}
            \forall\varepsilon>0,\exists N>0:x<-N\implies|f(x)-\ell|<\varepsilon
        \end{equation*}
    \end{enumerate}
\end{definition}
\begin{definition}
    Sia $X\subset\mathbb{R},X\neq\emptyset$.
    \begin{enumerate}
        \item Supponiamo che esista $a\in\mathbb{R}$ per cui si abbia
        \begin{equation*}
            X\supseteq(a,+\infty)
        \end{equation*}
        Sia $f:X\rightarrow\mathbb{R}$. Si dice che $+\infty$ è il limite di $f$ per $x$ che tende
        a $+\infty$ se 
        \begin{equation*}
            \forall M>0,\exists N>0:x>N\implies f(x)>M
        \end{equation*}
        \item Supponiamo che esista $a\in\mathbb{R}$ per cui si abbia
        \begin{equation*}
            X\supseteq(a,+\infty)
        \end{equation*}
        Sia $f:X\rightarrow\mathbb{R}$. Si dice che $-\infty$ è il limite di $f$ per $x$ che tende
        a $-\infty$ se 
        \begin{equation*}
            \forall M>0,\exists N>0:x>N\implies f(x)<-M
        \end{equation*}
        \item Supponiamo che esista $a\in\mathbb{R}$ per cui si abbia
        \begin{equation*}
            X\supseteq(-\infty,a)
        \end{equation*}
        Sia $f:X\rightarrow\mathbb{R}$. Si dice che $+\infty$ è il limite di $f$ per $x$ che tende
        a $-\infty$ se 
        \begin{equation*}
            \forall M>0,\exists N>0:x<-N\implies f(x)>M
        \end{equation*}
        \item Supponiamo che esista $a\in\mathbb{R}$ per cui si abbia
        \begin{equation*}
            X\supseteq(-\infty,a)
        \end{equation*}
        Sia $f:X\rightarrow\mathbb{R}$. Si dice che $-\infty$ è il limite di $f$ per $x$ che tende
        a $-\infty$ se 
        \begin{equation*}
            \forall M>0,\exists N>0:x<-N\implies f(x)<-M
        \end{equation*}
    \end{enumerate}
\end{definition}
\section{Limite destro e limite sinistro}
\begin{definition}[Limite destro]
    Sia $X\subset\mathbb{R},X\neq\emptyset$. Sia $f:X\rightarrow\mathbb{R}$. Sia $x_0$ tale per cui esista $R>0$ tale che 
    \begin{equation*}
        X\supseteq(x_0x,x_0+R)\quad\textnormal{($X$ contiene un intorno destro di $x_0$)}
    \end{equation*}
    \begin{enumerate}
        \item Sia $\ell\in\mathbb{R}$. Si dice che $\ell$ è il limite destro di $f$ per $x$ che tende a $x_0$ da destra 
        se 
        \begin{equation*}
            \forall\varepsilon>0,\exists\delta>0:|x\in(x_0,x_0+\delta)\implies|f(x)-\ell|<\varepsilon
        \end{equation*}
        \item Si dice che $+\infty$ è il limite destro di $f$ per $x$ che tende a $x_0$ da destra 
        se 
        \begin{equation*}
            \forall M>0,\exists\delta>0:|x\in(x_0,x_0+\delta)\implies f(x)>M
        \end{equation*}
        \item Si dice che $-\infty$ è il limite destro di $f$ per $x$ che tende a $x_0$ da destra 
        se 
        \begin{equation*}
            \forall M>0,\exists\delta>0:|x\in(x_0,x_0+\delta)\implies f(x)<-M
        \end{equation*}
    \end{enumerate}
\end{definition}
\begin{definition}[Limite sinistro]
    Sia $X\subset\mathbb{R},X\neq\emptyset$. Sia $f:X\rightarrow\mathbb{R}$. Sia $x_0$ tale per cui esista $R>0$ tale che 
    \begin{equation*}
        X\supseteq(x_0x,x_0+R)\quad\textnormal{($X$ contiene un intorno destro di $x_0$)}
    \end{equation*}
    \begin{enumerate}
        \item Sia $\ell\in\mathbb{R}$. Si dice che $\ell$ è il limite sinistro di $f$ per $x$ che tende a $x_0$ da destra 
        se 
        \begin{equation*}
            \forall\varepsilon>0,\exists\delta>0:|x\in(x_0-\delta,x_0)\implies|f(x)-\ell|<\varepsilon
        \end{equation*}
        \item Si dice che $+\infty$ è il limite sinistro di $f$ per $x$ che tende a $x_0$ da destra 
        se 
        \begin{equation*}
            \forall M>0,\exists\delta>0:|x\in(x_0-\delta,x_0)\implies f(x)>M
        \end{equation*}
        \item Si dice che $-\infty$ è il limite sinistro di $f$ per $x$ che tende a $x_0$ da destra 
        se 
        \begin{equation*}
            \forall M>0,\exists\delta>0:|x\in(x_0-\delta,x_0)\implies f(x)<-M
        \end{equation*}
    \end{enumerate}
\end{definition}
\begin{theorem}
    Sia $X\subset\mathbb{R},X\neq\emptyset$. Sia $f:X\rightarrow\mathbb{R}$. Sia $x_0\in\mathbb{R}$ tale per cui esista
    $R>0$ tale che 
    \begin{equation*}
        X\supseteq(x_0-R,x_0+R)\setminus\{x_0\}
    \end{equation*}
    Allora
    \begin{equation*}
        \lim_{x\rightarrow x_0}f(x)
    \end{equation*}
    esiste se e solo se esistono e coincidono tra di loro 
    \begin{align*}
        \lim_{x\rightarrow x_0^-}&f(x) & \lim_{x\rightarrow x_0^+}&f(x)
    \end{align*}
\end{theorem}
\section{Operazioni tra limiti}
\begin{theorem}
    Sia $X\subset\mathbb{R},X\neq\emptyset$. Sia $f:X\rightarrow\mathbb{R}$. Supponiamo che le funzioni $f$ e $g$ 
    ammettano limiti finiti. Allora si ha:
    \begin{enumerate}
        \item $\lim_{x\rightarrow x_0}(f(x)+g(x))=\ell_1+\ell_2$
        \item $\lim_{x\rightarrow x_0}(f(x)-g(x))=\ell_1-\ell_2$
        \item $\lim_{x\rightarrow x_0}(f(x)\cdot g(x))=\ell_1\cdot \ell_2$
        \item se $c\in\mathbb{R}$, $\lim_{x\rightarrow x_0}cf(x)=c\lim_{x\rightarrow x_0}f(x)$
        \item se $g(x)\neq 0$ e $\ell_2\neq 0$, $\lim_{x\rightarrow x_0}(f(x)/g(x))=\ell_1/\ell_2$
    \end{enumerate}
\end{theorem}
\section{Cambiamento di variabile e continuità}
\begin{theorem}[Cambiamento di variabile]
    Siano $X,Y\subseteq\mathbb{R}$ insiemi non vuoti e siano date due funzioni $f:Y\rightarrow\mathbb{R}$ e $g:X\rightarrow\mathbb{R}$.
    Siano $x_0,y_0\in\mathbb{R}\cup\{\pm\infty\}$ e sia $\ell\in\mathbb{R}\cup\{\pm\infty\}$ tali che si abbia 
    \begin{align*}
        \lim_{x\rightarrow x_0}g(x)&=y_0 & \lim_{y\rightarrow y_0}f(y)&=\ell
    \end{align*}
    Supponiamo inoltre che $g(x)\neq y_0$ per ogni $x\in X\setminus\{x_0\}$. Allora si ha 
    \begin{equation*}
        \lim_{x\rightarrow x_0}f(g(x))=\ell
    \end{equation*}
\end{theorem}
Supponiamo di dover calcolare un limite che si presenta nella forma:
\begin{equation*}
    \lim_{x\rightarrow x_0}f(g(x))
\end{equation*}
Si introduce una nuova variabile $y=g(x)$:
\begin{equation*}
    x\rightarrow x_0 \implies y=g(x)\rightarrow y_0 \implies y\rightarrow y_0
\end{equation*}
sostituisco $g(x)$ con $y$ nel limite:
\begin{equation*}
    \lim_{y\rightarrow y_0}f(y)
\end{equation*}
\begin{definition}[Continuità]
    Sia $X\subseteq\mathbb{R}$ non vuoto e sia $f:X\rightarrow\mathbb{R}$. Dato $x_0\in X$ si dice che $f$ è continua in 
    $x_0$ se vale la seguente condizione:
    \begin{equation*}
        \forall\varepsilon>0,\exists\delta>0:|x-x_0|<\delta\implies|f(x)-f(x_0)|<\varepsilon
    \end{equation*}
    Si dice che $f$ è continua su $A\subseteq X$ se è continua in ogni punto di $A$.
\end{definition}
Se $x_0$ e $X$ sono tali per cui esiste $R>0$ per cui $(x_0-R,x_0+R)\subseteq X$ allora $f$ è continua in $x_0$ se e soltanto
se $\lim_{x\rightarrow x_0}f(x)=s(x_0)$.

Le funzioni elementari sono continue in ogni punto del loro dominio naturale.
\begin{theorem}[Teorema del confronto]
    Sia $X\subset\mathbb{R}$ e siano $f:X\rightarrow\mathbb{R}$ e $g:X\rightarrow\mathbb{R}$ tali che 
    \begin{equation*}
        f(x)\leq g(x), \forall x\in X
    \end{equation*}
    Supponiamo che esistano, finiti o infiniti, i limiti
    \begin{align*}
        \lim_{x\rightarrow x_0}&f(x) & \lim_{x\rightarrow x_0}&g(x)
    \end{align*}
    con $x_0\in\mathbb{R}\cup\{\pm\infty\}$. Allora si ha 
    \begin{equation*}
        \lim_{x\rightarrow x_0}f(x)\leq\lim_{x\rightarrow x_0}g(x)
    \end{equation*}
\end{theorem}
\begin{proof}
    Dalla definizione di limite si deduce che 
    \begin{align*}
        \forall\varepsilon>0,\exists\delta_1>0:|x-x_0|<\delta_1\implies|f(x)-\ell_1|<\varepsilon\\
        \forall\varepsilon>0,\exists\delta_2>0:|x-x_0|<\delta_2\implies|g(x)-\ell_2|<\varepsilon
    \end{align*}
    Posto $\delta=min\{\delta_1,\delta_2\}$ per ogni $x\in X$ tale che $0<|x-x_0|<\delta$ si ha:
    \begin{align*}
        |f(x)-\ell_1|<\varepsilon\implies \ell_1-\varepsilon<f(x)<\ell_1+\varepsilon\\
        |g(x)-\ell_2|<\varepsilon\implies \ell_2-\varepsilon<g(x)<\ell_2+\varepsilon
    \end{align*}
    ed in particolare si ottiene
    \begin{equation*}
        \ell_1-\varepsilon<f(x)\leq g(x)<\ell_2+\varepsilon\implies\ell_1-\varepsilon<\ell_2+\varepsilon\implies\ell_1-\ell_2<2\varepsilon\implies\ell_1\leq\ell_2
    \end{equation*}
\end{proof}
\begin{theorem}[Teorema dei due Carabinieri]
    Sia $X\subseteq\mathbb{R}$ e siano $f:X\rightarrow\mathbb{R},g:X\rightarrow\mathbb{R},h:X\rightarrow\mathbb{R}$ tali 
    che 
    \begin{equation*}
        g(x)\leq f(x)\leq h(x)\quad\forall x\in X
    \end{equation*}
    Supponiamo che 
    \begin{equation*}
        \lim_{x\rightarrow x_0}g(x)=\lim_{x\rightarrow x_0}h(x)
    \end{equation*}
    e sia $\ell\in\mathbb{R}\cup\{\pm\infty\}$ il valore comune di questi due limiti. Allora anche $f$ ammette limite 
    per $x\rightarrow x_0$ ed inoltre
    \begin{equation*}
        lim_{x\rightarrow x_0}f(x)=\ell
    \end{equation*}
\end{theorem}
\begin{proof}
    Dalla definizione di limite si deduce che 
    \begin{align*}
        \forall\varepsilon>0,\exists\delta_1>0:|x-x_0|<\delta_1\implies|g(x)-\ell|<\varepsilon\\
        \forall\varepsilon>0,\exists\delta_2>0:|x-x_0|<\delta_2\implies|h(x)-\ell|<\varepsilon
    \end{align*}
    Posto $\delta=min\{\delta_1,\delta_2\}$ per ogni $x\in X$ tale che $0<|x-x_0|<\delta$ si ha:
    \begin{align*}
        |g(x)-\ell|<\varepsilon\implies \ell-\varepsilon<g(x)<\ell+\varepsilon\\
        |h(x)-\ell|<\varepsilon\implies \ell-\varepsilon<h(x)<\ell+\varepsilon
    \end{align*}
    ed in particolare si ottiene
    \begin{equation*}
        \ell-\varepsilon<g(x)\leq f(x)\leq h(x)<\ell+\varepsilon\implies\ell-\varepsilon<f(x)<\ell+\varepsilon\implies|f(x)-\ell|<\varepsilon
    \end{equation*}
\end{proof}
\section{Relazione di asintotico e asintoti}
\begin{definition}
    Sia $X\subseteq\mathbb{R}$ e siano $f:X\rightarrow\mathbb{R},g:X\rightarrow\mathbb{R}$. Sia $x_0\in\mathbb{R}\cup\{\pm\infty\}$
    tale che $X$ contenga $U\setminus\{x_0\}$ con $U$ intorno di $x_0$. Supponiamo che $f(x)\neq 0$ e $g(x)\neq 0$ per 
    ogni $x\in X\setminus\{x_0\}$. Si dice che $f(x)$ è asintotica a $g(x)$ per $x\rightarrow x_0$ se 
    \begin{equation*}
        \lim_{x\rightarrow x_0} \frac{f(x)}{g(x)}=1
    \end{equation*}
\end{definition}
Si usa la notazione
\begin{equation*}
    f(x)\sim g(x)\quad\textnormal{per } x\rightarrow x_0
\end{equation*}
Si tratta di una relazione di equivalenza poichè possiede la proprietà riflessiva, simmetrica e transitiva.
\begin{itemize}
    \item proprietà riflessiva: $f(x)\sim f(x)$ per $x\rightarrow x_0$
    \item proprietà simmetrica: se $f(x)\sim g(x)$ per $x\rightarrow x_0$, allora per $g(x)\sim f(x)$ per $x\rightarrow x_0$
    \item proprietà transitiva: se $f(x)\sim g(x)$ e $g(x)\sim h(x)$ per $x\rightarrow x_0$, allora  $f(x)\sim h(x)$ per $x\rightarrow x_0$
\end{itemize}
Nel calcolo dei limiti si potrà sostituire in alcune situazioni una funzione con un'altra ad essa asintotica.

Le funzioni asintotiche si possono sostituire solo in presenza di moltiplicazione e divisione.
\subsection{Asintoti}
\subsubsection*{Asintoto verticale}
Si dice che una funzione $f$ possiede un asintoto vertical in $x_0$ se almeno uno tra 
\begin{equation*}
    \lim_{x\rightarrow x_0^+}f(x)\qquad\textnormal{e}\qquad\lim_{x\rightarrow x_0^-}f(x)
\end{equation*}
esiste ed è infinito. In tal caso, si dice che la retta verticale di equazione $x=x_0$ è un asintoto verticale per $f$.
\subsubsection*{Asintoto orizzontale}
Si dice che una funzione $f$ possiede un asintoto orizzontale a $+\infty$ se esiste ed è finito
\begin{equation*}
    \lim_{x\rightarrow +\infty}f(x)=\ell\in\mathbb{R}
\end{equation*}
In tal caso si dice che la retta orizzontale di equazione $y=\ell$ è un asintoto orizzontale per $f$ a $+\infty$.

Analogamente si parla di asintoto orizzontale a $-\infty$ se esiste ed è finito
\begin{equation*}
    \lim_{x\rightarrow -\infty}f(x)=\ell\in\mathbb{R}
\end{equation*}
In tal caso si dice che la retta orizzontale di equazione $y=\ell$ è un asintoto orizzontale per $f$ a $-\infty$.
\subsubsection*{Asintoto obliquo}
Si dice che una funzione $f$ possiede un asintoto obliquo a $+\infty$ se esistono e sono finiti i seguenti limiti
\begin{equation*}
    m=\lim_{x\rightarrow +\infty}\frac{f(x)}{x}\in\mathbb{R}\setminus\{0\}, q=\lim_{x\rightarrow +\infty}[f(x)-mx]
\end{equation*}
In tal caso si dice che la retta di equazione $y=mx+q$ è un asintoto orizzontale per $f$ a $+\infty$.
\section{Successioni}
Sia $X$ un insieme non vuoto. Una funzione $a:\mathbb{N}\rightarrow X$ viene detta successione. Quando $X=\mathbb{R}$ si 
parla di successioni reali o successioni in $\mathbb{R}$.
\begin{definition}
    Sia $\{a_n\}$ una successione in $\mathbb{R}$.
    \begin{enumerate}
        \item Si dice che $\ell\in\mathbb{R}$ è il limite di $\{a_n\}$ per $n\rightarrow +\infty$ se 
        \begin{equation*}
            \forall\varepsilon>0,\exists N\in\mathbb{N}:n>N\implies|a_n-\ell|<\varepsilon
        \end{equation*}
        Si scrive 
        \begin{equation*}
            \lim_{n\rightarrow +\infty}a_n=\ell
        \end{equation*}
        \item Si dice che $+\infty$ è il limite di $\{a_n\}$ per $n\rightarrow +\infty$ se 
        \begin{equation*}
            \forall M>0,\exists N\in\mathbb{N}:n>N\implies a_n>M
        \end{equation*}
        Si scrive 
        \begin{equation*}
            \lim_{n\rightarrow +\infty}a_n=+\infty
        \end{equation*}
        \item Si dice che $-\infty$ è il limite di $\{a_n\}$ per $n\rightarrow +\infty$ se
        \begin{equation*}
            \forall M>0,\exists N\in\mathbb{N}:n>N\implies a_n<-M
        \end{equation*}
        Si scrive 
        \begin{equation*}
            \lim_{n\rightarrow +\infty}a_n=-\infty
        \end{equation*}
    \end{enumerate}
\end{definition}
Le tecniche di calcolo dei limiti delle successioni sono essentialmente le stesse che si usano per il calcolo dei limiti 
per $x\rightarrow +\infty$ nel caso delle funzioni di variabile reale.

Anche nelle successioni è possibile definire relazioni di asintotico: date due successioni $\{a_n\}$ e $\{b_n\}$ con 
$a_n\neq 0$ e $b_n\neq 0$ per ogni $n\in\mathbb{N}$ si dice che 
\begin{equation*}
    a_n\sim b_n\qquad\textnormal{per }n\rightarrow +\infty\qquad\textnormal{se }\lim_{n\rightarrow +\infty}\frac{a_n}{b_n}=1
\end{equation*}
\chapter{Continuità}
\begin{definition}[Continuità]
    Sia $X\subseteq\mathbb{R}$ non vuoto e sia $f:X\rightarrow\mathbb{R}$. Dato $x_0\in X$ si dice che $f$ è continua in 
    $x_0$ se vale la seguente condizione:
    \begin{equation*}
        \forall\varepsilon>0,\exists\delta>0:|x-x_0|<\delta\implies|f(x)-f(x_0)|<\varepsilon
    \end{equation*}
    Si dice che $f$ è continua su $A\subseteq X$ se è continua in ogni punto di $A$.
\end{definition}
\begin{theorem}
    Sia $X\subseteq\mathbb{R}$ insieme non vuoto e siano $f:X\rightarrow\mathbb{R}$ e $g:X\rightarrow\mathbb{R}$. Supponiamo
    che $f$ e $g$ siano continue in un punto $x_0\in X$. Allora valgono le seguenti:
    \begin{enumerate}
        \item $f+g$ è continua in $x_0$
        \item $f-g$ è continua in $x_0$
        \item $f\cdot g$ è continua in $x_0$
        \item se $g(x)\neq 0$ per ogni $x\in X$ allora $\frac{f}{g}$ è continua in $x_0$
    \end{enumerate}
\end{theorem}
\begin{theorem}[Continuità delle funzioni composte]
    Siano $X,Y\subseteq\mathbb{R}$ insiemi non vuoti e siano $f:Y\rightarrow\mathbb{R}$ e $g:X\rightarrow Y$. Supponiamo 
    che $g$ sia continua in un punto $x_0\in X$ e che $f$ sia continua in $y_0=g(x_0)\in Y$. Allora la funzione composta 
    $f\circ g:X\rightarrow\mathbb{R}$ è continua in $x_0$.
\end{theorem}
\begin{definition}[Massimo e minimo di una funzione]
    Sia $X$ insieme non vuoto e sia 
    \begin{equation*}
        f:X\rightarrow\mathbb{R}
    \end{equation*}
    \begin{enumerate}
        \item Si dice che $f$ ammette massimo in $X$ se l'insieme $f(X)$ ammette massimo. Si pone inoltre 
        \begin{equation*}
            \max_{x\in X}f(x)=\max(f(X))
        \end{equation*}
        \item Si dice che $f$ ammette minimo in $X$ se l'insieme $f(X)$ ammette massimo. Si pone inoltre 
        \begin{equation*}
            \min_{x\in X}f(x)=\min(f(X))
        \end{equation*}
    \end{enumerate}
\end{definition}
\begin{definition}[Estremo superiore ed estremo inferiore di una funzione]
    Sia $X$ insieme non vuoto e sia $f:X\rightarrow\mathbb{R}$
    \begin{enumerate}
        \item Si dice che $f$ ammette estremo superiore in $X$ se l'insieme $f(X)$ ammette estremo superiore. Si pone 
        inoltre 
        \begin{equation*}
            \sup_{x\in X}f(x)=\sup(f(X))
        \end{equation*}
        \item Si dice che $f$ ammette estremo inferiore in $X$ se l'insieme $f(X)$ ammette estremo inferiore. Si pone 
        inoltre 
        \begin{equation*}
            \inf_{x\in X}f(x)=\inf(f(X))
        \end{equation*}
    \end{enumerate}
\end{definition}
Una funzione reale ammette sempre estremo superiore ed inferiore purchè si ammettano come loro possibili valori rispettivamente
$+\infty$ e $-\infty$.
\begin{theorem}[Weierstrass]
    Siano $a,b\in\mathbb{R}$ con $a\leq b$. Sia $f:[a,b]\rightarrow\mathbb{R}$ continua in $[a,b]$. Allora $f$ ammette 
    massimo e minimo in $[a,b]$.
\end{theorem}
\begin{theorem}[Teorema di Continuità della Funzione Inversa]
    Siano $I,J\subseteq\mathbb{R}$ intervalli e sia $f:I\rightarrow J$ una funzione invertibile. Se $f$ è continua in $I$
    allora la funzione inversa $f^{-1}:J\rightarrow I$ è continua in $J$.
\end{theorem}
\section{Punti di discontinuità}
I punti di discontinuità vengono classificati in tre specie. Sia $I\subseteq\mathbb{R}$ un intervallo di estremi $a$ e $b$:
\begin{equation*}
    a=\inf I\in\mathbb{R}\cup\{-\infty\}, b=\sup I\in\mathbb{R}\cup\{+\infty\}
\end{equation*}
Sia $f:I\rightarrow\mathbb{R}$ e sia $x_0\in(a,b)$. In tal modo la funzione $f$ è definita sia a destra che a sinistra di 
$x_0$.
\subsubsection*{Discontinuità di prima specie o salto}
Si dice che $x_0$ è un punto di discontinuità di prima specie o di tipo salto se esistono finiti ma diversi tra loro 
\begin{equation*}
    \lim_{x\rightarrow x_0^-}f(x)\qquad\textnormal{e}\qquad\lim_{x\rightarrow x_0^+}f(x)
\end{equation*}
\subsubsection*{Discontinuità di seconda specie}
Si dice che $x_0$ è un punto di discontinuità di seconda specie se almeno uno tra
\begin{equation*}
    \lim_{x\rightarrow x_0^-}f(x)\qquad\textnormal{e}\qquad\lim_{x\rightarrow x_0^+}f(x)
\end{equation*}
non esiste oppure è infinito.
\subsubsection*{Discontinuità di terza specie o eliminabile}
Si dice che $x_0$ è un punto di discontinuità di terza specie o eliminabile se esiste ed è finito
\begin{equation*}
    \lim_{x\rightarrow x_0}f(x)
\end{equation*}
ma è diverso fa $f(x_0)$. 
\chapter{Derivate}
\begin{definition}
    Siano $a,b\in\mathbb{R}$ con $a<b$. Siano $f:(a,b)\rightarrow\mathbb{R}$ e $x_0\in(a,b)$. La seguente funzione 
    \begin{equation*}
        \varphi_{x_0}:(a,b)\setminus\rightarrow\mathbb{R},\varphi_{x_0}(x)=\frac{f(x)-f(x_0)}{x-x_0}\quad\forall x\in(a,b)\setminus\{x_0\}
    \end{equation*}
    viene detta rapporto incrementale.
\end{definition}
\begin{definition}
    Siano $a,b\in\mathbb{R}$ con $a<b$. Siano $f:(a,b)\rightarrow\mathbb{R}$ e $x_0\in(a,b)$. Si dice che $f$ è derivabile 
    in $x_0$ se esiste ed è finito 
    \begin{equation*}
        \lim_{x\rightarrow x_0}\frac{f(x)-f(x_0)}{x-x_0}
    \end{equation*}
    In tal caso si pone 
    \begin{equation*}
        f'(x_0)=\lim_{x\rightarrow x_0}\frac{f(x)-f(x_0)}{x-x_0}
    \end{equation*}
\end{definition}
Con il cambiamento di variabile si può scrivere:
\begin{equation*}
    f'(x_0)=\lim_{h\rightarrow 0}\frac{f(x_0+h)-f(x_0)}{h}
\end{equation*}
La retta passante per un punto $(x_0,y_0)$ di coefficiente angolare $m$ è 
\begin{equation*}
    y=y_0+m(x-x_0)
\end{equation*}
Nel nostro caso $y_0=f(x_0)$ ed $m=f'(x_0)$:
\begin{equation*}
    y=f(x_0)+f'(x_0)(x-x_0)
\end{equation*}
\begin{definition}
    Siano $a,b\in\mathbb{R}$ con $a<b$. Siano $f:(a,b)\rightarrow\mathbb{R}$ e $x_0\in(a,b)$.
    \begin{enumerate}
        \item Si dice che $f$ è derivabile da destra in $x_0$ se esiste ed è finito 
        \begin{equation*}
            \lim_{x\rightarrow x_0^+}\frac{f(x)-f(x_0)}{x-x_0}
        \end{equation*}
        In tal caso questo limite viene detto derivata destra di $f$ in $x_0$.
        \item Si dice che $f$ è derivabile da sinistra in $x_0$ se esiste ed è finito 
        \begin{equation*}
            \lim_{x\rightarrow x_0^-}\frac{f(x)-f(x_0)}{x-x_0}
        \end{equation*}
        In tal caso questo limite viene detto derivata sinistra di $f$ in $x_0$.
    \end{enumerate}
\end{definition}
\begin{theorem}
    Siano $a,b\in\mathbb{R}$ con $a<b$. Siano $f:(a,b)\rightarrow\mathbb{R}$ e $x_0\in(a,b)$. Allora $f$ è derivabile in
    $x_0$ se e soltanto se $f$ è derivabile sia da destra che da sinistra ed inoltre le derivate destra e sinistra coincidono.
\end{theorem}
\begin{theorem}[Continuità delle funzioni derivabili]
    Sia $I\subseteq\mathbb{R}$ un intervallo e sia $x_0\in I$. Sia $f:I\rightarrow\mathbb{R}$ derivabile in $x_0$. Allora 
    $f$ è continua in $x_0$.
\end{theorem}
\begin{proof}
    Se $f$ è derivabile in $x_0$, allora 
    \begin{equation*}
        \lim_{x\rightarrow x_0}\frac{f(x)-f(x_0)}{x-x_0}
    \end{equation*}
    deve esistere ed è finito. Ma il denominatore tende a 0 e l'unico modo per avere un rapporto con limite finito è che 
    il numeratore tenda anch'esso a 0. Ciò dimostra che $f$ è continua in $x_0$.
\end{proof}
\begin{theorem}
    Sia $I\subseteq\mathbb{R}$ un intervallo e sia $x_0\in I$. Siano $f:I\rightarrow\mathbb{R}$ e $g:I\rightarrow\mathbb{R}$
    derivabili in $x_0$. Allora valgono le seguenti conclusioni:
    \begin{enumerate}
        \item $f+g$ è derivabile in $x_0$ ed inoltre 
        \begin{equation*}
            (f+g)'(x_0)=f'(x_0)+g'(x_0)
        \end{equation*}
        \item sia $c\in\mathbb{R}$ allora $cf$ è derivabile in $x_0$ ed inoltre 
        \begin{equation*}
            cf'(x_0)=cf'(x_0)
        \end{equation*}
        \item $f-g$ è derivabile in $x_0$ ed inoltre 
        \begin{equation*}
            (f-g)'(x_0)=f'(x_0)-g'(x_0)
        \end{equation*}
        \item $fg$ è derivabile in $x_0$ ed inoltre 
        \begin{equation*}
            (fg)'(x_0)=f'(x_0)g(x_0)+f(x_0)g'(x_0)
        \end{equation*}
        \item $fg$ è derivabile in $x_0$ ed inoltre 
        \begin{equation*}
            \Bigl(\frac{f}{g}\Bigr)'(x_0)=\frac{f'(x_0)g(x_0)-f(x_0)g'(x_0)}{(g(x_0))^2}
        \end{equation*}
    \end{enumerate}
\end{theorem}
\begin{theorem}[Derivazione della funzione composta]
    Siano $I,J\subseteq\mathbb{R}$ intervalli e siano $f:J\rightarrow\mathbb{R}$ e $g:I\rightarrow J$.
    Sia $x_0\in I$ e sia $g$ derivabile in $x_0$.
    Posto $y_0=g(x_0)\in J$ si supponga che $f$ sia derivabile in $y_0$. Allora $f\circ g:I\rightarrow\mathbb{R}$ è 
    derivabile in $x_0$ ed inoltre 
    \begin{equation*}
        (f\circ g)'(x_0)=f'(y_0)g'(x_0)=f'(g(x_0))g'(x_0)
    \end{equation*}
\end{theorem}
\end{document}
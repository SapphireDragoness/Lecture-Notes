\documentclass[11pt]{book}
\usepackage[margin=.9in]{geometry}
\usepackage[italian]{babel}
\usepackage{tikz}
\usepackage{amsfonts} 
\usepackage{amsthm}
\usepackage{amsmath}
\usepackage{thmtools}
\usepackage{float}
\usepackage{hyperref}

\title{Appunti Analisi Matematica}

\newtheorem{theorem}{Teorema}[chapter]
\newtheorem*{proprietà}{Proprietà}
\newtheorem*{lemma}{Lemma}
\newtheorem*{corollary}{Corollario}
\newtheorem{definition}{Definizione}[chapter]

\hypersetup{
    colorlinks=true,
    linkcolor=black,
    filecolor=magenta,      
    urlcolor=cyan,
    }

\urlstyle{same}

\begin{document}
\pagenumbering{roman}
\tableofcontents
\
\newpage
\pagenumbering{arabic}
\chapter{Insiemi}
\section{Concetti di base sugli insiemi}
Un \textit{insieme} è un raggruppamento di oggetti detti \textit{elementi}, che possono essere di natura qualsiasi, Si 
dice che gli elementi di un insieme \textit{appartengono} all'insieme.
\subsubsection{Simboli}
Per indicare gli insiemi si usano solitamente lettere maiuscole, come
\begin{equation*}
    A,B,C\dots
\end{equation*}

Per indicare gli elementi di un insieme si usano solitamente lettere minuscole ($a,b,c\dots$) e per indicare che un elemento 
$x$ appartiene all'insieme $A$ scriviamo
\begin{equation*}
    x\in A \quad \textnormal{oppure} \quad A\ni x
\end{equation*}
\subsubsection{Rappresentazione}
È possibile rappresentare un insieme elencando i suoi elementi, in caso questo sia finito. Ad esempio
\begin{equation*}
    A=\{1,2,5\}
\end{equation*}
significa che l'insieme $A$ ha come elementi i numeri 1,2,5. In questo caso si dice che l'abbiamo definito \textit{per tabulazione}.

Oppure è possibile rappresentare un insieme descrivendolo mediante una proprietà che lo caratterizzi univocamente. Ad esempio
\begin{equation*}
    X=\{n:n \textnormal{ intero pari}\}
\end{equation*}

Un insieme privo di elementi viene detto \textit{insieme vuoto} e viene indicato con il simbolo $\emptyset$.
\subsubsection{Relazioni tra insiemi: inclusione}
\begin{definition}
    Si dice che un insieme $X$ è un sottoinsieme di un insieme $Y$ se ogni elemento di $X$ appartiene ad $Y$. Si utilizza 
    il simbolo di inclusione (larga) $X\subseteq Y$. Se $X$ non coincide con $Y$, si usa il simbolo di inclusione stretta
    $X\subset Y$.
\end{definition}
\subsubsection*{Relazioni tra insiemi: uguaglianza}
Due insiemi sono uguali quando possiedono gli stessi elementi.
Siano $X$ e $Y$ due insiemi. Allora $X=Y$ se e solo se $X\subseteq Y$ e $Y\subseteq X$.
\subsubsection*{Operazioni insiemistiche}
\begin{itemize}
    \item \textit{Unione}. L'unione di due insiemi $X$ e $Y$ è definita da:
    \begin{equation*}
        X\cup Y=\{x:x\in X \textnormal{ o } x\in Y\}
    \end{equation*}
    È l'insieme degli elementi che appartengono sia al primo sia al secondo insieme.
    \item \textit{Intersezione}. L'intersezione di due insiemi $X$ e $Y$ è definita da:
    \begin{equation*}
        X\cap Y=\{x:x\in X \textnormal{ e } x\in Y\}
    \end{equation*}
    È l'insieme degli elementi che appartengono al primo e al secondo insieme, intendendo la "o" in modo non esclusivo.
    \item \textit{Unione}. La differenza tra due insiemi $X$ e $Y$ è definita da:
    \begin{equation*}
        X\setminus Y=\{x:x\in X \textnormal{ e } x\notin Y\}
    \end{equation*}
    È l'insieme degli elementi che appartengono al primo ma non al secondo insieme.
    \item \textit{Complementare}. È un tipo particolare di differenza. Siano $X$ e $Y$ insiemi con $X\subseteq Y$. Si 
    definisce insieme complementare di $X$ in $Y$ l'insieme $Y\setminus X$. Si indica con il simbolo $X^C$.
\end{itemize}
\subsubsection*{Proprietà delle operazioni insiemistiche}
\begin{itemize}
    \item \textit{Proprietà dell'unione}.
        \begin{itemize}
            \item [] commutativa: $X\cup Y=Y\cup X$
            \item [] associativa: $/^(X\cup Y)\cup Z = X\cup (Y\cup Z)$
            \item [] idempotenza: $X\cup X=X$
        \end{itemize}
    \item \textit{Proprietà dell'intersezione}.
    \begin{itemize}
        \item [] commutativa: $X\cap Y=Y\cap X$
        \item [] associativa: $/^(X\cap Y)\cap Z = X\cap (Y\cap Z)$
        \item [] idempotenza: $X\cap X=X$
    \end{itemize}
    \item \textit{Proprietà distributive}.
    \begin{itemize}
        \item [] $X\cup (Y\cap Z)=(X\cup Y)\cap(X\cup Z)$
        \item [] $X\cap (Y\cup Z)=(X\cap Y)\cup(X\cap Z)$
    \end{itemize}
    \item \textit{Formule di De Morgan}.
    Siano $A,B\subseteq X$ e denotiamo con $A^C$ e $B^C$ i loro insiemi complementari in $X$.
    \begin{itemize}
        \item [] $(A\cup B)^C =A^C \cap B^C$
        \item [] $(A\cap B)^C =A^C \cup B^C$
    \end{itemize}
\end{itemize}
\section{Insieme delle parti e prodotto cartesiano}
\subsubsection*{Insieme delle parti}
Dato un insieme $X$, si dice \textit{insieme delle parti} di $X$ l'insieme di tutti i sottoinsiemi di $X$. Si indica con 
$\mathcal{P}^X$ o con $2^X$. Ad esempio, si consideri l'insieme $X=\{1,4,5\}$:
\begin{equation*}
    \mathcal{P}(X)=\{\emptyset,\{1\},\{4\},\{5\},\{1,4\},\{1,5\},\{4,5\},\{1,4,5\}\}
\end{equation*}
Se $X$ è un insieme finito con $n$ elementi, allora $\mathcal{P}(X)$ è un insieme finito con $2^n$ elementi. Se $X$ è un 
insieme infinito, allora anche $\mathcal{P}(X)$ è un insieme infinito.
\subsubsection*{Prodotto cartesiano}
Si dice \textit{coppia} un insieme ordinato di due elementi. Ad esempio:
\begin{align*}
    \{3,7\}&=\{7,3\}:\textnormal{ insieme non ordinato}\\
    \{3,7\}&\neq\{7,3\}:\textnormal{ insieme ordinato (coppia)}
\end{align*}
Dati due insiemi $X$ e $Y$, il \textit{prodotto cartesiano} di $X$ e $Y$ è l'insieme delle coppie $(x,y)$ in cui $x\in X$
e $y\in Y$.
\begin{equation*}
    X\times Y=\{(x,y):x\in X\textnormal{ e }y\in Y\}
\end{equation*}
L'insieme dei numeri reali è indicato con $\mathbb{R}$, e il prodotto cartesiano $\mathbb{R}\times\mathbb{R}$ viene 
indicato con $\mathbb{R}^2$.
\begin{equation*}
    \mathbb{R}^2=\mathbb{R}\times\mathbb{R}=\{(x,y):x,y\in\mathbb{R}\}
\end{equation*}
\section{Insiemi numerici fondamentali}
\begin{itemize}
    \item [$\mathbb{N}$] insieme dei numeri naturali (interi positivi, zero compreso)
    \begin{equation*}
        \mathbb{N}=\{0,1,2,3,4,5,6,\dots\}
    \end{equation*}
    \item [$\mathbb{Z}$] insieme dei numeri interi
    \begin{equation*}
        \mathbb{Z}=\{0,1,-1,2,-2,3,-3,\dots\}
    \end{equation*}
    \item [$\mathbb{Q}$] insieme dei numeri razionali (sono frazioni di numeri interi)
    \begin{equation*}
        \mathbb{Q}=\{\frac{a}{b}:a\in\mathbb{Z}\textnormal{ e }b\in\mathbb{Z}\setminus 0\}
    \end{equation*}
    \item [$\mathbb{R}$] insieme dei numeri reali
    \item [$\mathbb{I}$] insieme dei numeri irrazionali ($\mathbb{I}=\mathbb{R}\setminus\mathbb{Q}$)
\end{itemize}
\chapter{Funzioni}
\section{Il concetto di funzione}
\begin{definition}
    Siano $X,Y\neq\emptyset$. Si dice funzione da $X$ in $Y$ una legge che fa corrispondere ad ogni elemento di $X$ uno 
    ed un solo elemento di $Y$. L'insieme $X$ viene detto dominio della funzione mentre l'insieme $Y$ viene detto codominio 
    della funzione.
\end{definition}
Viene utilizzata la seguente notazione:
\begin{align*}
    f&:X\rightarrow Y \textnormal{ ($f$ definita da $X$ a $Y$)} & f:x&\mapsto f(x) \textnormal{ ($f$ associa $f(x)$ ad $x$)}
\end{align*}
Per ogni $x\in X$, l'elemento di $Y$ che la funzione $f$ fa corrispondere a $x$ viene detto \textit{immagine} di $x$ 
mediante $f$ e si indica con $f(x)$. La proprietà caratteristica di $f$, affinchè la si possa chiamare funzione, è 
l'\textit{univocità} della corrispondenza: assegnato l'elemento in "ingresso" $a\in A$, l'elemento in "uscita" $b=f(a)$ 
dev'essere univocamente determinato.
\begin{definition}
    Siano $X,Y \neq \emptyset$ e sia $f:X\rightarrow Y$. Si dice grafico di $F$ l'insieme
    \begin{equation*}
        \mathcal{G}(f)=\{(x,y)\in X\times Y:x\in X \textnormal{ e } y=f(x)\}
    \end{equation*}
\end{definition}
Ossia, il grafico di una funzione $f$ è l'insieme dei punti del piano di coordinate $(x,y)$ con $y=f(x)$ e $x$ variabile 
nel dominio $D$.
\begin{definition}
    Siano $X,Y \neq \emptyset$ e sia $f:X\rightarrow Y$. Siano rispettivamente $A\subseteq X$ e $B\subseteq Y$. Si dice 
    immagine di $A$ il sottoinsieme di $Y$ costruito dalle imagini dei singoli elementi di $A$. Tale insieme viene denotato 
    con $f(A)$. In altre parole si ha 
    \begin{equation*}
        f(A)=\{f(x)\in Y:x\in A\}\subseteq Y 
    \end{equation*}
    Si dice immagine inversa di $B$ o controimmagine di $B$ il sottoinsieme di $X$ costituito da quegli elementi di $X$ 
    la cui immagine appartiene a $B$. Tale insieme viene denotato con $f^{-1}(B)$. In altre parole si ha 
    \begin{equation*}
        f^{-1}(B)=\{x\in X:f(x)\in B\}\subseteq X
    \end{equation*}
\end{definition}
\begin{definition}
    Siano $X,Y \neq \emptyset$ e sia $f:X\rightarrow Y$. Sia inoltre $A\subseteq X$ insieme non vuoto. Si dice restrizione 
    di $f$ all'insieme $A$ la funzione 
    \begin{equation*}
        f_{|A}:A\rightarrow Y
    \end{equation*}
\end{definition}
\begin{theorem}
    Siano $X,Y \neq \emptyset$ e sia $f:X\rightarrow Y$. Siano inoltre $A,B\subseteq X$ e $C,D\subseteq Y$. Valgono le 
    seguenti conclusioni:
    \begin{enumerate}
        \item $f(A\cup B)=f(A)\cup f(B)$
        \item $f(A\cap B)=f(A)\cap f(B)$
        \item $f^{-1}(f(A))\supset A$
        \item $f^{-1}(C\cup D)=f^{-1}(C)\cup f^{-1}(D)$
        \item $f^{-1}(C\cap D)=f^{-1}(C)\cap f^{-1}(D)$
        \item $f(f^{-1}(C))\subseteq C$
    \end{enumerate}
\end{theorem}
\begin{definition}
    Siano $X,Y,Z$ insiemi non vuoti. Siano $f:y\rightarrow Z$ e $g:X\rightarrow Y$. Viene detta composizione di $f$ con 
    $g$ la funzione $f\circ g$ definita da
    \begin{equation*}
        f\circ g:X\rightarrow Z
    \end{equation*}
\end{definition}
\begin{definition}
    Siano $X$ e $Y$ insiemi non vuoti e sia $f:X\rightarrow Y$. Si dice che $f$ è iniettiva se elementi distinti del dominio 
    $X$ hanno immagini distinte:
    \begin{equation*}
        x_1,x_2\in X,x_1\neq x_2\Rightarrow f(x_1)\neq f(x_2)
    \end{equation*}
\end{definition}
\begin{definition}
    Siano $X$ e $Y$ insiemi non vuoti e sia $f:X\rightarrow Y$. Si dice che $f$ è suriettiva se ogni elemento del codominio 
    $Y$ è immagine di almeno un elemento del dominio $X$:
    \begin{equation*}
        \forall y\in Y, \exists x\in X:f(x)=y
    \end{equation*}
\end{definition}
\begin{definition}
    Siano $X$ e $Y$ insiemi non vuoti e sia $f:X\rightarrow Y$. Si dice che $f$ è biettiva (biiettiva o biunivoca) se è 
    contemporaneamente iniettiva e suriettiva. In altre parole, si dice che $f$ è biettiva se ogni elemento del codominio
    $Y$ è immagine di esattamente un elemento del dominio $X$:
    \begin{equation*}
        \forall y\in Y,\exists !x\in X:f(x)=y
    \end{equation*}
\end{definition}
Dato un insieme $X$ non vuoto, indicheremo con $Id_X:X\rightarrow X$ la \textit{funzione identità}, cioè quella funzione 
che ad ogni $x\in X$ associa se stesso:
\begin{equation*}
    Id_X(x)=x \forall x\in X
\end{equation*}
\begin{definition}
    Siano $X$ e $Y$ insiemi non vuoti e sia $f:X\rightarrow Y$. Si dice che $f$ è invertibile se esiste una funzione 
    $g:Y\rightarrow X$ tale che $f\circ g=Id_Y$ e $g\circ f=Id_X$. In altre parole si ha che:
    \begin{align*}
        f(g(y))=y\quad&\forall y\in Y &  g(f(x))=x\quad&\forall x\in X
    \end{align*}
    Se $f$ è invertibile, una tale funzione $g$ è unica e viene denominata funzione inversa di $f$. Si indica con il 
    simbolo $f^{-1}:Y\rightarrow X$.
\end{definition}
\begin{theorem}
    Siano $X$ e $Y$ insiemi non vuoti e sia $f:X\rightarrow Y$. Allora $f$ è invertibile se e solo se $f$ è biettiva.
\end{theorem}
\begin{definition}
    Siano $X$ e $Y$ due insiemi non vuoti. Si dice che $X$ e $Y$ hanno la stessa cardinalità se esiste una funzione 
    $f:X\rightarrow Y$ biettiva.
\end{definition}
Non tutti gli insiemi infiniti hanno la stessa cardinalità. Gli insiemi con la stessa cardinalità di $\mathbb{N}$ si 
dicono \textit{numerabili}. Gli insiemi $\mathbb{Z}$ e $\mathbb{Q}$ sono numerabili ed hanno quindi la stessa cardinalità 
di $\mathbb{N}$, nonostante $\mathbb{N}\subset\mathbb{Z}\subset\mathbb{Q}$. L'insieme $\mathbb{R}$ dei numeri reali non 
è numerabile: ha una cardinalità di \textit{ordine superiore} rispetto ad $\mathbb{N}$.
\chapter{Insiemi in $\mathbb{R}$}
\section{Intervalli}
Si dice \textit{intervallo} un sottoinsieme di $\mathbb{R}$ tale per cui ogni numero reale compreso tra due elementi di 
questo sottoinsieme appartiene al sottoinsieme medesimo.

Dati $a,b\in \mathbb{R}$ con $a<b$, si definiscono i seguenti intervalli:
\begin{align*}
    [a,b]&=\{x\in\mathbb{R}:a\leq x\leq b\} & (a,b)&=\{x\in\mathbb{R}:a< x< b\}
\end{align*}
\begin{definition}
    
\end{definition}
\end{document}
\documentclass[12pt]{article}
\usepackage[margin=1in]{geometry}

\title{RAID}

\begin{document}
\begin{center}
    \Huge\textbf{RAID}
\end{center}
Il termine RAID significa Redundant Array of Independent Disks: sono tecniche di organizzazione dei dischi che permettono di migliorare performance e affidabilità.\\
Al sistema operativo un RAID appare come un signolo disco logico; il RAID controller si occupa della traduzione delle richieste da logiche a fisiche.\\
Il controller RAID può essere implementato tramite hardware o software. Sono inoltre usati dischi \textbf{hot-spare} per sostituire dischi fallati automaticamente.
In un sistema RAID, la performance è migliorata tramite parallelismo e l'affidabilità tramite ridondanza.
\section*{Parallelismo}
Per ottenere parallelismo, viene utilizzata una tecnica detta \textbf{"data striping"}: i dischi logici sono divisi in \textbf{strip} e strip consecutive su dischi diversi sono dette \textbf{stripe}.
Il data striping può avvenire a livello di bit, byte o blocco di byte.\\
Per trovare disco e offset uno strip in un array di \textit{n} dischi senza ridondanza si possono usare le seguenti equazioni (IL = indirizzo logico):
\begin{itemize}
    \item Disco = IL mod n
    \item Offset = IL / n
\end{itemize}
Richieste di strip differenti su dischi differenti possono avvenire in parallelo, ma richieste che coinvolgono strip sullo stesso disco devono avvenire sequenzialmente.
Lo striping è vantaggioso nel caso vi siano richieste I/O di considerevoli dimensioni.\\
\section*{Ridondanza}
Un array di dischi è meno affidabile di un solo disco (il MTTF dell'array è equivalente al minore tra i dischi): si sfrutta la ridondanza.\\
Con la tecnica del mirroring ogni disco è duplicato: se un disco fallisce, si usa il suo clone. I dati sono perduti se il disco clone fallisce prima di essere rimpiazzato.
Ogni scrittura su un disco è effettuata anche sul suo clone: avvenendo in parallelo, il transfer rate è lo stesso.
\section*{RAID0}
RAID0 non offre alcun tipo di ridondanza (quindi i disk failure non sono tollerati), i blocchi di dati sono distribuiti tra i dischi (striping) seguendo una politica round-robin.\\
Le performance sono elevate, data la possibilità di eseguire richieste in parallelo, così come la capacità, ma non è consigliato per sistemi che richiedono pochi dati alla volta.
\section*{RAID1}
RAID1 usa mirroring per fornire ridondanza: sono tollerati tanti disk fault quanti sono i dischi presenti, ma serve il doppio dei dischi per implementarlo.
\section*{RAID2}
RAID2 usa striping a livello di bit e ECC (error correction codes): i dati sono divisi in gruppi di bit e per ogni gruppo è calcolato un ECC.
I bit di ECC e di dati sono memorizzati su dischi diversi: ci sono quindi \textit{n} dischi di dati e \textit{m} dischi per ECC.
Le richieste di lettura coinvolgono solo i dischi di dati, mentre quelle di scrittura entrambi, dato che gli ECC vanno ogni volta ricalcolati.\\
Nonostante l'alta affidabilità e la buona performance in lettura, RAID2 non è molto utilizzato, data la sua complessità e costo.
\section*{RAID3}
RAID3 usa striping a livello di bit più un bit di parità: sono necessari \textit{n} dischi di dati e 1 per parità.\\
Il bit di parità è un bit posto alla fine di una stringa binaria che indica se il numero di bit a 1 della stringa è pari o dispari. Due varianti:
\begin{itemize}
    \item even parity bit: bit di parità settato a 1 se il numero di bit a 1 è dispari
    \item odd parity bit: bit di parità setttato a 1 se il numero di bit a 1 è pari
\end{itemize}
I bit di parità in RAID3 permettono di correggere errori: sappiamo quale disco è fallito, quindi basta sostituire il bit appartenente al disco in questione per riprostinare la parità.\\
Come RAID2, anche RAID3 è poco utilizzato.
\section*{RAID4} 
RAID4 usa striping a livello di blocchi e un blocco di parità: sono necessari \textit{n} dischi di dati e 1 per parità.\\
Il blocco di parità è calcolato mettendo in XOR (0 se i bit sono uguali) due blocchi della stessa stripe.
Ci sono dua approcci per la computazione del blocco di parità:
\begin{itemize}
    \item parità additiva: tutte le strip sono lette e messe in XOR (utile per operazioni che coinvolgono molti dischi)
    \item parità sottrattiva: solo la strip da sostituire, la nuova strip e il blocco di parità sono messi in XOR (utile per operazioni che coinvolgono pochi dischi)
\end{itemize}
Il cutoff per l'uso di parità sottrattiva è $\frac{n}{2}-1$.\\
RAID4 offre buona performance su grandi richieste, ma non può gestire scritture di pochi dati su stripe diverse, dato che devono scrivere sullo stesso disco di parità.
\section*{RAID5}
RAID5 usa striping a livello di blocchi e blocco di parità distribuito: interlacciando blocchi di dati e parità di risolve il problema del collo di bottiglia che si presenta in RAID4.
Tuttavia, il problema delle scritture di pochi dati persiste.
\section*{RAID6}
RAID6 usa striping a livello di blocchi e blocchi di parità distribuiti: ogni stripe ha due blocchi di parità contenuti in dischi diversi.
Molto affidabile, ma la performance in scrittura è peggiore di quella di RAID5.
\section*{RAID ibridi}
RAID 0+1 (mirror of stripes) consiste nello striping e mirroring di \textit{n} dischi.\\
RAID 1+0 (stripe of mirrors) consiste nel mirroning e poi striping di \textit{n} dischi.\\
\end{document}

\documentclass[11pt]{article}
\usepackage[margin=.8in]{geometry}
\usepackage[italian]{babel}
\usepackage{amsmath}
\usepackage{automata}

\title{Appunti FLT}

\begin{document}
\section*{Nota ai lettori}
Questi appunti sono basati sulle lezioni dell A.A. 2023/2024, integrate con passi tratti dal libro "Linguaggi Formali e 
Compilazione", e formattati seguendo la suddivisione in paragrafi di quest'ultimo.
\section{Teoria formale del linguaggio}
\subsection{Alfabeto e linguaggio}
Un \textbf{alfabeto} è un insieme finito di elementi chiamati \textbf{simboli terminali} o \textbf{caratteri}.
$\Sigma=\{a_1,a_2,\dots,a_k\}$ è un alfabeto composto da $k$ elementi (la sua cardinalità è $k$). Una \textbf{stringa} (o 
\textbf{parola}) è una sequenza, ovvero un insieme ordinato eventualmente con ripetizioni, di caratteri.

Un \textbf{linguaggio} è un insieme di stringhe di un alfabeto specifico. Dato un linguaggio, una stringa che gli appartiene 
è detta \textbf{frase}. 

La \textbf{cardinalità} di un linguaggio è definita dal numero di frasi che contiene. Se la cardinalità è finita, il linguaggio 
si dice \textbf{finito}. 

Un linguaggi finito è una collezione di parole, solitamente chiamate \textbf{vocabolario}. Il linguaggio che non contiene 
frasi è chiamato \textbf{insieme vuoto} o \textbf{linguaggio $\emptyset$}.

La \textbf{lunghezza} $|x|$ di una stringa $x$ è il numero di caratteri che contiene.
\subsubsection{Operazioni sulle stringhe}
Date le stringhe
\begin{align*}
    x=&a_1a_2\dots a_h   &   y=&b_1b_2\dots b_k
\end{align*}
la \textbf{concatenazione}, indicata con $\cdot$, è definita come:
\begin{align*}
    x\cdot y=a_1a_2\dots a_hb_1b_2\dots b_k
\end{align*}
La concatenazione non è commutativa, ma è associativa.
\subsubsection{Stringa vuota}
La \textbf{stringa vuota} (o \textbf{nulla}), denotata con $\epsilon$, soddisfa l'identità:
\begin{align*}
    x\cdot \varepsilon=\varepsilon \cdot x = x
\end{align*}
La stringa vuota non deve essere confusa con l'insieme vuoto; infatti, l'insieme vuoto è un linguaggio che non contiene 
stringhe, mentre il set $\{\varepsilon\}$ ne contiene una, la stringa vuota.
\subsubsection{Sottostringa}
Sia la stringa $x=uyv$ il prodotto della concatenazione delle stringhe $u$, $y$ e $v$: le stringhe $u$, $y$ e $v$ sono 
\textbf{sottostringhe} di $x$. In questo caso, la stringa $u$ è un \textbf{prefisso} di $x$ e la stringa $v$ è un 
\textbf{suffisso} di $x$. Una sottostringa non vuota è detta \textbf{propria} se non coincide con $x$.
\subsubsection{Inversione di stringa}
L'\textbf{inverso} di una stringa $x=a_1a_2\dots a_h$ è la stringa $x^R=a_ha_{h-1}\dots a_1$.
\subsubsection{Ripetizione}
La potenza m-esima $x^m$ di una stringa $x$ è la concatenazione di $x$ con se stessa per $m-1$ volte. Esempi:
\begin{align*}
    x=&ab    &   x^0=&\varepsilon    &   x^2=&(ab)^2=abab\\  
\end{align*}
\subsection{Operazioni sul linguaggio}
L'inverso $L^R$ di un linguaggio $L$ è l'insieme delle stringhe che sono l'inverso di una frase di $L$.

\section{Automi a pila e parsing}
\subsection{Automi a pila}
Gli \textbf{automi a pila} sono automi a stati finiti che utilizzano una \textbf{pila} (stack) come memoria aggiuntiva.
È possibile svolgere tre operazioni sullo stack:
\begin{itemize}
    \item \textbf{push}: inserisce un simbolo in cima allo stack
    \item \textbf{pop}: rimuove un simbolo dalla cima dello stack (se questo non è vuoto)
    \item \textbf{test di vuotezza}: controlla la presenza di simboli nello stack 
\end{itemize}
\subsubsection{Definizione}
Un automa a pila è definito dalla 7-upla: $\langle Q,\Sigma,\Gamma,\delta,q_0,Z_0,F \rangle$.
\begin{itemize}
    \item $Q$: insieme degli stati
    \item $\Sigma$: alfabeto che descrive il linguaggio
    \item $\Gamma$: alfabeto della pila 
    \item $\delta$: funzione di transizione
    \item $q_0$: stato iniziale 
    \item $Z_0$: fondo della pila 
    \item $F$: stato (o stati) finale
\end{itemize}
L'input è una tripla, denotata come:
\begin{equation*}
    (q,a,A)\rightarrow(x,XX)
\end{equation*}
\end{document}
\documentclass[12pt]{article}
\usepackage[margin=0.8in]{geometry}
\usepackage{listings}

\lstdefinestyle{mystyle}{
    basicstyle=\ttfamily\footnotesize,
    breakatwhitespace=false,         
    breaklines=true,                    
    keepspaces=true,                  
    showspaces=false,                
    showstringspaces=false,
    showtabs=false,                  
    tabsize=2
}

\lstset{style=mystyle}

\title{Esame 28/6/2023}

\begin{document}
\begin{center}
    \large\textbf{Esame 28/6/2023}
\end{center}
\subsection*{Esercizio 1}
Le funzioni di libreria esistono per svolgere operazioni tramite codice scritto in precedenza, in modo da evitare di 
riscrivere una funzione ogni volta che si vuole eseguire una determinata operazione. Le chiamate di sistema permettono di 
interagire con dispositivi e parti del sistema operativo non altrimenti accedibili in modalità utente, in maniera controllata. 
I programmi ottengono quanto desiderato senza dover conoscere come comandare i dispositivi o le specifiche del sistema 
operativo, e senza poterli modificare a piacimento.\\\\
Il passaggio da Running a Ready è causato dallo scadere del quanto di tempo ed è realizzato tramite l'interruzione da timer; 
se un processo CPU bound non fa operazioni sospensive o non ci sono altri interrupt, l'interruzione da timer è l'unico 
modo per far intervenire il sistema operativo e dare la CPU ad un altro processo.
\subsection*{Esercizio 2}
Con la paginazione, ad ogni processo è assegnato uno spazio di indirizzi virtuali, diviso in parti di ugual dimensione dette 
pagine. La memoria fisica è divisa in porzioni di dimensione prefissate dette frame. 
Gli indirizzi virtuali non vanno direttamente sul bus di memoria ma vanno alla MMU che mappa indirizzi virtuali a fisici. 
Non tutte le pagine appartenenti ad un processo devono essere presenti in memoria contemporaneamente: esse vengono caricate 
in memoria a seconda delle richieste del processo. La tabella delle pagine contiene il mapping tra pagine virtuali e fisiche.
Una voce nella tabella delle pagine contiene: numero di page frame, bit che identifica se la pagina è valida o meno, bit 
che indicano i tipi di accesso permessi, bit modifica e riferimento che tengono traccia dell'uso della pagina. Per aumentare 
la velocità di traduzione degli indirizzi, viene usata una componente hw chiamata TLB.\\\\
Per indirizzare i byte all'interno di una pagina servono 12 bit in quanto $4K=2^{12}$. Negli indirizzi logici e fisici, 
i 12 bit meno significativi rappresentano l'offset all'interno della pagina/frame, mentre i restanti bit indicizzano la 
specifica pagina/frame. In particolare, negli indirizzi logici $48-12=36$ bit servono per specificare la pagina logica e 
negli indirizzi fisici $36-12=24$ bit servono ad identificare il frame. La tabella delle pagine è indicizzata rispetto al 
numero di pagina logica e ciascuna sua entry mantiene l'identificativo del frame dove è memorizzata in RAM la pagina logica.
L'occupazione di ciascuna entry è complessivamente di 4 byte: 3 byte per memorizzare l'indice del frame più un byte aggiuntivo 
per i bit d'utilità. Il numero totale di entry è uguale al numero totale di pagine logiche ossia $2^{36}$. L'occupazione 
totale è quindi $4*2^{36}$ byte.
\subsection{}
\end{document}
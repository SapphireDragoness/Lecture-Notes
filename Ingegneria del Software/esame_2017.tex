\documentclass[11pt]{article}
\usepackage[margin=.8in]{geometry}
\usepackage[italian]{babel}

\title{Esame 2017}

\begin{document}
\subsubsection*{Fornire una definizione di Ingegneria del software:}
Con Ingegneria del software s'intende l'applicazione del processo dell'Ingegneria alla produzione di sistemi software. 
L'Ingegneria del software dovrebbe limitare la presenza di difetti alla consegna del sistema software e fare in modo che
il sistema soddisfi il committente.
\subsubsection*{Indicare gli scopi principali di ogni fase del processo software:}
Il processo software è diviso in 5 fasi: specifica, progettazione, implementazone, collaudo e manutenzione.

Durante la fase di specifica viene compiuto lo studio di fattibilità, vengono dedotti ed analizzati i requisiti funzionali 
e non funzionali ed infine vengono validati. È un processo ciclico.

Durante la fase di progettazione determinata la struttura del sistema, il suo deployment ed il suo funzionamento.

Durante la fase di implementazione viene implementato il sistema, ossia vengono programmate le varie componenti specificate 
nel documento dei requisiti.

Durante la fase di collaudo vengono eseguiti l'ispezione ed il testing del sistema per assicurare il corretto funzionamento 
di tutte le componenti ed identificare e correggere eventuali errori.

Durante la fase di manutenzione il sistema, ormai rilasciato, viene modificato per adattarlo alle esigenze dei 
committenti o per correggere errori non individuati durante la fase di testing.
\subsubsection*{Supponendo di dover sviluppare un sistema software per la vendita on-line di prodotti, fornire tre
esempi di requisito funzionale, un esempio di requisito non funzionale di prodotto, un esempio di
requisito non funzionale organizzativo, un esempio di requisito non funzionale esterno:}
Requisiti funzionali:
\begin{itemize}
    \item lista prodotti 
    \item login utente
    \item effettua ordine
\end{itemize}

Requisiti non funzionali:
\begin{itemize}
    \item di prodotto: gestione delle richieste rapida
    \item organizzativo: database implementato con MongoDB 
    \item esterno: compatibilità con PayPal
\end{itemize}
\subsubsection*{Indicare gli scopi principali di ogni fase del processo di Specifica:}
Il processo di specifica ha 4 fasi: studio di fattibilità, deduzione dei requisiti, analisi dei requisiti e validazione 
dei requisiti.

Durante lo studio di fattibilità viene valutata la possibilità di sviluppare il sistema e dei suoi vantaggi per il committente,
a questo scopo viene preparato un rapporto di fattibilità.

Durante la fase di deduzione dei requisiti si compie un'indagine per dedurre i requisiti. Le informazioni vengono raccolte 
dallo studio del dominio applicativo e dialogo con stakeholder.

Durante la fase di analisi dei requisiti questi vengono classificati ed organizzati, stabilendo priorità e modellandoli 
con l'utilizzo di strumenti CASE.

Durante la fase di validazione dei requisiti si verifica che il documento dei requisiti non contenga errori di specifica 
che potrebbero impattare fasi successive del prodotto software.
\end{document}
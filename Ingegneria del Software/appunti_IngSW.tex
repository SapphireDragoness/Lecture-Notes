\documentclass[11pt]{article}
\usepackage[margin=.8in]{geometry}
\usepackage[italian]{babel}
\usepackage{graphicx}

\title{Appunti Ingegneria del Software}

\begin{document}
\section{Introduzione}
\subsection{Sistemi software}
Un \textit{sistema software} è un insieme di componenti software che funzionano in modo coordinato allo scopo di informatizzare 
una certa attività. La realizzazione di un sistema software richiede l'impiego di un gruppo di lavoro, nel quale ogni 
persona ricopre un ruolo ben preciso e le attività dei vari gruppi vanno coordinate, e tempo da dedicare alle varie fasi 
di sviluppo.

Esistono due categorie di sistemi software: i \textit{sistemi generici}, definiti in base alle tendenze di mercato, e i 
\textit{sistemi customizzati}, richiesti da uno specifico cliente (il committente).
\subsection{Il processo software}
Con \textit{ingegneria del software} si intende l'applicazione del processo dell'Ingegneria alla produzione di sistemi 
software. Il processo è suddiviso in:
\begin{itemize}
    \item specifica: definizione dei requisiti funzionali e non funzionali 
    \item progettazione: si definiscono architettura, controllo, comportamento dei componenti, strutture dati, algoritmi,
    struttura del codice, interfaccia utente
    \item implementazione: scrittura del codice e integrazione dei moduli 
    \item collaudo: si controlla se il sistema ha difetti di funzionamento e se soddisfa i requisiti 
    \item manutenzione: modifiche del sistema dopo la consegna 
\end{itemize}
\subsection{Gestione del processo}
L'ingegneria del software si occupa anche della gestione del progetto che si svolge in parallelo al processo software.
Le principali attività di gestione sono l'\textit{assegnazione} di risorse (umane, finanziarie...), la \textit{stima del tempo}
necessario per ogni attività, la \textit{stima dei costi} e la \textit{stima dei rischi}.
\section{Specifica}
La \textit{specifica} è l'insieme di attività necessarie per generare il documento dei requisiti che descrive i 
\textit{requisiti funzionali} e i \textit{requisiti non funzionali}: descrive il "cosa" il sistema deve fare, non il "come".
I requisiti servono per una proposta di contratto e modellare fasi successive del processo software.
\subsection{Requisiti funzionali}
I requisiti funzionali sono i servizi che il cliente richiede al sistema. Per ogni servizio si descrive:
\begin{itemize}
    \item cosa accade nell'interazione tra utente e sistema 
    \item cosa accade in seguito ad un certo input o stimolo
    \item cosa accade in particolari sistuazioni, ad esempio in caso di eccezioni
\end{itemize}
Non viene descritto come funziona internamente il sistema, in quanto è oggetto della successiva fase di progettazione.
\subsection{Requisiti non funzionali}
I requisiti non funzionali sono divisi in tre categorie: \textit{requisiti di prodotto}, \textit{requisiti organizzativi}
e \textit{requisiti esterni}.

I requisiti di prodotto sono attributi che definiscono la qualità del sistema. Una \textit{proprietà complessiva} riguarda
il sistema nel suo complesso; una \textit{proprietà emergente} è una proprietà che "emerge" dal funzionamento del sistema, 
dopo che è stato implementato.

I requisiti organizzativi sono caratteristiche riguardanti le fasi del processo software o la gestione del progetto. I 
\textit{requisiti di sviluppo} sono i metodi e le tecniche di sviluppo utilizzati; i \textit{requisiti gestionali} sono
le risorse utilizzate.

I requisiti esterni derivano da fattori esterni al sistema e al processo software. Essi sono i requisiti di compatibilità
con altri sistemi e aspetti giuridici.
\begin{center}
    \includegraphics[scale=0.4]{reqnfunz.jpeg}
\end{center}
\subsubsection*{Usabilità}
L'\textit{usabilità} è il grafo di facilità con cui l'utente riesce a comprendere l'uso del software. Il sistema deve avere 
un'interfaccia utente intuitiva ed curata, in quanto è fattore critico per il successo di un prodotto.
L'uso del sistema deve essere ben documentato, per permettere all'utente di apprendere velocemente l'uso del prodotto.
Il \textit{training} degli utenti può migliorare l'usabilità del prodotto.
\subsubsection*{Mantenibilità}
la \textit{mantenibilità} è il grado di facilità di manutenzione. Le cause della manutenzione sono molteplici, e deve 
essere possibile l'evoluzione del software per soddisfare i requisiti nel tempo.
\subsubsection*{Portabilità}
La \textit{portabilità} è la capacità di migrazione da un ambiente ad un altro.
\subsubsection*{Recoverability}
La \textit{recoverability} è la capacità di ripristinare lo stato e i dati del sistema dopo che si è verificato un fallimento.
\subsubsection*{Efficienza}
L'\textit{efficienza} è il livello di prestazioni del sistema, e può essere misurato in vari modi: tempo di risposta, 
numero medio di richieste... 
\subsubsection*{Affidabilità}
L'\textit{affidabilità} è il grado di fiducia con cui si ritiene che il sistema svolga in modo corretto la propria funzione.
Ci sono varie misure di affidabilità:
\begin{itemize}
    \item \textit{reliability}: capacità di fornire i servizi in modo continuativo per una certa durata di tempo 
    \item \textit{availability}: capacità di fornire i servizi nel momento richiesto
    \item \textit{safety}: capacità di operare senza causare danni materiali
    \item \textit{security}: capacità di proteggersi da intrusioni e attacchi
\end{itemize}
Un sistema è definito \textit{critico} quando il suo non corretto funzionamento può provocare conseguenze "disastrose"
a persone e ambiente (\textit{safety critical system}) o perdite economiche (\textit{business critical system}).
Il costo cresce in modo esponenziale rispetto al grado di affidabilità richiesto.
\subsection{Processo di specifica}
Il \textit{processo di specifica} è il processo per generare il documento dei requisiti, ed è diviso in più fasi.
Lo stesso requisito viene definito con due gradi di dettaglio diversi. Il \textit{requisito utente} è descritto ad alto 
livello, in linguaggio naturale, ed è il risultato della deduzione dei requisiti.
Il \textit{requisito di sistema} è descritto dettagliatamente, fornendo tutti i dettagli necessari per la fase di progettazione,
ed è il risultato dell'analisi dei requisiti.
\subsubsection*{Studio di fattibilità}
Lo \textit{studio di fattibilità} è la valutazione della possibilità di sviluppare il sistema e dei suoi vantaggi per 
il committente. Si decide se la costruzione del sistema è fattibile date le risorse disponibili e se il sistema è 
effettivamente utile al cliente.
Per svolgere lo studio si raccolgono informazioni e si prepara un rapporto di fattibilità, che contiene la valutazione 
della possibilità di costruire un sistema e dei vantaggi che possono derivare dalla sua introduzione.
\subsubsection*{Deduzione dei requisiti}
La \textit{deduzione dei requisiti} è la raccolta di informazioni da cui dedurre quali sono i requisiti.
Le informazioni si possono raccogliere mediante uno studio del dominio applicativo del sistema richiesto, il dialogo con 
stakeholder, studio di sistemi simili già realizzati e studio di sistemi con cui dovrà interagire quello da sviluppare.

Il \textit{dominio applicativo} è l'insieme di entità reali su cui il sistema software ha effetto.

Uno \textit{stakeholder} è, in ambito economico, il soggetto che può influenzare il successo di un'impresa o che ha 
interessi nelle decisioni dell'impresa; in ambito del processo software sono persone che possono influenzare il processo 
o che hanno interesse nelle decisioni assunte in esso.

È possibile dialogare con gli stakeholder tramite \textit{interviste}, nelle quali viene chiesto di raccontare attraverso 
degli esempi reali come l'attività lavorativa funziona realmente, e tramite \textit{etnografia}, l'osservazione dei potenziali 
utenti nello svolgimento delle loro mansioni.

Il dialogo con gli stakeholder presenta vari problemi, in quanto non sono in grado di indicare chiaramente cosa vogliono 
dal sistema, omettendo informazioni ritenute ovvie ed utilizzando terminologia non adatta. Inoltre, lo stesso requisito 
può essere espresso da più stakeholder in maniera differente, ed addirittura essere in conflitto.

Molti problemi scaturiscono dal \textit{linguaggio naturale}, in quanto una descrizione ad alto livello di un requisito 
può generare confusione. La soluzione è quella di utilizzare il linguaggio in modo coerente, evitando gergo tecnico ed 
illustrando i requisiti tramite semplici diagrammi.
\subsubsection*{Analisi dei requisiti}
La \textit{analisi dei requisiti} è l'organizzazione, negoziazione e modellazione dei requisiti.
Comprende:
\begin{itemize}
    \item \textit{classificazione e organizzazione dei requisiti}
    \item \textit{assegnazione di priorità ai requisiti}: si stabilisce il grado di rilevanza di ogni requisito 
    \item \textit{negoziazione dei requisiti}
    \item \textit{modellazione analitica dei requisiti}: produzione di modelli che rappresentano o descrivono nel dettaglio 
    i requisiti 
\end{itemize}
I \textit{requisiti di sistema} sono l'espansione dei requisiti utente, e formano la base per la progettazione. Il linguaggio
naturale non è adatto alla definizione di un requisito di sistema, quindi è necessario usare template, modelli grafici o 
notazione matematica.

Il \textit{modello data-flow}, detto anche pipe \& filter, permette di modelalre il flusso e l'elaborazione dei dati, ma 
non prevede la gestione degli errori. L'elaborazione è di tipo batch: input $\rightarrow$ elaborazione $\rightarrow$ output.

I requisiti non funzionali si possono specificare definendone delle misure quantitative:
\begin{itemize}
    \item \textit{efficienza}: tempo di elaborazione delle richieste, occupazione di memoria
    \item \textit{affidabilità}: probabilità di malfunzionamento, disponibilità
    \item \textit{usabilità}: tempo di addestramento, aiuto contestuale
\end{itemize}

Il \textit{documento dei requisiti} contiene il risultato della deduzione e dell'analisi, ed è la dichiarazione ufficiale 
di ciò che si deve sviluppare. Il documento contiene una breve introduzione che descrive le funzionalità del sistema, un 
glossario contenente le definizione di termini tecnici, i requisiti utente e i requisiti di sistema, correlati con modelli 
UML. Il documento è letto da tutte le figure coinvolte nella realizzazione del progetto.
\subsubsection*{Validazione dei requisiti}
La \textit{validazione dei requisiti} è la verifica del rispetto di alcune proprietà da parte del documento dei requisiti,
serve ad evitare la scoperta di \textit{errori di specifica} durante le fasi successive del processo software. Sono da 
verificare le segueni proprietà:
\begin{itemize}
    \item \textit{completezza}: tutti i requisiti richiesti dal committente devono essere documentati
    \item \textit{coerenza}: la specifica dei requisiti non deve contenere definizioni tra loro contraddittorie
    \item \textit{precisione}: l'interpretazione di una definizione di requisito deve essere unica
    \item \textit{realismo}: i requisiti devono essere implementati date le risorse disponibili
    \item \textit{tracciabilità}
\end{itemize}

Quando si modifica un requisito bisogna valutarne l'impatto sul resto della specifica: è quindi necessario tracciarlo.
Vari tipi di \textit{tracciabilità}:
\begin{itemize}
    \item \textit{tracciabilità della sorgente}: reperire la fonte d'informazione relativa al requisito
    \item \textit{tracciabilità dei requisiti}: individuare i requisiti dipendenti
    \item \textit{tracciabilità del progetto}: individuare i componenti del sistema che realizzano il requisito
    \item \textit{tracciabilità dei test}: individuare i test-case usati per collaudare il requisito
\end{itemize}

Per validare i requisiti si può impegare un gruppo di \textit{revisori} che ricontrolli i requisiti e \textit{costruire 
dei prototipi}.
\section{Progettazione}
\subsection{Attività di progettazione}
Durante la fase di \textit{progettazione architetturale} viene definita la struttura del sistema, come questo verrà 
distribuito e come il sistema si dovrà comportare. Sono inoltre progettate le strutture dati, gli algoritmi e la GUI.
\subsection{Progettazione architetturale}
\subsubsection*{Strutturazione}
Il sistema può essere strutturato in vari sottosistemi (strati), tipicamente tre. Ogni strato interagisce con gli strati 
adiacenti. Gli strati sono:
\begin{itemize}
    \item \textit{presentazione}: l'interfaccia utente, raccoglie i dati dall'utente
    \item \textit{elaborazione}:  elabora i dati in input e produce dati in output
    \item \textit{gestione dei dati}: il database
\end{itemize}

Un \textit{sottosistema} è la parte del sistema dedicata a svolgere una certa attività, mentre un \textit{modulo} è la 
parte del sottosistema dedicata a svolgere particolari funzioni legate all'attività del sottosistema.
\subsubsection*{Deployment}
Con \textit{deployment} si intende la distribuzione dei componenti in vari dispositivi hardware. Nel \textit{deployment 
a 1-tier} i tre stati del sistema sono concentrati su un dispositivo, in quello a \textit{2-tiers} su due e in quello a 
\textit{3-tiers} su tre.

Con deployment a 2-tiers si possono avere due soluzioni: \textit{fat client} e \textit{thin client}.
Nel modello \textit{thin client}, il server si occupa dell'elaborazione e della gestione dei dati, mentre il client si 
occupa della presentazione.
Nel modello \textit{fat client}, il server si occupa della gestione dei dati, mentre il client si occupa della presentazione 
e dell'elaborazione.
\subsubsection*{Metodo di controllo}
Un componente fornisce servizi ad altri componenti. Un'\textit{interfaccia} è un insieme di operazioni che il componente
mette a disposizione di altri componenti ed è condivisa con i componenti che lo invocano. Un \textit{corpo} è la parte 
interna del componente e non è conosciuto agli altri componenti. La separazione tra interfaccia e corpo è detta 
\textit{information hiding}.

Esistono diversi stili di controllo (attivazione) tra componenti:
\begin{itemize}
    \item controllo \textit{centralizzato}: è presente un componente detto \textit{controllore}, che controlla l'attivazione 
    e il coordinamento degli altri componenenti
    \item controllo \textit{basato su eventi}: è basato su eventi esterni (ad esempio un segnale) ed ogni componente si
    occupa di gestire determinati eventi; il gestore degli eventi è detto \textit{broker}, che rileva l'evento e lo notifica 
    tramite broadcast (selettivo o non selettivo)
    \item controllo \textit{call-return}: il controllo passa dall'alto verso il basso
    \item controllo \textit{client-server}: un componente client chiede un servizio ad un componente server attraverso 
    una chiamata di proceduta e il componente server risponde
\end{itemize}
\subsubsection*{Modellazione del comportamenti ad oggetti}
I componenti del sistema sono considerati come oggetti che interagiscono. Un \textit{oggetto} è definito da \textit{attributi} e 
\textit{operazioni}. Gli oggetti comunicano tra di loro attraverso lo scambio di messaggi.
\section{Collaudo}
La fase di collaudo avviene alla fine dell'implementazione del sistema. Si ricercano e correggono difetti, si controlla 
che il prodotto realizzi ogni servizio senza malfunzionamenti (fase di \textit{verifica}) e che soddisfi i requisiti del 
committente (fase di \textit{validazione}).
\subsection{Ispezione}
\subsubsection*{Motivazioni}
L'\textit{ispezione} è una tecnica statica di analisi del codice, basata sulla lettura di questo e della documentazione.
L'ispezione è meno costosa del testing e può essere eseguita su una versione incompleta del sistema, ma alcuni requisiti 
non funzionali non si possono collaudare solo con l'ispezione (efficienza, affidabilità...).

Un team analizza il codice e segnala i possibili difetti. Viene seguita una checklist che indica i possibili difetti da 
investigare: 
\begin{itemize}
    \item errori nei dati 
    \item errori di controllo 
    \item errori di I/O 
    \item errori di interfaccia 
    \item errori nella gestione della memoria 
    \item errori di gestione delle eccezioni 
\end{itemize}
\subsubsection*{Ruoli nel team di ispezione}
Il team di ispezione è composto da \textit{autori} del codice, che correggono i difetti rilevati durante l'ispezione, 
\textit{ispettori}, che trovano difetti, e \textit{moderatore}, che gestisce il processo di ispezione.
\subsubsection*{Processo di ispezione}
Durante la fase di \textit{pianificazione}, il moderatore seleziona gli ispettori e controlla che il materiale sia completo.

Durante la fase di \textit{introduzione}, il moderatore organizza una riunione preliminare con autori e ispettori, nella 
quale è discusso lo scopo del codice e la checklist da seguire.

Durante la fase di \textit{preparazione individuale}, gli ispettori studiano il materiale e cercano difetti nel codice 
in base alla checklist ed all'esperienza personale.

Durante la fase di \textit{riunione di ispezione}, gli ispettori indicano i difetti individuati.

Durante la fase di \textit{rielaborazione}, il programma è modificato dall'autore per correggere i difetti.

Durante la fase di \textit{prosecuzione}, il moderatore decide se è necessario un ulteriore processo di ispezione.
\subsubsection*{Analisi statica del codice}
Gli strumenti CASE supportano l'ispezione del codice eseguendo su di esso l'\textit{analisi del flusso di controllo}, 
che assicura l'assenza di cicli con uscire multiple, salti incondizionati e codice non raggiungibile; l'\textit{analisi 
dell'uso dei dati}, che assicura l'assenza di problemi legati alle variabili; l'\textit{analisi delle interfacce} e 
l'\textit{analisi della gestione della memoria}. Due strumenti per l'analisi statica sono il compilatore gcc e SonarQube.

L'analisi statica precede l'ispezione del codice fornendo informazioni utili all'individuazione dei difetti, ma non è 
sufficiente.
\subsection{Testing}
\subsubsection*{Processo di testing}
Durante la fase di \textit{testing}, sono preparati ed eseguiti i test case, ed i loro risultati in output sono confrontati 
con quelli attesi.

Nella fase di \textit{debugging} viene controllata l'esecuzione del programma e il valore delle variabili, in modo da 
individuare l'errore. L'errore viene corretto e viene eseguito un \textit{testing di regressione}, nel quale si ripete 
l'ultimo test-case e tutti quelli precedenti.
\subsubsection{Test-case}
Un \textit{test-case} è composto da dati in input e dati in output attesi. 

Esistono due approcci per scegliere test-case. Nel testing \textit{black box}, la scelta dei test-case è basata sulla 
conoscenza di quali sono i dati in input e quelli in output. 

Dato l'insieme dei dati in input e l'insieme dei dati in 
output, una \textit{partizione di equivalenza} è un sottoinsieme dei dati in input per cui il sistema produce sempre lo 
stesso dato in output (oppure simili). Si può fare black box testing usando delle partizioni di equivalenza.
Per scrivere test-case, si individuano le partizioni di equivalenza e si scelgono un numero finito di test-case: in 
particolare si scelgono test-case con dati I/O al confine delle partizioni di equivalenza (\textit{Boundaty Value Analysis}).

Nel testing \textit{white box}, la scelta dei test-case è basata sulla struttura del codice. L'obiettivo è testare ogni 
parte del codice. Il codice viene rappresentato con un \textit{flow graph}, un grafo che rappresenta i possibili cammini 
nel codice.
\subsubsection{Complessità ciclomatica}
Nel codice, un cammino si dice \textit{indipendente} se introduce almeno una nuova sequenza di istruzioni o una nuova 
condizione. Il numero di cammini indipendenti equivale alla \textit{complessità ciclomatica} (CC) del flow graph. CC è 
il numero minimo di test-case richiesti per eseguire almeno una volta ogni parte del codice. I \textit{dynamic program
analyser} sono strumenti CASE che, dato un test-case, indicano quali parti del codice sono state interessate da un 
test-case e quali sono ancora da testare.
\subsubsection*{Testing d'integrazione}
I componenti possono essere implementati gradualmente oppure contemporaneamente. Il testing riguarda ciascun componente 
e la sua integrazione nel sistema. Prima vengono testati i componenti in maniera isolata, poi viene testato il sistema 
con le componenti integrate fino ad allora. Il testing compiuto sul sistema con tutte le componenti è detto \textit{release 
testing}.

Lo scopo del \textit{testing d'integrazione} è verificare l'interfacciamento corretto dei componenti. Qui si applicano i 
test-case applicati nei test d'integrazione precedenti e si applicano nuovi test. Se i componenti integrati sono pochi, 
si usa white box testing.

Nel \textit{top-down integration testing} si costruiscono prima i moduli primari, poi quelli secondari. I moduli secondari 
non ancora disponibili sono sostituiti da STUB, dei moduli "dummy" che forniscono servizi pre-impostati.

Nel \textit{bottom-up integration testing} si costruiscono prima i moduli secondari, poi quelli primari. I moduli primari 
non ancora disponibili sono sostituiti da DRIVER, dei test program che invocano i moduli secondo i dati selezionati per 
test-case.
\subsubsection*{Release testing}
Il \textit{release testing} riguarda il sistema completo. Riguarda la validazione ed è di tipo black-box.

Per i sistemi customizzati si ha il \textit{test di accettazione}, nel quale il committente controlla il soddisfacimento 
dei requisiti e si effettuano correzioni finchè il cliente non è soddisfatto.

Per i sistemi generici si hanno le fasi di \textit{alpha e beta testing}. La versione "alpha" del sistema viene resa 
disponibile ad un gruppo di sviluppatori per il collaudo finchè non si raggiunge ad un sistema soffisfacente; la versione 
"beta" del sistema viene resa disponibile ad un gruppo di clienti ed eventualmente corretta, dopodichè il prodotto viene 
messo sul mercato.
\subsubsection*{Stress testing}
Lo \textit{stress testing} serve per verificare l'efficienza e l'affidabilità del sistema. Viene svolto quando il sistema 
è completamente integrato. Vengono eseguiti test in cui il carico di lavoro è molto superiore a quello previsto normalmente.
\subsubsection*{Altri tipi di testing}
\begin{itemize}
    \item \textit{test di usabilità}: si valuta la facilità con cui gli utenti riescono ad usare il sistema 
    \item \textit{recovery testing}
    \item \textit{security testing}: simulazione di attacchi dall'esterno 
    \item \textit{deployment testing}: verifica dell'installazione sui dispositivi
    \item \textit{back-to-back testing}: si usa quando varie versioni del sistema sono disponibili, si effettua lo stesso 
    test su tutte le versiono e si confrontano gli output delle varie versioni
\end{itemize}
\section{Manutenzione}
La manutenzione riguarda tutte le modifiche fatte al sistema dopo la consegna. La manutenzione è un processo ciclico, che 
permette al sistema di "evolversi".
\subsection{Tipi di manutenzione e costo}
\subsubsection*{Manutenzione correttiva}
La \textit{manutenzione correttiva} corregge difetti non emersi in fase di collaudo. I difetti possono essere \textit{di 
implementazione}, i meno costosi da correggere, \textit{di progettazione} e \textit{di specifica}, i più costosi in quanto 
potrebbe essere necessario riprogettare il sistema.
\subsubsection*{Manutenzione adattiva}
Con \textit{manutenzione adattativa} s'intende l'adattamento del sistema a cambiamenti di piattaforma.
\subsubsection*{Manutenzione migliorativa}
Con \textit{manutenzione migliorativa} s'intende l'aggiunta, cambiamento o miglioramento di requisiti funzionali e non 
funzionali, secondo le richieste del committente.
\subsubsection*{Costi di manutenzione}
Molti sono i fattori che influenzano il costo di manutenzione: 
\begin{itemize}
    \item \textit{dipendenza dei componenti}: la modifica dell componenti potrebbe avere ripercussioni sugli altri componenti 
    \item \textit{linguaggio di programmazione}: i programmi scritti con linguaggi ad alto livello sono più facili da capire 
    e da mantenere
    \item \textit{struttura del codice}: codice ben strutturato e documentato rende più facile la manutenzione
    \item \textit{collaudo}: una fase di collaudo approfondita riduce il numero di difetti 
    \item \textit{qualità della documentazione}: una documentazione chiara e completa facilita la comprensione del sistema 
    da mantenere 
    \item \textit{stabilità dello staff}
    \item \textit{età del sistema}
    \item \textit{stabilità del dominio dell'applicazione}: se il dominio subisce variazioni, il sistema deve essere aggiornato
    \item \textit{stabilità della piattaforma}
\end{itemize}
\subsection{Processo di manutenzione}
Nel processo di manutenzione, bisogna innanzitutto \textit{identificare l'intervento} da svolgere e analizzarne l'impatto 
sul sistema, identificando quali componenti e tests saranno impattati dalla modifica. 
In seguito sono \textit{realizzate le modifiche}, aggiornando, se necessario, specifica, progettazione e/o implementazione.
Dopo il collaudo, la nuova versione viene rilasciata.
\subsubsection*{CRF}
Il \textit{Change Request Form} (CRF) è un documento formale che descrive una modifica.
È compilato dal proponente della modifica, gli sviluppatori e dalla Change Control Board. 
\subsection{Patch, versioni e release}
\subsubsection*{Patch}
Nel caso si presentino problemi che devono essere risolti in fretta, si richiede d'urgenza una \textit{manutenzione d'emergenza}.
Viene modificato direttamente il codice, applicando la soluzione più immediata, e si rilascai una versione aggiornata, 
detta \textit{patch}.
\subsubsection*{Versioni}
Una \textit{versione} è un'istanza del sistema che differisce per qualche aspetto dalle altre istanze.  Una versione è 
identificabile non solo da numeri di versione, ma anche dagli attributi
\subsubsection*{Release}
Una \textit{release} è una particolare verisone che viene distribuita a committenti/clienti. Una release comprende tutte 
le componenti software necessarie per far funzionare il sistema e la documentazione.
\subsubsection*{Configurazione software}
La \textit{configurazione software} è l'insieme di informazioni prodotto da un processo software. Include documentazione, 
codice e dati. Il \textit{database delle configurazioni} contiene varie configurazioni software corrispondenti alle release 
di un sistema.
\subsection{Sistemi ereditati e re-engineering}
Un \textit{sistema ereditato} è un vecchio sistema che deve essere mantenuto nel tempo. Sfruttano tecnologie obsolete e 
sono costosi e difficili da mantenere.
È possibile \textit{reimplementarli}, \textit{re-ingegnerizzarli} ed infine \textit{dismetterli}.

Il \textit{forward engineering} è il processo software nel quale si crea un nuovo progetto partendo dalla specifica dei 
requisiti.

Il \textit{re-engineering} è il processo mediante il quale un nuovo sistema nasce dalla trasformazione di uno vecchio, allo 
scopo di rinnovarlo e aumentare la mantenibilità. Le attività di re-engineering sono: traduzione del codice, ristrutturazione 
del codice, ristrutturazione dei dati.
\subsubsection*{Traduzione del codice}
Con \textit{traduzione del codice} s'intende il passaggio da un liguaggio di programmazione ad un altro, oppure ad una 
versione più recente di quel linguaggio.
\subsubsection*{Refactoring}
Le modifiche fatte al sistema nel corso del tempo tendono a rendere il codice sempre meno ordinato. La \textit{ristrutturazione}
cerca di rendere il codice più semplice, leggibile, comprensibile e modulare; vengono aggiunti commenti, eliminate le 
ridondanze, uniformato lo stile di programmazione. Non modifica il comportamento del codice, ma ne aumenta la leggibilità.
\subsubsection*{Ristrutturazione dei dati}
È il processo di riorganizzazione delle strutture dati.
\subsubsection*{Reverse engineering}
Per sistemi ereditati, la documentazione può essere scarsa o assente, complicando le attività di re-engineering. 
Il \textit{reverse engineering} è la generazione di documentazione partendo da codice e dati.
\subsubsection*{Reimplementazione del sistema}
Quando il sistema è troppo mal strutturato per il re-engineering, è inevitabile la reimplementazione del sistema. 
\section{Gestione del progetto}
Un progetto è l'insieme del processo software e dalle attività di gestione.
\subsection{Attività di pianificazione}
Il processo software è \textit{scomposto} in task, ai quali devono essere assegnati risorse e dei quali devono essere 
schedulati i tempi di esecuzione.
\subsubsection*{Milestone e deliverable}
Per valutare l'avanzamento del processo, sis stabiliscono dei momenti precisi in cui determinati risultati intermedi devono 
essere raggiunti.
Una \textit{milestone} è la terminazione di un certo insieme di task previsto per una certa data, e funge come punto di 
controllo per l'avanzamento del lavoro. Una \textit{deliverable} è una particolare milestone i cui risultati devono essere 
notificati al committente.
\subsubsection*{Tempistica}
Una dipendenza tra task indica che un task non può iniziare prima che un insieme di task sia stato ultimato.

Un \textit{cammino critico} è il cammino più lungo, in termini di tempo, dal "nodo iniziale" al "nodo finale". Ritardi 
di task nel cammino critico determinano ritardo nel completamento del progetto.
\subsection{Rischi}
Un \textit{rischio} è una circostanza avversa che si preferirebbe non accadesse. Il rischio ha due caratteristiche: l'
\textit{incertezza} che il rischio si verifichi, e la \textit{perdita} che il rischio potrebbe causare.

La \textit{gestione dei rischi} coinvolge l'identificazione dei rischi, la valutazione del loro impatto, la definizione 
di strategie preventive o reattive.
\subsubsection*{Tipi di rischio}
I rischi possono essere di vario tipo:
\begin{itemize}
    \item \textit{rischi tecnologici}: derivano dall'hardware e software utilizzati
    \item \textit{rischi riguardanti il personale}: derivano dal personale del team
    \item \textit{rischi organizzativi}: derivano dall'organizzazione aziendale
    \item \textit{rischi strumentali}: derivano dagli strumenti CASE
    \item \textit{rischi dei requisiti}: derivano dal cambiamento dei requisiti
    \item \textit{rischi di stima}: derivano dalle valutazioni relative a tempi, costi, ecc... 
\end{itemize}
I rischi di \textit{progetto} colpiscono la pianificazione, i rischi di \textit{prodotto} colpiscono la qualità del sistema 
e i rischi di \textit{business} colpiscono il committente o lo sviluppatore sul piano economico.
\subsubsection*{Analisi dei rischi}
Con \textit{analisi dei rischi} s'intende lo stimare la probabilità del rischio e la gravità dei suoi effetti.
\subsubsection*{Pianificazione dei rischi}
È necessario sviluppare strategie per limitare i rischi. Le \textit{strategie preventive} sono piani per ridurre la 
probabilità che il rischio si verifichi; le \textit{strategie reattive} sono piani per ridurre gli effetti nel caso il 
rischio si verifichi.
\subsubsection*{Monitoring dei rischi}
Si verifica, per ogni rischio identificato, se la sua probabilità può crescere o decrescere, e se la gravità dei suoi 
effetti può cambiare.
\subsection{Monitoraggio e revisione}
La gestione del progetto si basa sulle informazioni, inizialmente ridotte. Durante il progetto di raccolgono maggiori 
informazioni, e periodicamente di aggiornano pianificazione, stima dei costi e gestione dei rischi.

Il \textit{piano di progetto} è un documento che raccoglie le informazioni sulle attività di gestione di un progetto.
\section{Modelli di processo prescrittivi}
\subsection{Modello a cascata}
Il \textit{modello a cascata} è caratterizzato da fasi sequenziali: prima di passare alla fase successiva, la fase 
corrente deve essere stata completata. 
Il committente è coinvolto solo nella specifica, e può visionare il sistema solo alla consegna.

Il modello a cascata è caratterizzato da:
\begin{itemize}
    \item visibilità alta: ogni fase produce della documentazione e le fasi sono ben distinte
    \item affidabilità bassa: il sistema è collaudato in blocco e il committente può controllare il soddisfacimento dei 
    requisiti solo alla consegna 
    \item robustezza bassa: i requisiti non devono cambiare 
    \item rapidità bassa: il committente può disporre del sistema solo alla consegna, dopo le lunghe fasi di sviluppo
\end{itemize}
\subsubsection*{Prototipo}
Un \textit{prototipo} è una versione preliminare del sistema che realizza un sottoinsieme dei requisiti. Lo scopo del 
prototipo è comprendere un insieme di requisiti poco chiari. Non ha la qualità del prodotto finale, viene mostrato al 
committente e valutato. È utile se combinato con il modello a cascata.
\subsubsection*{Sviluppo basato sul riuso}
Lo sviluppo CBSE si basa sul riuso di componenti esistenti. È caratterizzato da:
\begin{itemize}
    \item visibilità alta
    \item affidabilità migliore: i componenenti pre-esistenti sono già stati collaudati in precedenza 
    \item robustezza migliore 
    \item rapidità migliore: alcuni componenti sono già pronti
\end{itemize}
\subsection{Modelli iterativi}
I \textit{modelli iterativi} prevedono la ripetizione di alcune fasi del processo e la partecipazione del committente
durante il processo per verificare il soddisfacimento dei requisiti. Il sistema è un \textit{prototipo evolutivo}: il 
prototipo evolve nel sistema finale, aggiungendo gradualmente i requisiti mancanti. 
\subsubsection*{Sviluppo evolutivo}
Lo sviluppo è diviso  in sei fasi:
\begin{itemize}
    \item si definiscono ad alto livello i requisiti
    \item si seleziona un sottoinsieme dei requisiti
    \item si aggiorna il prototipo corrente facendo la specifica, la progettazione, l'implementazione e il collaudo dei 
    nuovi requisiti
    \item si genera una release del sistema che realizza i requisiti considerati finora, si mostra il prototipo al committente 
    che fornisce un feedback
    \item se necessario, si corregge il prototipo in base al feedback 
    \item se ci sono altri requisiti da trattare, si ritorna al passo 2
\end{itemize}
Si comincia dai requisiti funzionali con maggiore priorità.

Attributi dello sviluppo evolutivo:
\begin{itemize}
    \item visibilità bassa: la documentazione non sempre viene aggiornata
    \item affidabilità alta: il collauddo riguarda un sottoinsieme di requisiti e il committente controlla il soddisfacimento 
    dei requisiti ad ogni versione 
    \item robustezza media: il processo supporta il cambiamento dei requisiti, ma il codice rischia di diventare poco 
    strutturato
    \item rapidità media: un prototipo funzionante è sempre disponibile, ma la consegna della versione richiede comunque 
    l'attesa dell'analisi, la progettazione, l'implementazione e il collaudo dei relativi requisiti
\end{itemize}
\subsubsection*{Sviluppo incrementale}
Mantiene i vantaggi dello sviluppo evolutivo ma ha maggiore visibilità.

Durante le \textit{fasi iniziali} si ha la specifica ad alto livello, si stabilisce quale sarà il numero di incrementi del 
sistema durante lo sviluppo, il loro ordine e viene progettato il sistema.
Durante le \textit{fasi cicliche} si specificano i requisiti dell'incremento, lo si implementa, lo si collauda e si genera 
una release corrente del sistema.

Attributi dello sviluppo incrementale:
\begin{itemize}
    \item visibilità alta: all'inizio del progetto sono specificati architettura, numero e ordine degli incrementi
    \item affidabilità alta: il collaudo riguarda un sottoinsieme di requisiti e il committente controlla il soddisfacimento 
    dei requisiti ad ogni versione 
    \item robustezza media: il processo supporta il cambiamento dei requisiti, e conoscendo l'architettura e il numero 
    degli incrementi, il codice può essere strutturato più ordinatamente
\end{itemize}
\subsection{Processo unificato}
Il \textit{processo unificato} (UP) comprende 4 fasi:
\begin{itemize}
    \item \textit{avvio}: l'inizio del progetto, con studio di fattibilità, requisiti preliminari, ecc... 
    \item \textit{elaborazione}: genera un primo prototipo per il sistema 
    \item \textit{costruzione}: si passa dal prototipo iniziale a quello finale 
    \item \textit{transizione}: prepara tutto ciò che è necessario per la consegna del sistema presso il committente 
\end{itemize}
Ogni fase richiede l'esecuzione dei seguenti \textit{workflow}: requisiti, analisi, progettazone, implementazione, test.
La quantità di lavoro dedicato ad ogni workflow varia a seconda della fase, ed ogni fase può essere eseguita più volte.
Alla fine di ogni fase si otttengono una milestone ed un prototipo eseguibile.

UP è contemporaneamente \textit{sequenziale}, \textit{iterativo} e \textit{incrementale}. UP si avvale di diagrammi UML 
per modellare le fasi di lavoro.

Attributi di UP:
\begin{itemize}
    \item visibilità media: le 4 fasi sono ben distinte e producono ducumentazione, ma non si sa a priori quante volte 
    verranno ripetute
    \item affidabilità alta: il collaudo riguarda un sottoinsieme di requisiti e il committente controlla il soddisfacimento 
    dei requisiti ad ogni versione 
    \item robustezza media: il processo supporta il cambiamento dei requisiti, e conoscendo l'architettura e il numero 
    degli incrementi, il codice può essere strutturato più ordinatamente
    \item rapidità media: prevede prototipo evolutivo, feedback del committente e gestione dei rischi
\end{itemize}
\section{Modelli di processo agili}
L'approccio \textit{agile} è pronto al cambiamento e flessibile. Lo scopo è la consegna rapida di un prodotto e riguarda
progetti limitati.
\subsection{eXtreme Programming}
L'\textit{eXtreme Programming} (XP) è una forma estrema di sviluppo evolutivo, dove la rapidità è l'obiettivo fondamentale.
Ci si concentra sulla fase di implementazione, mentre le altre fasi si seguono solo ad alto livello per guadagnare tempo.
\subsubsection*{Specifica}
La specifica è realizzata ad alto livello: gli scenari sono ordinati per priorità ed ogni scenario è diviso in incrementi.
Insieme al committentte vengono stabiliti quali incrementi saranno realizzati dalla prossima realease, i tempi di consegna 
e i test da eseguire nella fase di collaudo.
\subsubsection*{Progettazione}
Si definisce o si aggiorna l'architettura ad alto livello, i dettagli vengono stabiliti durante l'implementazione.
\subsubsection*{Implementazione}
Sono implementate varie strategie durante la fase di implementazione.

La programmazione avviene \textit{a coppie}:
\begin{itemize}
    \item aumenta la probabilità di notare errori nel codice 
    \item il driver scrive il codice, il navigatore lo controlla e fa il refactoring: il codice diventa più chiaro e 
    mantenibile
    \item la conoscenza del codice viene condivisa 
\end{itemize}

\textit{Test Driven Development}: test case sono definiti durante la fase di pianificazione, prima dell'implementazione.

Inoltre, ogni programmatore può modificare qualunque parte del codice (\textit{possesso collettivo}).
\subsubsection*{Collaudo}
Vengono eseguiti tre tipi di testing ad ogni incremento:
\begin{itemize}
    \item testing dell'incremento (unit test): svolto sull'incremento
    \item testing d'integrazione: i test precedenti sono ripetuti
    \item test di accettazione: il committente controlla il soddisfacimento dei requisiti
\end{itemize}
Vengono usati gli strumenti CASE per ridurre il tempo di collaudo.
\subsubsection*{Attributi}
Il committente è attivamente coinvolto nel processo, in quanto assegna priorità ai requisiti, il contenuto e la scadenza 
delle release, definisce e controlla i test-case.

XP è caratterizzato da:
\begin{itemize}
    \item visibilità bassa: la documentazione è composta da test-case, scenari e codice; i requisiti vengono spesso 
    modificati
    \item rapidità alta: le release sono molto frequenti e vengono usati strumenti CASE durante il collaudo 
    \item affidabilità alta: lo sviluppo è guidato dai test, il committente può verificare la qualità del prodotto 
    costantemente
    \item robustezza alta: il processo supporta il cambiamento dei requisiti e il codice rimane ben strutturato e definito
\end{itemize}
\subsubsection*{Refactoring}
Il codice è soggetto a frequenti modifiche. Il refactoring non modifica il comportamento del codice, ma ne migliora la 
struttura interna.
Durante il refactoring:
\begin{itemize}
    \item viene eliminato il codice duplicato
    \item viene ridotta la lunghezza dei metodi/funzioni 
    \item vengono dati nomi significativi alle variabili ed alle classi 
    \item viene uniformata l'indentazione 
\end{itemize}
\subsection{Scrum}
\textit{Scrum} è un altro modello di processo agile. Le fasi del processo prevedono la definizione del \textit{product
backlog}, l'insieme di requisiti, e da 3 a 8 \textit{sprint}. Ogni sprint contiene una micro-cascata (progettazione, 
implementazione e collaudo), genera un prototipo evolutivo da mostrare al committente e dura un mese.

Il \textit{product backlog} è l'elenco di tutti i requisiti e relative priorità. È definito all'inizio del processo dal 
product owner, ed è aggiornato dopo ogni sprint.

Lo \textit{sprint backlog} è il sottoinsieme del product backlog da sviluppare durante uno specifico sprint.
\subsubsection*{Meeting}
Lo \textit{sprint planing meeting} è la riunione all'inizio di ogni sprint. Viene aggiornato il product backlog (se 
necessario), viene definito lo sprint backlog e lo sprint goal (obiettivo dell sprint corrente).

Lo \textit{scrum meeting} è la riunione giornaliera. Ogni sviluppatore descrive ciò che ha fatto il giorno prima e cosa 
farà nel giorno attuale. Si controlla l'avanzamento del lavoro, si aggiorna lo sprint backlog e si cercano soluzioni ai 
problemi riscontrati.

Lo \textit{sprint review meeting} è la riunione alla fine di ogni sprint. Si mostra il prototipo corrente e il product 
owner decide se lo sprint goal è stato raggiunto o meno.
\subsubsection*{Attributi}
\begin{itemize}
    \item visibilità media: l'avanzamento del lavoro è controllato giornalmente, ma non è possibile sapere a priori il 
    numero degli sprint  
    \item affidabilità alta: il committente è costantemente coinvolto e il collaudo è eseguito su ogni prototipo
    \item robustezza alta: gli impedimenti sono notificati e risolti giornalmente, il backlog può essere aggiornato ad 
    ogni sprint
    \item rapidità alta: un prototipo viene rilasciato ogni mese
\end{itemize}
\subsection{DevOps}
Il \textit{DevOps} è caratterizzato da collaborazione tra team di sviluppo (Dev) e amministratori di sistemi (Ops). 
DevOps è scaturito dalla necessità di mantenere "systems of engagement", ossia applicazioni mobili o social media, che 
richiedono prestazioni elevate e tempi rapidi di sviluppo e manutenzione.

Lo scopo di DevOps è quello di creare un "ponte" tra sviluppatori, amministratori e utenti, in modo da rilasciare aggiornamenti 
continui e ricevere feedback frequenti.
\subsubsection*{Processi interni}
\begin{itemize}
    \item Dev: sviluppare e testare come in un ambiente di produzione
    \item Release: distribuire con processi automatici 
    \item Ops: monitorare la qualità operativa 
    \item Ritorno di feedback da parte degli utenti 
\end{itemize}

Nel processo \textit{CI} (Continuous Integration and testing), gli sviluppatori integrano il codice diverse volte al giorno
tramite repo condiviso. Il codice viene assemblato da uno strumento di build automatico per rilevare eventuali problemi 
e si eseguono test funzionali e non in modo automatico.

Nel processo \textit{CD} (Continuous Delivery and deployment), l'obiettivo è quello di costruire il software in modo da 
rilasciarlo in qualsiasi momento. Ogni singola modifica che passa con successo i test viene automaticamente distribuita.

Nel processo \textit{CO} (Continuous Operations), il sistema viene monitorato costantemente.

Nel processo \textit{CA} (Continuous Assessment), si ascolta il feedback degli utenti e si pianificano di conseguenza le 
azioni da compiere in futuro (anche a livello finanziario).
\subsubsection*{Attributi}
\begin{itemize}
    \item visibilità media: non si conosce la frequenza di release 
    \item affidabilità alta: integrazione e collaudo automatico dopo ogni release 
    \item robustezza alta: feedback frequenti e monitoring costante 
    \item rapidità alta: software rilasciato in modo continuo e automatico
\end{itemize}
\end{document}

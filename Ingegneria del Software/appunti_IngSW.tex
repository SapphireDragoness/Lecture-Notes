\documentclass[11pt]{article}
\usepackage[margin=.8in]{geometry}
\usepackage[italian]{babel}
\usepackage{graphicx}

\title{Appunti Ingegneria del Software}

\begin{document}
\section{Introduzione}
\subsection{Sistemi software}
Un \textit{sistema software} è un insieme di componenti software che funzionano in modo coordinato allo scopo di informatizzare 
una certa attività. La realizzazione di un sistema software richiede l'impiego di un gruppo di lavoro, nel quale ogni 
persona ricopre un ruolo ben preciso e le attività dei vari gruppi vanno coordinate, e tempo da dedicare alle varie fasi 
di sviluppo.

Esistono due categorie di sistemi software: i \textit{sistemi generici}, definiti in base alle tendenze di mercato, e i 
\textit{sistemi customizzati}, richiesti da uno specifico cliente (il committente).
\subsection{Il processo software}
Con \textit{ingegneria del software} si intende l'applicazione del processo dell'Ingegneria alla produzione di sistemi 
software. Il processo è suddiviso in:
\begin{itemize}
    \item specifica: definizione dei requisiti funzionali e non funzionali 
    \item progettazione: si definiscono architettura, controllo, comportamento dei componenti, strutture dati, algoritmi,
    struttura del codice, interfaccia utente
    \item implementazione: scrittura del codice e integrazione dei moduli 
    \item collaudo: si controlla se il sistema ha difetti di funzionamento e se soddisfa i requisiti 
    \item manutenzione: modifiche del sistema dopo la consegna 
\end{itemize}
\subsection{Gestione del processo}
L'ingegneria del software si occupa anche della gestione del progetto che si svolge in parallelo al processo software.
Le principali attività di gestione sono l'\textit{assegnazione} di risorse (umane, finanziarie...), la \textit{stima del tempo}
necessario per ogni attività, la \textit{stima dei costi} e la \textit{stima dei rischi}.
\section{Specifica}
La \textit{specifica} è l'insieme di attività necessarie per generare il documento dei requisiti che descrive i 
\textit{requisiti funzionali} e i \textit{requisiti non funzionali}: descrive il "cosa" il sistema deve fare, non il "come".
I requisiti servono per una proposta di contratto e modellare fasi successive del processo software.
\subsection{Requisiti funzionali}
I requisiti funzionali sono i servizi che il cliente richiede al sistema. Per ogni servizio si descrive:
\begin{itemize}
    \item cosa accade nell'interazione tra utente e sistema 
    \item cosa accade in seguito ad un certo input o stimolo
    \item cosa accade in particolari sistuazioni, ad esempio in caso di eccezioni
\end{itemize}
Non viene descritto come funziona internamente il sistema, in quanto è oggetto della successiva fase di progettazione.
\subsection{Requisiti non funzionali}
I requisiti non funzionali sono divisi in tre categorie: \textit{requisiti di prodotto}, \textit{requisiti organizzativi}
e \textit{requisiti esterni}.

I requisiti di prodotto sono attributi che definiscono la qualità del sistema. Una \textit{proprietà complessiva} riguarda
il sistema nel suo complesso; una \textit{proprietà emergente} è una proprietà che "emerge" dal funzionamento del sistema, 
dopo che è stato implementato.

I requisiti organizzativi sono caratteristiche riguardanti le fasi del processo software o la gestione del progetto. I 
\textit{requisiti di sviluppo} sono i metodi e le tecniche di sviluppo utilizzati; i \textit{requisiti gestionali} sono
le risorse utilizzate.

I requisiti esterni derivano da fattori esterni al sistema e al processo software. Essi sono i requisiti di compatibilità
con altri sistemi e aspetti giuridici.
\begin{center}
    \includegraphics[scale=0.4]{reqnfunz.jpeg}
\end{center}
\subsubsection*{Usabilità}
L'\textit{usabilità} è il grafo di facilità con cui l'utente riesce a comprendere l'uso del software. Il sistema deve avere 
un'interfaccia utente intuitiva ed curata, in quanto è fattore critico per il successo di un prodotto.
L'uso del sistema deve essere ben documentato, per permettere all'utente di apprendere velocemente l'uso del prodotto.
Il \textit{training} degli utenti può migliorare l'usabilità del prodotto.
\subsubsection*{Mantenibilità}
la \textit{mantenibilità} è il grado di facilità di manutenzione. Le cause della manutenzione sono molteplici, e deve 
essere possibile l'evoluzione del software per soddisfare i requisiti nel tempo.
\subsubsection*{Portabilità}
La \textit{portabilità} è la capacità di migrazione da un ambiente ad un altro.
\subsubsection*{Recoverability}
La \textit{recoverability} è la capacità di ripristinare lo stato e i dati del sistema dopo che si è verificato un fallimento.
\subsubsection*{Efficienza}
L'\textit{efficienza} è il livello di prestazioni del sistema, e può essere misurato in vari modi: tempo di risposta, 
numero medio di richieste... 
\subsubsection*{Affidabilità}
L'\textit{affidabilità} è il grado di fiducia con cui si ritiene che il sistema svolga in modo corretto la propria funzione.
Ci sono varie misure di affidabilità:
\begin{itemize}
    \item \textit{reliability}: capacità di fornire i servizi in modo continuativo per una certa durata di tempo 
    \item \textit{availability}: capacità di fornire i servizi nel momento richiesto
    \item \textit{safety}: capacità di operare senza causare danni materiali
    \item \textit{security}: capacità di proteggersi da intrusioni e attacchi
\end{itemize}
Un sistema è definito \textit{critico} quando il suo non corretto funzionamento può provocare conseguenze "disastrose"
a persone e ambiente (\textit{safety critical system}) o perdite economiche (\textit{business critical system}).
Il costo cresce in modo esponenziale rispetto al grado di affidabilità richiesto.
\subsection{Processo di specifica}
Il \textit{processo di specifica} è il processo per generare il documento dei requisiti, ed è diviso in più fasi.
Lo stesso requisito viene definito con due gradi di dettaglio diversi. Il \textit{requisito utente} è descritto ad alto 
livello, in linguaggio naturale, ed è il risultato della deduzione dei requisiti.
Il \textit{requisito di sistema} è descritto dettagliatamente, fornendo tutti i dettagli necessari per la fase di progettazione,
ed è il risultato dell'analisi dei requisiti.
\subsubsection*{Studio di fattibilità}
Lo \textit{studio di fattibilità} è la valutazione della possibilità di sviluppare il sistema e dei suoi vantaggi per 
il committente. Si decide se la costruzione del sistema è fattibile date le risorse disponibili e se il sistema è 
effettivamente utile al cliente.
Per svolgere lo studio si raccolgono informazioni e si prepara un rapporto di fattibilità, che contiene la valutazione 
della possibilità di costruire un sistema e dei vantaggi che possono derivare dalla sua introduzione.
\subsubsection*{Deduzione dei requisiti}
La \textit{deduzione dei requisiti} è la raccolta di informazioni da cui dedurre quali sono i requisiti.
Le informazioni si possono raccogliere mediante uno studio del dominio applicativo del sistema richiesto, il dialogo con 
stakeholder, 
\subsubsection*{Analisi dei requisiti}
La \textit{analisi dei requisiti} è l'organizzazione, negoziazione e modellazione dei requisiti.

\subsubsection*{Validazione dei requisiti}
La \textit{validazione dei requisiti} è la verifica del rispetto di alcune proprietà da parte del documento dei requisiti.
\end{document}
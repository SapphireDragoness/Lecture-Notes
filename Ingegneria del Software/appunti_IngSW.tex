\documentclass[11pt]{article}
\usepackage[margin=.8in]{geometry}
\usepackage[italian]{babel}
\usepackage{graphicx}

\title{Appunti Ingegneria del Software}

\begin{document}
\section{Introduzione}
\subsection{Sistemi software}
Un \textit{sistema software} è un insieme di componenti software che funzionano in modo coordinato allo scopo di informatizzare 
una certa attività. La realizzazione di un sistema software richiede l'impiego di un gruppo di lavoro, nel quale ogni 
persona ricopre un ruolo ben preciso e le attività dei vari gruppi vanno coordinate, e tempo da dedicare alle varie fasi 
di sviluppo.

Esistono due categorie di sistemi software: i \textit{sistemi generici}, definiti in base alle tendenze di mercato, e i 
\textit{sistemi customizzati}, richiesti da uno specifico cliente (il committente).
\subsection{Il processo software}
Con \textit{ingegneria del software} si intende l'applicazione del processo dell'Ingegneria alla produzione di sistemi 
software. Il processo è suddiviso in:
\begin{itemize}
    \item specifica: definizione dei requisiti funzionali e non funzionali 
    \item progettazione: si definiscono architettura, controllo, comportamento dei componenti, strutture dati, algoritmi,
    struttura del codice, interfaccia utente
    \item implementazione: scrittura del codice e integrazione dei moduli 
    \item collaudo: si controlla se il sistema ha difetti di funzionamento e se soddisfa i requisiti 
    \item manutenzione: modifiche del sistema dopo la consegna 
\end{itemize}
\subsection{Gestione del processo}
L'ingegneria del software si occupa anche della gestione del progetto che si svolge in parallelo al processo software.
Le principali attività di gestione sono l'\textit{assegnazione} di risorse (umane, finanziarie...), la \textit{stima del tempo}
necessario per ogni attività, la \textit{stima dei costi} e la \textit{stima dei rischi}.
\section{Specifica}
La \textit{specifica} è l'insieme di attività necessarie per generare il documento dei requisiti che descrive i 
\textit{requisiti funzionali} e i \textit{requisiti non funzionali}: descrive il "cosa" il sistema deve fare, non il "come".
I requisiti servono per una proposta di contratto e modellare fasi successive del processo software.
\subsection{Requisiti funzionali}
I requisiti funzionali sono i servizi che il cliente richiede al sistema. Per ogni servizio si descrive:
\begin{itemize}
    \item cosa accade nell'interazione tra utente e sistema 
    \item cosa accade in seguito ad un certo input o stimolo
    \item cosa accade in particolari sistuazioni, ad esempio in caso di eccezioni
\end{itemize}
Non viene descritto come funziona internamente il sistema, in quanto è oggetto della successiva fase di progettazione.
\subsection{Requisiti non funzionali}
I requisiti non funzionali sono divisi in tre categorie: \textit{requisiti di prodotto}, \textit{requisiti organizzativi}
e \textit{requisiti esterni}.

I requisiti di prodotto sono attributi che definiscono la qualità del sistema. Una \textit{proprietà complessiva} riguarda
il sistema nel suo complesso; una \textit{proprietà emergente} è una proprietà che "emerge" dal funzionamento del sistema, 
dopo che è stato implementato.

I requisiti organizzativi sono caratteristiche riguardanti le fasi del processo software o la gestione del progetto. I 
\textit{requisiti di sviluppo} sono i metodi e le tecniche di sviluppo utilizzati; i \textit{requisiti gestionali} sono
le risorse utilizzate.

I requisiti esterni derivano da fattori esterni al sistema e al processo software. Essi sono i requisiti di compatibilità
con altri sistemi e aspetti giuridici.
\begin{center}
    \includegraphics[scale=0.4]{reqnfunz.jpeg}
\end{center}

\end{document}
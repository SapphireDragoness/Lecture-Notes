\documentclass[12pt]{article}
\usepackage[margin=0.8in]{geometry}
\usepackage{graphicx}

\title{File System}

\begin{document}
\begin{center}
    \huge\textbf{File System}
\end{center}
Per memorizzare dati a lungo termine, il sistema operativo fornisce l'astrazione del \textbf{file} attraverso un 
\textbf{file system}. I file sono l'unità logica di informazione creata da un processo e le informazioni contenute
in un file sono persistenti.\\
Dal punto di vista utente, file system è il servizio fornito dal sistema operativo che permette di facilitare le 
operazioni sui file. Dal punto di vista del designer di sistemi, è una collezione di strutture dati e meccanismi
usati dal sistema operativo per memorizzare e cercare informazioni. 
\section{File}
\subsection{Denominazione}
Ogni sistema operativo ha le proprie regole per la denominazione dei file: di solito sono ammesse stringhe di 1-8 
caratteri alfanumerici e alcuni caratteri speciali. Alcuni sistemi operativi sono case sensitive, altri no.\\
Molti SO supportano nomi a due parti, con le due parti separate da un punto (parte1.parte2): la seconda parte è
detta \textbf{estensione del file}, indica una caratteristica del file. In UNIX, sono solo una convenzione, mentre 
su Windows hanno un significato: le estensioni indicano con quale programma il file può essere aperto.
\subsection{Struttura}
Tre modi comunemente usati per strutturare file:
\begin{itemize}
    \item sequenza di byte: un file è una sequenza di byte non strutturata, il significato è imposto dai programmi a
    livello utente
    \item sequenza di record: un file è una sequenza di record a lunghezza fissa
    \item albero di record: un file è un albero di record, ognuno dei quali contiene un campo chiave
\end{itemize}
\subsection{Tipi}
La maggior parte dei SO supporta almeno due tipi di file: \textbf{file regolari} e \textbf{directory}.\\
I file regolari contengono dati e possono essere file di testo o binari, le directory sono file di sistema che
mantengono la struttura del file system. Altri tipi di file sono di file a caratteri e file a blocchi, che servono 
per I/O.\\
I file sono divisi in varie sezioni: la prima è l'header, contenente il \textbf{numero magico} che identifica il
formato.
\subsection{Accesso}
Tipi di accesso:
\begin{itemize}
    \item accesso sequenziale: i byte sono letti/scritti in ordine a partire dall'inizio
    \item accesso diretto: i byte possono essere letti/scritti in qualsiasi ordine; il SO fornisce una funzione di
    seek per accedere al file in qualsiasi posizione
\end{itemize}
\subsection{Attributi}
Gli attributi sono le proprietà associate ad ogni file: data di creazione, dimensioni e permessi sono attributi.
\subsection{Operazioni}
Operazioni possibili su file:
\begin{itemize}
    \item create: crea un file vuoto
    \item delete: elimina un file
    \item open: apre un file
    \item close: chiude un file 
    \item read: legge dati dal file dalla posizione corrente
    \item write: scrive dati su file nella posizione corrente
    \item seek: cambia la posizione del puntatore 
    \item append: aggiunge dati alla fine di un file
    \item get attributes: legge proprietà di un file
    \item set attributes: modifica proprietà di un file
    \item rename: rinomina file
\end{itemize}
\section{Directory}
\subsection{Organizzazione}
L'organizzazione può essere a directory singola o a gerarchia: la prima ha una directory contenente tutti i file, la
seconda più directory ordinate ad albero.\\
\subsection{Path name}
Il path name è il modo per specificare il nome di un file all'interno di un albero di directory: è una sequenza di 
nomi di directory separate da un carattere detto \textbf{path separator}.\\
La directory corrente è rappresentata con \textbf{"."}, la directory genitore con \textbf{".."}.\\
Diversi tipi di path:
\begin{itemize}
    \item assoluto: path dalla root directory al file, inizia con separatore e può contenere . e ..
    \item canonico: path diretto, non può contenere . o ..
    \item relativo: path relativo alla directory corrente, può contenere . e ..
\end{itemize}
\subsection{Operazioni}
Operazioni possibili su directory:
\begin{itemize}
    \item create: crea directory vuota
    \item delete: elimina una directory vuota
    \item opendir: apre directory
    \item closedir: chiude directory
    \item readdir: legge directory
    \item rename: rinomina directory
    \item link: crea link ad un file
    \item unlink: rimuove link al file
\end{itemize}
Il linking è una tecnica che permette ad un file di apparire in più di una directory. Possono essere hard link o
soft link (detto anche symbolic link).
\section{Implementazione del file system}
\subsection{Layout}
Una disco può essere diviso in più partizioni. Una partizione è divisa in più settori.\\
Il settore 0 è detto \textbf{Master Boot Record}(MBR) e contiene il boot loader del SO. Alla fine del MBR c'è la 
tavola di partizione.\\
Il \textbf{superblock} è un blocco che contiene i parametri chiave del file system ed è portato in memoria a boot 
time.\\
\textbf{Free space mgmt} contiene info sui blocchi liberi del file system.\\
\textbf{I-nodes} contiene info su ogni file.\\
\textbf{Root dir} è la root directory.\\
\includegraphics[width=\textwidth]{mbr.png}
Molte strutture dati sono caricate e create nella memoria centrale quando il file system viene usato (montato), sono
distrutte quando viene smontato. Queste strutture dati dipendono del SO. Tipicamente sono: 
\begin{itemize}
    \item mount table: contiene info su tutti i file system montati
    \item open file table del sistema: contiene copie del file control block di ogni file aperto
    \item open file table dei processi: contiene, per ogni file aperto da un processo, puntatori per ogni voce
    della file table del sistema
    \item cache della struttura delle directory
    \item cache dei blocchi del disco
\end{itemize}
Alla creazione di un file, il SO crea un nuovo file control block e aggiunge una nuova voce nella directory.\\
All'apertura di un file, il SO cerca nelle directory il file appropriato e controlla se esiste una voce nella 
system-wide open file table: se esiste, crea una nuova voce nella per-process file table e incrementa il counter
dei file aperti, altrimenti crea una nuova voce nella system-wide open file table e pone il counter a 1.
L'operazione di open ritorna un pointer all'voce appropriata nel per-process file table.\\
Alla chiusura di un file, l'voce nella per-process file table è rimossa e il counter decrementato: se il counter 
nella system-wide file table diventa 0, anche quella voce è rimossa.
\subsection{Implementazione dei file}
Un file è un insieme di blocchi. Esistono vari modi per allocare i file.\\
Nel metodo ad \textbf{allocazione contigua}, il file è memorizzato come una sequenza contigua di blocchi: l'voce nella
directory contiene info sul numero di blocchi da cui è composto il file e indirizzo del primo blocco.
Questo metodo è facile da implementare e garantisce maggiori performance, ma causa frammentazione esterna: i buchi
lasciati da file eliminati potrebbero essere troppo piccoli per poter memorizzare nuovi file.
Inoltre è necessario sapere, alla creazione del file, quanti byte questo occuperà.\\
Il metodo di \textbf{allocazione a extents} è simile a quello contiguo: ogni file consiste in una o più regioni di 
blocchi contigui detti extents. Ogni file necessita di una tavola per tener traccia degli extents allocati. Con extents
di lunghezza fissa si va incontro a frammentazione interna, con quelli a lunghezza variabile a frammentazione esterna.\\
Nell'\textbf{allocazione a liste concatenate}, i blocchi di un file sono organizzati in una lista concatenata, l'voce 
nella directory contiene solo l'indirizzo del primo blocco.
Mentre non causa frammentazione esterna, questo metodo non garantisce buone performance e presenta overhead, poichè ogni
blocco deve riservare spazio per il puntatore al prossimo blocco (qualche byte).\\
L'\textbf{allocazione a cluster} accorpa blocchi contigui in cluster di dimensioni fisse e li collega. Performa meglio
dell'allocazione a liste concatenate ma va sempre incontro a frammentazione interna(un cluster può essere pieno solo in
parte).\\
L'allocazione a liste collegate può essere estesa usando una \textbf{File Allocation Table} (FAT): i puntatori ai
blocchi successivi sono memorizzati nella tabella, con una voce per ogni blocco. Per velocizzare l'operazione di seek, 
la tabella è portata in RAM; rimane ancora possibile la perdita di dati legata al danneggiamento dei pointer.\\
Con l'\textbf{allocazione a indici} ogni file ha la sua tavola d'allocazione (chiamata index block), memorizzata in uno
o più blocchi separati, contenenti gli indirizzi dei blocchi. L'voce del file nella directory contiene l'indirizzo 
dell'index block. L'index block è portato in RAM, ma solo quando il file associato è aperto. La dimensione del file è 
limitata dalla dimensione dell'index block.
Ci sono varie estensioni dell'allocazione a indici: lo \textbf{schema a liste collegate} ha index block organizzati come
una lista collegata e lo \textbf{schema a indici multilivello} ha index block che puntano ad altri index block e così 
via, fino a puntare ai file block. Il numero di livelli determina la dimensione massima del file.
L'efficienza dipende dalle dimensioni del file e dal blocco che si vuole accedere.
Nello \textbf{schema i-node} eiste un index block principale, l'i-node: le prime n voce sono puntatori diretti ai
blocchi, gli altri sono puntatori indiretti, che puntano ad indici multilivello.
\subsection{Implementazione delle directory}
Una directory è una collezione di voce, una per ogni file contenuto nella directory. In alcuni SO, l'voce contiene
anche attributi dei file, in altri sono memorizzati nel file control block.\\
Come gestire file con nomi lunghi? Usare campi di lunghezza fissa consuma molto spazio, quindi si usano campi di
lunghezza variabile: ogni file ha un header di lunghezza fissa e un cammpo di lunghezza variabile per il nome del file.
Ogni nome è terminato da un carattere speciale. Questo approccio causa frammentazione esterna; si possono invece tenere
i nomi dei file in un area separata, ad esempio l'heap alla fine di una directory: ogni voce contiene un puntatore
nell'heap dove può trovare il nome del file. Ciò causa frammentazione interna ma è risolvibile con la compattazione.\\
L'operazione fondamentale è cercare file: può essere reso efficiente tramite una hashtable. Ogni directory ha una
hashtable che mappa il nome del file al puntatore dell'voce della directory dove può essere trovato.
In alternativa, si può usare una struttura ad albero: ogni nodo contiene un numero variabile di chiavi e figli, e le 
foglie hanno tutte la stessa profondità. Di solito viene usato uno schema ibrido: se la directory contiene pochi file,
viene usata una lista, altrimenti viene usato un albero.\\
I nomi dei file sono memorizzati in una cache: questa è ispezionata prima di ogni ricerca, quindi se contiene l'
indirizzo del file, questo è locato immediatamente. Se il nome del file non è contenuto nella cache, si risale alla
directory genitore finchè non viene trovata una voce nella cache.
\subsection{Condivisione file}
File condivisi apppaiono in directory diverse appartenenti ad utenti diversi. Per ottenere ciò si usa la tecnica del
\textbf{linking}: le directory coinvolte contengono un link al file condiviso. Il linking è gestito tramite un 
\textbf{grafo diretto aciclico}.\\
Se le voce contenessero indirizzi a disco, i cambiamenti del file apparirebbero solo all'utente che lo ha modificato.
La prima soluzione è quella di usare \textbf{hard linking}: i blocchi sono listati nel file control block del file 
condiviso (in UNIX, gli hard link ad un file puntano al suo i-node). Se però un utente rimuovesse il file, gli altri
utenti avrebbero una voce che punta ad un file control block non valido, e la rimozione dei link non validi sarebbe
un'operazione non pratica. Di solito, il file control block contiene un counter che tiene traccia di quante directory
voce puntano al file: quando il counter raggiunge lo 0, il SO può rimuovere il file control block.\\
La seconda soluzione è quella di usare \textbf{symbolic linking}: il SO crea un nuovo file di tipo LINK, che contiene 
solo l'indirizzo al file a cui è collegato. A differenza degli hard-link, i symbolic-link non causano incremento o 
decremento del counter e sono trasparenti alle applicazioni. 
Solo l'utente che ha creato il file ha il puntatore al file control block del file condiviso, quindi il file control
block è distrutto solo quando il proprietario elimina il file.\\
I symbolic link possono linkare a directory (ciò può causare loop) e sono file speciali (gli hard link sono file
regolari); i file simbolici possono linkare file attraverso macchine diverse, su diversi file system, ma richiedono
più overhead rispetto agli hard link, dato che è necessario un ulteriore file control block per ogni symbolic link.\\
Alcuni programmi non riconoscono il link, mentre altri forniscono solo supporto limitato.
\subsection{Mounting/unmounting}
Prima di poter accedere ai file su un file system, questo deve essere montato. Montare un file system collega questo ad
un \textbf{mount point} e lo rende disponibile al sistema. Di solito il mount point è una directory vuota, ma alcuni 
sistemi operativi permettono il mounting su directory contenenti file.
Su UNIX, il \textbf{root file system} "/" è sempre montato a boot time, e ogni altro file system può essere collegato o
scollegato dal root file system.\\
Per completare un'operazione di mounting, il SO deve verificare che il dispositivo contenga un file system valido. Se lo
è, il SO memorizza nella \textbf{mount table} che il file system è stato montato. Ad esempio, su UNIX, il SO:
\begin{itemize}
    \item controlla la validità del file system
    \item crea strutture dati che contengono metadati del nuovo file system (superblock)
    \item aggiorna la mount table con la nuova voce
    \item collega il nuovo file system al mount point settando flag del mount point e puntatore alla voce nella mount
    table
\end{itemize}
\subsection{Virtual File Systems}
Un \textbf{virtual file system} integra multipli file system in una singola struttura omogenea: possono essere file 
system memorizzati nelle partizioni del drive locale oppure file system remoti. L'idea chiave è l'astrazione, ovvero 
isolare gli aspetti che sono comuni a tutti i file system in uno strato che fa da ponte tra gli strati superiori e
quelli inferiori.\\
Il VFS possiede un'interfaccia "superiore" che si interfaccia con i processi utente ed una inferiore che si interfaccia 
con i file system concreti (in kernel space). Le astrazioni tipicamente usate dalla VFS sono:
\begin{itemize}
    \item superblock: descrive il file system  montato
    \item v-node: descrive il file control block associato ad uno specifico file
    \item file: descrive un file aperto associato ad un processo
    \item directory entry: descrive una specifica voce in una directory
\end{itemize}
Inoltre, la VFS contiene strutture dati interne, quali mount table, system-wide open file table e per-process open file
table. Non memorizza nulla su disco, gestisce tutto in memoria.\\
Ad esempio, per aprire un file data system call \textit{open}, la VFS cerca il superblock del file system nella mount 
table, trova la root directory e cerca il giusto path tramite funzioni del file system concreto; una volta trovato il 
file corrispondente, crea un v-node che contiene puntatori alle funzioni per manipolare file, esegue la funzione di
\textit{open}, crea una nuova voce nella v-node table e nella file descriptor table, ritornando il file descriptor
appena creato.
\section{Gestione e ottimizzazione dei file system}
\subsection{Gestione dello spazio su disco}
La dimensione dei blocchi è di solito determinata durante la formattazione ad alto livello. Qual è la giusta misura?
Blocchi troppo grandi sono proni a frammentazione interna e rischiano di sprecare spazio su disco.
Blocchi troppo piccoli richiedono molto overhead e leggere file composti da molti blocchi richiede maggior tempo.
La dimensione dei blocchi influenza il data rate (bytes trasferiti al secondo) e spazio usato/allocato.
Il tempo d'accesso è dominato dal tempo di seek e dal delay rotazionale; più grandi sono i blocchi, maggiore sarà il
data rate, ma minore sarà l'efficienza in spazio (viceversa per blocchi piccoli).\\
Per tenere traccia dei blocchi liberi sono usati due approcci: liste collegate e bitmap. 
Il primo consiste in una lista contenente blocchi contenenti puntatori a blocchi liberi (e nodo successivo della lista).
Il secondo consiste in una mappa di bit dove ogni bit corrisponde ad un blocco su disco. La bitmap richiede spazio 
costante indipendentemente del numero di blocchi liberi, mentre la lista è più efficiente quando il disco è quasi pieno.
I blocchi liberi di solito sono collocati uno di fianco all'altro, quindi la lista potrebbe tenere conto solo della
grandezza di questa fila continua di blocchi liberi.\\
Le strutture dati per la gestione dello spazio libero sono di solito memorizzate su disco ma portate in memoria per 
ridurre le operazioni di I/O. Con lo schema a lista si può tenere un solo nodo in memoria, fino a quando non si
esauriscono i puntatori a blocchi liberi: questo metodo può però risultare in molte operazioni I/O nel caso il nodo in
memoria sia quasi vuoto. Per limitare il problema è possibile dividere un nodo pieno per evitare di avere nodi quasi
vuoti in memoria. Per le bitmap, è possibile memorizzare in memoria solo alcuni blocchi al posto dell'intera bitmap.\\
Le \textbf{disk quotas} sono un meccanismo per evitare che gli utenti di un sistema multiutente consumino troppo spazio.
Allo scopo, eiste una \textit{quota table} con voce per ogni utente. Per ogni file aperto, il corrispondente file 
control block nella system-wide open file table contiene l'ID del proprietario e un puntatore alla quota. Ogni volta
che un blocco viene aggiunto ad un file aperto, il numero di blocchi totali per l'utente è incrementato e vengono 
controllati i limiti. Se il soft limit viene superato, viene visualizzato un avviso e il numero di avvisi rimanenti è
decrementato (periodo di grazia). 
\subsection{Backups}
Creare backup è un'operazione lunga e costosa in termini di spazio. Di solito non è necessario fare un backup dell'
intero sistema, ma solo dei file veramente importanti e che sono stati modificati dall'ultimo backup.\\
ll \textbf{backup differenziale} consiste nel fare backup dei soli file modificati dall'ultimo backup totale: rende 
facile il recupero dei file ma richiede più spazio.
Il \textbf{backup incrementale} consiste nel fare backup delle modifiche intercorse dall'ultimo backup (totale o 
parziale): occupa meno spazio ma rende il recupero dei file più complesso e meno affidabile.\\
Comprimere i dati permette di risparmiare spazio ma un singolo errore può invalidare un intero backup, quindi questa
operazione deve essere considerata con precauzione.\\
Creare backup quando un sistema è online può causare inconsistenze, quindi sono stati creati algoritmi per fare
snapshot dello stato del file system: quando uno snapshot è creato, ogni seguente modifica a file e directory è
applicata alle nuove copie dei blocchi originali.\\
Un disco può essere copiato su un disco di backup in due modi: dump fisico e dump logico.
Il \textbf{dump fisico} inizia al primo blocco del disco (o partizione) e procede sequenzialmente: è indipendente dal
file system utilizzato, è molto veloce, non modifica metadati e può copiare file system non montati. In questo modo,
però, sono copiati anche i blocchi vuoti: per evitarlo, il programma di backup deve sapere a priori come il file
system è implementato, e questo processo rompe la mappatura uno ad uno dei blocchi tra dischi. Inoltre, si rischia di
copiare blocchi corrotti e file appartenenti al SO che non dovrebbero essere copiati.\\
Il \textbf{dump logico} inizia da una o più directory specificate e copia ricorsivamente tutti i file e directory
presenti; deve includere tutte le directory contenute nel path di una directory o file modificato, per 
rendere possibile il recupero del suddetto file. Il dump logico è più lento del dump fisico, ma è più preciso.\\
Un algoritmo di dump basato su UNIX mantiene una bitmap indicizzata al numero di i-node ed opera in 4 fasi:
\begin{enumerate}
    \item Tutte le velocizzare (a partire dalla directory specificata) sono esaminate, e per ogni file modificato il suo 
    i-node è segnato nella bitmap
    \item L'albero delle directory è ripercorso ricorsivamente, tenendo conto delle directory che non sono da copiare
    \item Gli i-node della bitmap sono scansionati e le directory da copiare sono copiate
    \item Gli i-node della bitmap sono scansionati e i file da copiare sono copiati
\end{enumerate}
Con questo metodo, il recupero del file system è molto semplice: prima sono recuperate le directory e poi i file.
Ci sono però dei problemi: la lista dei blocchi liberi non viene copiata quindi dev'essere ricostruita dopo ogni dump,
i file linkati devono essere copiati una volta sola, i file speciali e i blocchi vuoti contenuti negli sparse file 
(file con "buchi") non devono essere copiati.
\subsection{Consistenza}
Per aumentare efficienza, il sistema mantiene i blocchi modificati in memoria e li scrive su disco solo successivamente.
Un file system può ritrovarsi in uno stato inconsistente a causa di crash o spegnimento improvviso del sistema.\\
I SO hanno programmi per controllare la consistenza del file system; in UNIX è \textit{fsck}: \textit{fsck} controlla
sia la consistenza del blocchi che quella dei file.\\
Per controllare la consistenza dei blocchi, \textit{fsck} usa due tabelle, una per i blocchi in uso e una per blocchi
liberi. Per prima cosa, \textit{fsck} legge tutti gli i-node, legge gli indirizzi allocati per ogni file e incrementa il
contatore corrispondente nella tavola dei blocchi in uso, poi esamina la tabella dei blocchi liberi per trovare tutti i
blocchi non in uso. Un file system è consistente a livello di blocchi se ogni blocco ha il contatore corrispondente
settato a 1 in una tavola e 0 nell'altra. Si possono verificare diversi tipi di incosistenza:
\begin{itemize}
    \item blocco con contatore a 0 in entrambe le tabelle, \textit{fsck} lo aggiunge alla tabella dei blocchi liberi
    \item blocco con contatore a 1 in entrambe le tabelle, \textit{fsck} può rimuoverlo da uno delle dei tabelle
    \item blocco con contatore $>$ 1 nella tabella dei blocchi liberi, \textit{fsck} ricostruisce la tabella dei blocchi
    liberi per eliminare l'inconsistenza
    \item blocco con contatore $>$ 1 nella tabella dei blocchi in uso, \textit{fsck} alloca un blocco libero, copia il
    contenuto del blocco duplicato in quel blocco e inserisce la copia in uno dei file
\end{itemize}
\textit{fsck} controlla la consistenza dei file, usando una singola tabella che tiene traccia di quante volte un file 
appare nell'albero di una directory (una voce per ogni i-node). Un file system è consistente a livello file se per ogni 
file il contatore correlato presente nella tabella combacia con il contatore dei link nell'i-node corrispondente.
Possono presentarsi due tipi di errori:
\begin{itemize}
    \item contatore dei link di un i-node > del contatore nella tabelle dei file in uso: quando tutti i link ad un i-node
    sono rimossi dalle directory, il contatore sarà > 0 e l'i-node non verrà rimosso, quindi \textit{fsck} setta il 
    contatore dei link con il valore corretto presente nella sua tabella
    \item contatore dei link di un i-node < del contatore nella tabelle dei file in uso: quando il contatore dei link
    raggiunge lo 0, il file system lo marca come non in uso, ma alcuni file punteranno ancora a tale i-node, quindi
    \textit{fsck} setta il contatore dei link con il valore corretto presente nella sua tabella
\end{itemize}
Sono possibili altri controlli: controllo dello stato degli i-node (se è corrotto, i blocchi associati sono rilasciati 
e l'i-node eliminato), directory (segnala directory con numero di file sospetto o eccessivi permessi), superblock e 
bad-blocks (blocchi che contengono bad-sectors).
\subsection{Journaling file systems}
L'obiettivo principale è garantire robustezza in caso di problemi, mantenendo un log di ciò che il file system sta per
fare prima di farlo (\textbf{write-ahead log}), in modo da poter riprendere da quel punto in caso di crash del sistema.
Per rendere il jounaling funzionale, le operazioni loggate devono essere idempotenti, ovvero possono essere ripetute
senza causare danni. Per garantire maggiore efficienza, vengono usati \textbf{log circolari}. Ogni voce non vuota
rappresenta una transazione da svolgere; quando tutte le operazioni sono state svolte, la voce è marcata come completata
(checkpoint). Il recupero inizia dalla transazione seguente l'ultimo checkpoint e procede finchè non viene trovata una
voce vuota.\\
Il jounaling introduce problemi di efficienza, dato le scritture su disco sono raddoppiate. Per ovviare a questo 
problema, il log è allocato sequenzialmente (meno opeazioni di seek) e contiene solo metadati (non i dati dei file).
Programmi come \textit{fsck} servono anche in un file system dotato di journaling, dato che permettono di ricoverarsi
da situazioni inaspettate e forniscono ulteriori controlli.
\subsection{Performance}
Il \textbf{block caching} è una tecnica utilizzata per ridurre l'accesso a disco. Una \textbf{block cache} è un insieme 
di blocchi che logicamente appartiene al disco ma tenuto in memoria per motivi di efficienza: per ogni operazione
di lettura/scrittura, viene prima controllata la presenza del blocco nella cache; se non è presente, viene letto da
disco e aggiunto alla cache.\\
Per verificare velocemente se un blocco è contenuto in cache, si usa una hashtable. 
Per selezionare una vittima dalla cache, sono usati gli stessi algoritmi impiegati nella paginazione: nel caso di politica
LRU, la vittima è il blocco acceduto meno recentemente. Questa politica è implementata tramite lista bidirezionale, con
i blocchi usati meno di recente di fronte e quelli usati recentemente nel retro. Nel caso di crash, però, le modifiche 
in cache vanno perdute, lasciando il file system in uno stato inconsistente: è quindi assegnata priorità maggiore a
blocchi critici (file control block, tabella dei blocchi liberi...), i quali vengono scritti immediatamente su disco.
In questo modo, potrebbe verificarsi la situazione in cui i blocchi di dati non sono frequentemente scritti su disco. In
UNIX, la soluzione è usare \textbf{write-back caches}: quando un blocco in cache è modificato, è marcato come "dirty" e 
il programma \textit{update} scrive periodicamente i blocchi "dirty" su disco. Tale sistema è poco affidabile e complesso
da implementare. In passato, venivano usate \textbf{write-through caches}, dove i blocchi modificati venivano subito
scritti su disco, a discapito dell'efficienza.\\
Con il metodo \textbf{block read-ahead}, i blocchi sono portati in cache prima del loro utilizzo per incrementare 
l'hit-rate della cache: quando il blocco di un file viene letto, gli altri blocchi corrispondenti al file sono portati 
nella cache. Questa strategia funziona perfettamente quando i file sono letti sequenzialmente, ma è dannosa quando sono
letti randomicamente, quindi il SO tiene traccia dei pattern d'accesso di ogni file e decide se è meglio usare la 
suddetta strategia o meno.\\
Esistono varie tecniche per ridurre il movimento del braccio del disco. La prima consiste nell'allocare blocchi che
sono accceduti sequenzialmente in modo che siano vicini tra loro. L'operazione di allocazione dei blocchi liberi per un
file è veloce se il SO tiene traccia dei blocchi in uso e non tramite una bitmap, ma lenta se usa una lista. Può essere
possibile allocare un gruppo di blocchi quando ne è richiesto uno solo, e rilasciarli nel caso non servano.\\
La seconda consiste nell'allocare gli i-node (nei sistemi che li posseggono) a metà del disco, invece che all'inizio;
oppure dividere il disco in \textit{cylinder groups}, ognuno con i suoi i-node, blocchi dati e tabella dei blocchi
liberi: quando bisogna allocare un nuovo blocco, si cerca di trovarne uno disponibile nello stesso cylinder group in cui
è contenuto l'i-node.
\end{document}
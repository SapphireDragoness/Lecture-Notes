\documentclass[11pt]{article}
\usepackage[margin=.8in]{geometry}
\usepackage[italian]{babel}
\usepackage{tikz}
\usepackage{amsfonts} 
\usepackage{amsthm}
\usepackage[Algoritmo]{algorithm}
\usepackage{algpseudocode}
\usepackage{float}

\title{Appunti Algoritmi 2}

\newtheorem*{theorem}{Teorema}
\newtheorem*{proprietà}{Proprietà}
\newtheorem*{lemma}{Lemma}
\theoremstyle{remark}
\newtheorem*{remark}{Proprietà}

\begin{document}
\section{Grafi come strutture dati}
\subsection{Introduzione e terminologia}
Un grafo è una coppia di elementi (insiemi) \textbf{G=(V,E)} e consiste in:
\begin{itemize}
    \item un insieme $V$ di \textbf{vertici} (o \textbf{nodi})
    \item un insieme $E$ ($E$ sottoinsieme del prodotto cartesiano $V\times V$) di coppie di vertici, detti \textbf{archi}
    o \textbf{spigoli}; ogni arco connette due vertici
\end{itemize}
I grafi possono essere:
\begin{itemize}
    \item \textbf{orientati}: relazioni asimmetriche, insieme di coppie ordinate
    \item \textbf{non orientati}: relazioni simmetriche, insieme di coppie non ordinate
\end{itemize}
Un arco è \textbf{incidente} per i nodi che si toccano.\\
Il \textbf{grado} di un vertice è dato dal numero di archi incidenti.\\
Un vertice $B$ è \textbf{adiacente} ad $A$ se da $B$ si può percorrere un solo arco e giungere ad $A$.\\
Un \textbf{sottografo} è una porzione di grafo (notazione $H\subseteq G$): i vertici di $H$ sono sottoinsieme dei vertici di $G$ 
e gli archi di $H$ sono sottoinsieme degli archi di $G$.\\
Un \textbf{cammino} è una sequenza ordinata di archi che collegano due nodi. I cammini devono rispettare l'orientamento 
degli archi. La \textbf{lunghezza} è il numero di archi di cui è composto un cammino.\\ Un cammino si dice \textbf{semplice} 
se non passa due volte per lo stesso vertice. Se esiste almeno un cammino $p$ tra i vertici $v$ e $w$, si dice che $w$ è 
\textbf{raggiungibile} da $v$. Inoltre $v$ è un \textbf{antenato} di $w$ e $w$ è un \textbf{discendente} di $v$.\\ 
Un cammino tra due nodi $v$ e $w$ si dice \textbf{minimo} se tra $v$ e $w$ non esiste nessun altro cammino di lunghezza 
minore. La lunghezza del cammino minimo è detta \textbf{distanza} ($\delta (v,w)$).\\
Un grafo può essere \textbf{pesato}. La funzione peso è definita come $W:E\rightarrow \mathbb{R}$; per ogni arco
$(v,w)\in E,W(v,w)$ definisce il \textbf{peso} di $(v,w)$. In un grafo pesato, la lunghezza/peso di un cammino si calcola 
sommando i pesi degli archi che contiene.\\
I grafi non orientati possono essere:
\begin{itemize}
    \item \textbf{connessi}: esiste un cammino da ogni vertice verso ogni altro vertice
    \item \textbf{non connessi}
\end{itemize}
I grafi orientati possono essere:
\begin{itemize}
    \item \textbf{fortemente connessi}: esiste un cammino da ogni vertice verso ogni altro vertice
    \item \textbf{debolmente connessi}: ignorando il verso degli archi
\end{itemize}
Un cammino $\langle w_1,w_2,...,w_n \rangle$ si dice \textbf{chiuso} se $w_1=w_n$. Un cammino chiuso, semplice, di lunghezza 
almeno 1 si dice \textbf{ciclo}. Se un grafo non contiene cicli, si dice \textbf{aciclico}.\\
Un \textbf{grafo completo} è un grafo con un arco per ogni coppia di vertici. Un grafo completo ha numero di archi $E$ pari 
a $|E| = \frac{|V|(|V|-1)}{2}$.\\
Un grafo non orientato, connesso e aciclico è definito \textbf{albero libero}. Se un vertice è designato ad essere radice, 
si definisce \textbf{albero radicato}. Un grafo non orientato, aciclico ma non connesso è definito \textbf{foresta}.
\subsection{Rappresentazione}
Per valutare un approccio di rapppresentazione, bisogna considerare lo \textbf{spazio} occupato dalla struttura dati e il 
\textbf{costo computazionale} delle operazioni da effettuare su di essa.
\subsubsection{Lista di archi}
Dati $n$ (numero di vertici) e $m$ (numero di archi), lo spazio occupato è $\mathcal{O}(n+m)$: è una rappresentazione 
inefficiente, in quanto bisogna percorrere tutto il grafo per scandire la lista di archi. Introdurre un vertice o arco ha 
costo $\mathcal{O}(1)$, ma la rimozione ha costo $\mathcal{O}(m)$.
\subsubsection{Liste di adiacenza}
Ogni vertice $v$ ha una lista contenente i vertici ad esso adiacenti. Calcolare il grado di un vertice è un'operazione 
semplice, in quanto basta scorrere la lista di adiacenza. Occupa spazio $\mathcal{O}(n+m)$, ed è adatta per grafi \textbf{sparsi} 
(il numero di archi è molto minore del numero di vertici).
\subsubsection{Liste di incidenza}
Ogni vertice $v$ ha una lista contenente un riferimento agli archi ad esso incidenti. Occupa spazio $\mathcal{O}(n+m)$.
\subsubsection{Matrici di adiacenza}
Il grafo è rappresentato tramite una matrice di interi di grandezza $n \times n$ (spazio occupato $\mathcal{O}(n^2)$); è 
adatta per grafi \textbf{densi}. Calcolare il grado e archi incidenti ha costo $\mathcal{O}(n)$ (basta scorrere la matrice). 
La modifica dei vertici ha costo $\mathcal{O}(n^2)$ in quanto bisogna ricostruire completamente la matrice.
Una matrice di adiacenza rappresenta anche la presenza di un cammino di lunghezza 1 tra ogni coppia di vertici $v$ e $w$. 
In particolare, $v \rightarrow_1 w$ se e solo se $M[v,w]\neq 0$: moltiplicando la matrice per sè stessa, il risultato è 
diverso da 0 solo se esiste un cammino di lunghezza 2 (e via dicendo).
\subsubsection{Matrici di incidenza}
Il grafo è rappresentato tramite una matrice di interi di grandezza $n \times m$ (spazio occupato $\mathcal{O}(n \times m)$), 
in cui le righe indicizzano i vertici e le colonne indicizzano gli archi.
\section{Visite}
\subsection{Visita generica}
Una \textbf{visita} di un grafo $G$ permette di esaminare i nodi e gli archi in maniera sistematica, senza passare due 
volte per lo stesso nodo.
\subsubsection{Inizializzazione}
Una tattica per evitare di visitare un nodo più volte è quella di mappare lo stato della visita ad un colore:
\begin{itemize}
    \item \textbf{bianco} (o \textbf{nodi inesplorati}): vertice non ancora esplorato
    \item \textbf{grigio} (o \textbf{nodi aperti}): vertice visitato, ma con nodi adiacenti ancora inesplorati
    \item \textbf{nero} (o \textbf{nodi chiusi}): vertice visitato, con adiacenti esplorati
\end{itemize}
Dati $n$ nodi, si utilizza un vettore \textit{color} di colori, di grandezza $n$: all'inizio della visita, tutte le celle 
del vettore \textit{color} sono impostate a \textit{white}.
\begin{algorithm}[H]
    \caption{INIZIALIZZA(G)}
    \begin{algorithmic}
        \State color $\gets$ vettore di lunghezza n
        \For{ogni $u\in V$}
        \algstore{1}
    \end{algorithmic}
\end{algorithm}
\begin{algorithm}[h]
    \begin{algorithmic}
        \algrestore{1}
            \State color[$u$] $\gets$ white
        \EndFor
    \end{algorithmic}
\end{algorithm}
La visita parte da un nodo $s$, detto \textbf{nodo sorgente}.
\begin{algorithm}[H]
    \caption{VISITA(G,s)}
    \begin{algorithmic}
        \State INIZIALIZZA(G)
        \State color $\gets$ gray
        \State \{visita $s$\}
        \While{ci sono vertici grigi}
            \State $u \gets$ scegli un vertice grigio
            \If{esiste $v$ bianco adiacente ad $u$}
                \State color[$v$] $\gets$ gray
                \State \{visita $v$\}
            \Else{ color[$v$] $\gets$ black}
            \EndIf
        \EndWhile
    \end{algorithmic}
\end{algorithm}
Il cambiamento di colore è \textbf{monotono} (bianco $\rightarrow$ grigio $\rightarrow$ nero).
\subsubsection{Invarianti}
Un'\textbf{invariante} è una condizione che è verificabile come vera sia all'inizio sia alla fine di un ciclo:
\begin{itemize}
    \item Invariante 1: se esiste un arco $(u,v)\in E$ ed $u$ è nero, allora $v$ è grigio o nero 
    \item Invariante 2: tutti i vertici grigi o neri sono raggiungibili dalla sorgente 
    \item Invariante 3: qualunque cammino dalla sorgente ad un vertice bianco deve contenere almeno un vertice grigio 
\end{itemize}
\begin{theorem}
    Al termine dell'algoritmo di visita, $v$ è nero se e solo se $v$ è raggiungibile dalla sorgente.
\end{theorem}
\begin{proof}
    Per l'invariante 2, all'uscita dal ciclo tutti i vertici neri sono raggiungibili da $s$. Dall'invariante 
    3 si ricava che tra $s$ e $v$ esiste almeno un vertice grigio, oppure $v$ non è bianco. Dato che la condizione di uscita 
    dal ciclo è quella che non esistano più vertici grigi, si ricava che $v$ non è bianco (cambiamento monotono) e non può 
    essere grigio. Quindi, all'uscita dal ciclo, tutti i vertici raggiungibili dalla sorgente sono neri.
\end{proof}
\subsubsection{Predecessori}
L'algoritmo può essere modificato in modo da ricordare, per ogni vertice che viene scoperto, quale vertice grigio ha permesso 
di scoprirlo, ossia ricordare l'arco percorso. Ad ogni vertice $u$ si associa un attributo $\pi[u]$ che rappresenta il 
vertice che ha permesso di scoprirlo.
\begin{algorithm}[H]
    \caption{VISITA(G,s)}
    \begin{algorithmic}
        \State INIZIALIZZA(G)
        \State color $\gets$ gray
        \State \{visita $s$\}
        \While{ci sono vertici grigi}        
        \algstore{3}
    \end{algorithmic}
\end{algorithm}
\begin{algorithm}[H]
    \begin{algorithmic}
        \algrestore{3}
        \State $u \gets$ scegli un vertice grigio
            \If{esiste $v$ bianco adiacente ad $u$}
                \State color[$v$] $\gets$ gray
                \State $\pi[v] \gets u$
                \State \{visita $v$\}
            \Else{ color[$v$] $\gets$ black}
            \EndIf
        \EndWhile
    \end{algorithmic}
\end{algorithm}
\begin{remark}
    Al termine dell'esecuzione di VISITA(G,s), tutti e soli i vertici neri diversi da $s$ hanno predecessore diverso da 
    NULL.
\end{remark}
Il sottografo dei predecessori è un albero (\textbf{albero dei predecessori}) di radice $s$.\\
Se il grafo non è connesso:
\begin{algorithm}[H]
    \caption{VISITA TUTTI I VERTICI(G)}
    \begin{algorithmic}
        \State INIZIALIZZA(G)
        \For{ogni $u\in V$}
            \If{color[$u$] = white} 
            \State VISITA(G,u)
            \EndIf
        \EndFor
    \end{algorithmic}
\end{algorithm}
\subsubsection{Gestione dei vertici grigi}
Per gestire i nodi grigi si usa una struttura dati ordinata $D$ (\textbf{frangia}). Sulla frangia è possibile eseguire le 
seguenti operazioni:
\begin{itemize}
    \item \textbf{Create()}: restituisce una $D$ vuota
    \item \textbf{Add(D,x)}: aggiunge un elemento x a $D$
    \item \textbf{First(D)}: restituisce il primo elemento di $D$
    \item \textbf{RemoveFirst(D)}: elimina il primo elemento di $D$
    \item \textbf{NotEmpty(D)}: restituisce vero se D contiene almeno un elemento, falso altrimenti
\end{itemize}
$D$ è una \textbf{coda} se Add(D,x) aggiunge l'elemento in coda a $D$, uno \textbf{stack} se Add(D,x) aggiunge l'elemento 
in testa a $D$.
\begin{algorithm}[H]
    \caption{VISITA(G,s)}
    \begin{algorithmic}
        \State INIZIALIZZA(G)
        \State Create() 
        \State color[$s$] $\gets$ gray
        \State \{visita $s$\}
        \State Add(D,s)
        \While{NotEmpty(D)}
        \State $u \gets$ First(D)
        \If{esiste $v$ bianco adiacente ad $u$}
            \State color[$v$] $\gets$ gray
        \algstore{5}
    \end{algorithmic}
\end{algorithm}
\begin{algorithm}[h]
    \begin{algorithmic}
        \algrestore{5}
            \State $\pi[v] \gets u$
            \State \{visita $v$\}
            \State Add(D,v)
        \Else
            \State color[$v$] $\gets$ black
            \State RemoveFirst(D)
        \EndIf
    \EndWhile
    \end{algorithmic}
\end{algorithm}
\subsubsection{Complessità}
Il costo di visita è $\mathcal{O}(n+adj)$: $adj$ è il tempo impiegato a controllare se esiste un nodo $v$ bianco adiacente 
ad $u$, e dipende dalla rappresentazione; $n$ è il numero di vertici, che vengono inseriti e rimossi da $D$.\\
Il costo di $adj$ è:
\begin{itemize}
    \item con lista di archi: bisogna scandire l'intera lista ($\mathcal{O}(m)$) per $n$ volte ($\mathcal{O}(n)$), quindi 
    $\mathcal{O}(n)+\mathcal{O}(n*m)=\mathcal{O}(mn)$
    \item con matrice di adiacenza: bisogna scandire l'intera riga della matrice ($\mathcal{O}(n)$), quindi $\mathcal{O}(n)+\mathcal{O}(n*n)=\mathcal{O}(n^2)$
    \item con liste di adiacenza: si possono ottimizzare le prestazioni utilizzando dei puntatori che puntano all'inizio 
    delle liste di adiacenza. Se l'elemento è grigio, il puntatore è spostato all'elemento successivo; quando il puntatore 
    giunge alla fine della lista, il primo elemento è colorato di nero. Ogni lista è percorsa una volta sola, in tutte le 
    iterazioni del ciclo. Complessità: $\mathcal{O}(n+m)$.
\end{itemize}
\subsection{Visita in ampiezza}
\subsubsection{Inizializzazione}
La \textbf{visita in ampiezza} (\textbf{BFS}, Breadth First Search), esamina i vertici del grafo in un ordine ben preciso,
costruendo un albero di visita chiamato \textbf{albero BFS}. Nell'albero BFS, ogni vertice si trova il più vicino possibile 
alla radice. La visita è realizzata usando la frangia come coda: quando un nodo grigio ha tutti gli adiacenti grigi, esso 
è rimosso dalla cosa (il vertice in testa rimane nella coda finchè non diventa nero).
\begin{algorithm}[h]
    \caption{VISITA BFS(G,s)}
    \begin{algorithmic}
        \State INIZIALIZZA(G)
        \State queue() 
        \State color[$s$] $\gets$ gray
        \State \{visita $s$\}
        \State enqueue(D,s)
        \While{NotEmpty(D)}
        \State $u \gets$ head(D)
        \If{esiste $v$ bianco adiacente ad $u$}
            \State color[$v$] $\gets$ gray
            \State $\pi[v] \gets u$
            \State \{visita $v$\}
            \State enqueue(D,v)
            \algstore{6}
    \end{algorithmic}
\end{algorithm}
\begin{algorithm}[h]
    \begin{algorithmic}
        \algrestore{6}
        \Else
            \State color[$v$] $\gets$ black
            \State dequeue(D)
        \EndIf
    \EndWhile
    \end{algorithmic}
\end{algorithm}
\subsubsection{Albero di visita}
L'albero BFS viene costruito a livelli; l'albero rappresenta i cammini minimi. Anche se le liste di adiacenza vengono 
invertite, i nodi per livello non cambiano.
Si può inizializzare un \textbf{vettore di distanze} (stimate) $d$, inizializzato ad infinito: se un determinato vertice non
è stato trovato (distanza $\infty$ a fine BFS), allora non è raggiungibile da $s$.
\subsubsection{Proprietà}
\begin{proprietà}[1]
    In D ci sono tutti e soli i vertici grigi.
\end{proprietà}
\begin{proprietà}[2]
    Se $\langle v_1,v_2,\dots ,v_n \rangle$ è il contenuto di D, allora:
    \begin{itemize}
        \item[i] $d[v_i] \leq d[v_{i+1}]$: i vertici sono ordinati per livelli nella coda
        \item[ii] $d[v_n] \leq d[v_1]+1$: la coda contiene al massimo due livelli
    \end{itemize}
\end{proprietà}
\begin{proof}
    Nel caso base, in $D$ è presente solo la sorgente. La proprietà 2 è vera.\\
    Il passo ha due casi:
    \begin{itemize}
        \item dequeue(D): o $D$ rimane vuota (banalmente vera), o rimangono $\langle v_2,\dots ,v_n \rangle$, e 
        \begin{itemize}
            \item[i.] le disuguaglianze sono ancora vere e quindi
            \item[ii.] anche $d[v_n] \leq d[v_1]+1 \leq d[v_2]+1$
        \end{itemize}
        \item enqueue(D,v): $v$ è reso figlio di $v_1$ e accodato, quindi $d[v]=d[v_1]+1$ e
        \begin{itemize}
            \item[i.] $d[v_n] \leq d[v_1]+1=d[v]$
            \item[ii.] $d[v]=d[v_1]+1 \leq d[v_1]+1$
        \end{itemize}
    \end{itemize}
\end{proof}
\subsubsection{Dimostrazione $d[v]=\delta(s,v)$}
\begin{lemma}[Invariante 4]
    $d[v]=\delta(s,v)$ per tutti i vertici grigi o neri.
\end{lemma}
\textit{\textbf{Dimostrazione $d[v] \geq \delta(s,v)$.}} Dato che l'albero dei predecessori $\pi$ contiene solo archi appartenenti 
a $G$, il cammino da $s$ a $v$ è un cammino che appartiene anche a $G$, quindi la lunghezza del cammino da $s$ a $v$ 
nell'albero è maggiore o uguale alla distanza tra $s$ e $v$.

\textit{\textbf{Dimostrazione $d[v] \leq \delta(s,v)$.}} Definiamo l'insieme dei vertici a distanza $k$ dalla sorgente nel grafo 
come $V_k={v\in V|\delta(s,v)=k}$ ($v_0$ contiene solo la sorgente).\\ Nel caso base, $d[v_0]\leq \delta(s,v)$ (distanza di 
$s$ da sè stesso: $0\leq 0$). Sia $v\in V_k$: allora $\delta(s,v)=k$ (per definizione).\\ Con $k>0$ (passo), esisterà almeno 
un vertice $w$ tale che $\delta(s,w)=k-1$ e $(w,v)\in E$, ovvero un arco che va da $v$ a $w$. Definiamo l'insieme dei vertici 
appartenenti a $V_{k-1}$ con arco entrante in $v$ come $U_{k-1}={w\in V_{k-1}|(w,v)\in E}$. Tra questi, sia $u$ il primo 
vertice di $U_{k-1}$ ad essere scoperto ed inserito nella coda: per politica FIFO, $u$ sarà anche il primo ad essere estratto
dalla coda. Quando guarderò i vertici adiacenti a $u$, $v$ sarà ancora bianco (perchè più lontano), e $v$ verrà inserito 
nell'albero come figlio di $u$, con $d[v]=d[u]+1$. Inoltre, per ipotesi induttiva, $d[u]\leq k-1$.\\
Quindi, quando inseriremo $v$ nell'albero:
\begin{itemize}
    \item $d[v]=d[u]+1$
    \item ma $d[u]\leq k-1$, quindi $d[v]\leq (k-1)+1$
    \item $d[v]\leq k$
\end{itemize}
\subsubsection{Distanza nell'albero BFS}
\begin{theorem}
    Al termine dell'esecuzione della visita BFS, si ha $d[v]=\delta(s,v)$ per tutti i vertici $v\in V$.
\end{theorem}
\begin{proof}
    caso base: se $v$ non è raggiungibile da $s$, allora $d[v]$ rimane $\infty$.\\
    altrimenti, $v$ è nero (per invariante 2). Per ogni vertice $v$ raggiungibile da $s$, il cammino da $s$ a $v$ sull'albero 
    ottenuto dalla visita è un \textbf{cammino minimo}.
\end{proof}
\subsection{Visita in profondità}
\subsubsection{Inizializzazione}
La \textbf{visita in profondità} (\textbf{DFS}, Depth First Search), esamina i vertici del grafo partendo dall'ultimo 
vertice incontrato. Si ottiene dall'algoritmo di visita generico, implementando la frangia come \textbf{stack}.
\begin{algorithm}
    \caption{VISITA DFS (ottimizzata)}
    \begin{algorithmic}
        \State D $\gets$ emptyStack(D)
        \State color[$s$] $\gets$ gray
        \State \{visita $s$\}
        \State push(D)
        \While{esiste $v$ non considerato adiacente a top(D)}
            \If{color[$v$] = white}
                \State color[$v$] = gray 
                \State $\pi[v] \gets$ top(D) 
                \State \{visita $s$\}
                \State push(D,s)
            \EndIf 
            \State color[top(D)] $\gets$ black
            \State pop(D)
        \EndWhile
    \end{algorithmic}
\end{algorithm}
\subsubsection{Caratteristiche dell'albero DFS}
Un vertice viene chiuso (colorato di nero) solo quando tutti i suoi discendenti sono stati chiusi.\\
Gli intervalli di attivazione di una qualunque coppia di vertici sono o \textbf{disgiunti} o \textbf{uno contenuto interamente 
nell'altro}. Questo è l'ordine delle attivazioni delle chiamate di una procedura ricorsiva, ed è quindi possibile costruire 
un algoritmo DFS ricorsivo. Inoltre, è possibile introdurre un contatore per ricordare l'ordine delle attivazioni.
\begin{algorithm}[H]
    \caption{VISITA DFS RICORSIVA(G,u)}
    \begin{algorithmic}
        \State color[$u$] $\gets$ gray
        \State \{visita $s$\}
        \algstore{8}
    \end{algorithmic}
\end{algorithm}
\begin{algorithm}[H]
    \begin{algorithmic}
        \algrestore{8}
        \State d[$u$] $\gets$ time
        \State time++
        \For{ogni $v$ adiacente ad $u$}
            \If{color[$v$] = white}
                \State $\pi[v] \gets$ $u$
                \State VISITA DFS RICORSIVA(G,v)
            \EndIf 
            \State color[$u$] $\gets$ black
            \State f[$u$] $\gets$ time
            \State time++
        \EndFor
    \end{algorithmic}
\end{algorithm}
\subsubsection{Teorema delle parentesi e corollari}
\begin{theorem}[Parentesi]
    In ogni visita DFS di un grafo, per ogni coppia di vertici $u$, $v$, una ed una sola delle seguenti condizioni è soddisfatta: 
    \begin{itemize}
        \item $d[u]<d[v]<f[v]<f[u]$ ed $u$ è un \textbf{antenato} di $v$ in un albero della foresta DFS 
        \item $d[v]<d[u]<f[u]<f[v]$ ed $u$ è un \textbf{discendente} di $v$ in un albero della foresta DFS
        \item $d[u]<f[u]<d[v]<f[v]$ e tra $u$ e $v$ non esiste relazione (non sono adiacenti)
    \end{itemize}
\end{theorem}
\begin{theorem}[Annidamento degli intervalli]
    Una visita DFS di un grafo colloca un vertice $v$ come discendente proprio di un vertice $u$ in un albero della foresta 
    DFS se e solo se $d[u]<d[v]<f[v]<f[u]$.
\end{theorem}
\begin{theorem}[Cammino bianco]
    In una foresta DFS, un vertice $v$ è discendente del vertice $u$ se e solo se al tempo d[$u$] $v$ è raggiungibile da 
    $u$ con un cammino contenente solo vertici bianchi.
\end{theorem}
La complessità della visita DFS è $\mathcal(O)(m+n)$ (come la visita generica).
\section{Aciclicità}
È possibile verificare la presenza di un ciclo tramite una visita DFS.
\subsection{Grafi orientati}
Un arco $\langle u,v \rangle$ viene \textbf{percorso} quando si incontra $v$ nella lista degli adiacenti ad $u$. Durante 
la DFS di un grafo orientato, ogni arco è percorso una volta sola. Definiamo:
\begin{itemize}
    \item \textbf{Arco dell'albero}: arco inserito nella foresta DFS
    \item \textbf{Arco all'indietro}: arco che collega un vertice ad un suo antenato
    \item \textbf{Arco in avanti}: arco che collega un vertice ad un suo discendente 
    \item \textbf{Arco di attraversamento}: arco che collega due vertici che non sono in relazione 
\end{itemize}
Durante la visita di un grafo orientato, un arco $\langle u,v \rangle$ viene percorso quando si incontra $v$ nella lista 
degli adiacenti ad $u$. In quel momento, color[$v$] può essere:
\begin{itemize}
    \item \textbf{bianco}: $\langle u,v \rangle$ è un \textbf{arco dell'albero}
    \item \textbf{grigio}: $u$ è un discendente di $v$, $\langle u,v \rangle$ è un \textbf{arco all'indietro} 
    \item \textbf{nero}: $\langle u,v \rangle$ è un arco
    \begin{itemize}
        \item \textbf{in avanti} se $v$ è discendente di $u$
        \item \textbf{di attraversamento} altrimenti
    \end{itemize}
\end{itemize}
\subsection{Grafi non orientati}
Durante la DFS di un grafo orientato, ogni arco è percorso esattamente due volte. Definiamo:
\begin{itemize}
    \item \textbf{Arco dell'albero}: arco inserito nella foresta DFS
    \item \textbf{Arco all'indietro}: arco che collega un vertice ad un suo antenato
\end{itemize}
Durante la visita di un grafo orientato, un arco $\langle u,v \rangle$ viene percorso quando si incontra $v$ nella lista 
degli adiacenti ad $u$. In quel momento, color[$v$] può essere:
\begin{itemize}
    \item \textbf{bianco}: $(u,v)$ è un \textbf{arco dell'albero}
    \item \textbf{grigio}: $u$ è un discendente di $v$, $\langle u,v \rangle$ è un \textbf{arco all'indietro} 
    \item \textbf{nero}: $\langle u,v \rangle$ è un \textbf{arco all'indietro}
\end{itemize}
\subsection{Test di aciclicità}
\begin{theorem}[Grafo aciclico]
    Se un grafo (orientato o non orientato) contiene un ciclo, allora esiste un arco all'indietro. Viceversa, se una visita 
    in profondità produce un arco all'indietro, il grafo contiene un ciclo. Quindi, un grafo è aciclico se e solo se una 
    visita DFS non produce archi all'indietro.
\end{theorem} 
\begin{algorithm}[H]
    \caption{CICLICO(G)}
    \begin{algorithmic}
        \State INIZIALIZZA(G)
        \For{ogni nodo $u$ di $G$}
            \If{color[$u$] = white \textbf{and} VISITA RICORSIVA CICLO(G,u)}
                \State \textbf{return} true
            \EndIf 
        \EndFor
        \State \textbf{return} false
    \end{algorithmic}
\end{algorithm}
\begin{algorithm}[H]
    \caption{VISITA RICORSIVA CICLO(G,u)}
    \begin{algorithmic}
        \State color[$u$] $\gets$ gray
        \For{ogni $v$ adiacente ad $u$}
            \If{color[$v$ = white]}
                \State $\pi[v] \gets u$
                \If{VISITA RICORSIVA CICLO(G,v)}
                    \State \textbf{return} true
                \EndIf
                \ElsIf{$v\neq \pi[u]$ (color[$v$] = gray \textit{per grafi orientati})}
                    \State \textbf{return} true
            \EndIf 
        \EndFor 
        \State color[$u$] $\gets$ black 
        \State \textbf{return} false
    \end{algorithmic}
\end{algorithm}
\section{Ordinamento topologico}
Dato un \textbf{grafo orientato aciclico} (DAG) è sempre possibile ordinare i nodi in un \textbf{ordine topologico}, cioè 
in modo che non ci sia nessun arco all'indietro nell'ordinamento.
\subsection{Relazioni d'ordine parziale, totale e di raggiungibilità}
Una relazione di \textbf{ordine parziale}, su un insieme $I$, è definita come una relazione binaria $\leq$:
\begin{itemize}
    \item \textbf{riflessiva}
\end{itemize}
In un DAG, la relazione di \textbf{raggiungibilità} è una relazione di ordine parziale:
\begin{itemize}
    \item è riflessiva: ogni vertice è raggiungibile da se stesso 
    \item è antisimmetrica: se $v$ è raggiungibile da $u$ ed $u$ è raggiugibile da $v$, allora $v$ e $u$ sono coincidenti
    \item è transitiva: se $v$ è raggiungibile da $u$ e $w$ è raggiungibile da $v$, allora $w$ è raggiungibile da $u$
\end{itemize}

\section{}
\end{document}
\documentclass[11pt]{article}
\usepackage[margin=.8in]{geometry}
\usepackage[italian]{babel}
\usepackage{hyperref}
\usepackage{graphicx}
\usepackage{algorithm}
\usepackage{algpseudocode}

\title{Appunti Cybersecurity}

\begin{document}
\section{Cybersecurity essentials}
\subsection{Definizioni di sicurezza}
La \textbf{sicurezza informatica} è l'insieme dei servizi, delle regole organizzative e dei comportamenti individuali che proteggono 
i sistemi informatici di un'azienda. Ha il compito di proteggere le risorse da accessi indesiderati, garantire la riservatezza 
delle informazioni, assicurare il funzionamento e la disponibilità dei servizi a fronte di eventi imprevedibili.
\subsection{Proprietà di sicurezza}
\paragraph*{Autenticazione}
Il servizio di \textbf{autenticazione} si preoccupa dell'autenticità della comunicazione. Vengono definiti due tipi di 
autenticazione: \textbf{autenticazione delle controparti} e \textbf{autenticazione dell'origine dei dati}.
\paragraph*{Controllo degli accessi}
Il controllo degli accessi è l'abilità di limitare e controllare l'accesso ai sistemi e alle applicazioni tramite canali
di comunicazione. Ogni entità interessata ad accedere ad un servizio deve prima essere autenticata.
\paragraph*{Confidenzialità}
La confidenzialità è l'atto di proteggere i dati trasmessi dagli attacchi passivi. 
\paragraph*{Integrità dei dati}
Il principio di integrità è applicabile ad un flusso di dati

\section{Crittografia}
\subsection{Crittografia simmetrica}
Crittografia con chiave comune e unica, a basso carico di elaborazione.
\subsubsection{Algoritmi di crittografia simmetrica}
\paragraph*{DES}
Chiave a 56 bit + 8 di parità, se applicato 3 volte viene detto 3DES. 2DES è vulnerabile a un attacco di tipo known-plaintext
detto meet in the middle.
\paragraph*{IDEA}
Chiave a 128 bit, blocco dati da 64 bit, utilizza XOR, addizione mod16 e moltiplicazione mod$2^{16}+1$.
\paragraph*{RC2, RC4}
Più veloci di DES, con chiave a lunghezza variabile e blocchi da 64 bit.
\paragraph*{AES}
Ora si usa AES, con chiave fino a 256 bit e blocchi da almeno 128 bit.
\subsubsection{Applicazione algoritmi a blocchi}
\paragraph*{ECB (Electronic Code Book)}
Per cifrare dati in quantità superiore. Ogni blocco viene cifrato con lo stesso algoritmo separatamente. Sconsigliato 
perché cifra allo stesso modo blocchi identici.
\paragraph*{CBC (Cipher Block Chaining)}
Per cifrare dati in quantità superiore. Richiede IV, XOR tra blocco cifrato precedente e blocco da cifrare, poi applicazione 
algoritmo di cifratura.
\paragraph*{Padding}
Per cifrare dati in quantità inferiore. Aggiungo bit per riempire lo spazio vuoto. Alcuni tipi offrono controllo d'integrità,
applicando padding a tutti i blocchi.
\paragraph*{CTS (Cipher Text Stealing)}
Per cifrare dati in quantità inferiore. L'ultimo blocco è riempito con byte del penultimo blocco, questi due blocchi 
vengono scambiati durante la cifratura.
\paragraph*{CTR (Counter mode)}
Per cifrare dati in quantità inferiore. Accesso random al testo cifrato, usa algoritmo a blocchi per cifrare n bit alla 
volta
\subsubsection{Algoritmi stream}
Operano su un flusso di dati senza richiederne la divisione in blocchi, tipicamente su un bit o byte.
\paragraph*{Salsa20 e ChaCha20}
Chiavi da 128 o 256 bit. Operazione base: add-rotate-xor su 32 bit. Effettuano 20 volte mixing su input.
\subsubsection{Distribuzione chiavi}
Per una comunicazione privata tra $n$ persone occorrono $\frac{n(n-1)}{2}$ chiavi. Avviene tramite algoritmi per scambio
chiavi.
\subsection{Crittografia asimmetrica}
Le chiavi sono diverse e hanno funzionalità reciproca. È possibile generare un messaggio segreto per uno specifico destinatario 
conoscendone solo la chiave pubblica.
\subsubsection{Algoritmi a chiave pubblica}
\paragraph*{DSA}
Elevamento a potenza e logaritmo del risultato, utilizzato solo per firma digitale.
\paragraph*{RSA}
Può solo cifrare dati il cui valore sia inferiore al modulo pubblico. Funzionamento:
\begin{enumerate}
    \item modulo pubblico $n=pq$, con $p$ e $q$ primi, grandi e segreti
    \item $\phi=(p-1)(q-1)$
    \item esponente pubblico $e$ tale che $1<e<\phi$, $e$ coprimo $\phi$
    \item esponente privato: $d=e^{-1}\phi$
    \item chiave pubblica: $(n,e)$, chiave privata: $(n,d)$
\end{enumerate}
Solitamente le chiavi pubbliche hanno un $e$ che contiene solo due bit a 1 per ottimizzare le prestazioni.

RSA è debole se vengono utilizzati esponenti piccoli, stesse chiavi per firma e cifratura. Per renderlo più forte, 
aggiungere sempre del padding fresco prima di cifrare il messaggio e non firmare dati grezzi.
\subsubsection{Distribuzione chaivi per crittografia asimmetrica}
\paragraph*{Diffie-Hellman}
Sfrutta la difficoltà di risoluzione del problema dell'algoritmo discreto. 
\paragraph*{Curve ellittiche}
Problema del logaritmo discreto sulla curva, più complesso, permette di avere chiavi più corte. Firma digitale: ECDSA, key
agreement: ECDH, key distribution: ECIES.
\subsection{Funzioni di hash e digest}
\subsubsection{Digest}
È un riassunto a lunghezza fissa del messaggio da proteggere. Deve essere veloce da calcolare, difficile da invertire e 
non generare troppe collisioni (digest uguali). Un algoritmo di digest a $n$ bit è insicuro quando vengono generati più 
di $2^{\frac{n}{2}}$ digest perché si ha una probabilità di collisione pari al 50\%.
\subsubsection{Funzioni di hash}
Dividono il messaggio in blocchi e applicano la funzione base per ottenere il valore di hash.
\subsubsection{KDF (Key Derivation Function)}
Algoritmo che deriva una o più chiavi segrete da una password utilizzando una funzione pseudorandom.
\subsubsection{MAC, MIC, MID}
\begin{itemize}
    \item MIC (Message Integrity Code): per garantire l'integrità dei messaggi con l0aggiunta di un codice
    \item MAC (Message Authentication Code): fornisce autenticazione
    \item MID (Message IDentifier): identificatore univoco per evitare attacchi di tipo replay
\end{itemize}
\subsubsection{Autenticazione tramite cifratura simmetrica}
Si invia una copia cifrata dei dati, solo chi conosce la chiave può confrontare la copia con l'originale. Veriffica 
l'integrità esatta ma raddoppia tempo e spazio.
\subsubsection{Autenticazione tramite digest e cifratura simmetrica}
Si invia un digest cifrato dei dati, solo chi conosce la chiave può confrontare il digest trasmesso con quello calcolato 
sui dati ricevuti.
\subsubsection{Autenticazione tramite keyed-digest}
Si invia un digest calcolato non solo sui dati ma anche sulla chiave. Soluzione più veloce. Attaccabile scambiando l'ordine 
dei blocchi.
\paragraph*{HMAC}
La funzione di hash prende in input un blocco da $b$ byte e genera un blocco da $l$ byte, con $b\geq l$.
\paragraph*{CBC-MAC}
Sfrutta un algoritmo di cifratura simmetrico a blocchi, in modalità CBC con IV nullo, prendendo come MAC la cifratura 
dell'ultimo blocco. È sicuro solo per messaggi a lunghezza fissa.
\subsubsection{Garantire integrità e riservatezza}
\begin{enumerate}
    \item authenticate-and-encrypt: si decifra prima di verificare l'integrità, vulnerabile a attacchi DoS 
    \item authenticate-then-encypt: si decifra prima di verificare l'integrità, vulnerabile a attacchi DoS, sicura solo 
    con CBC o cifratura stream
    \item encrypt-then-authenticate: si può evitare di decifrare se il MAC è errato, sicura, bisogna includere nel MAC 
    l'IV e gli algoritmi
\end{enumerate}
\subsubsection{Authenticated encryption}
Unica opzione che garantisce riservatezza e autenticazione. Si usa una sola chiave e un solo algoritmo, più veloce, meno 
errori nel combinare le funzioni. Crea IGE (Infinite Garble Extension), che causa errore su tutti i blocchi dopo quello 
manipolato in caso di attacco.
Confronto algoritmi AE:
\begin{itemize}
    \item GCM (Galois/Counter Mode): il più popolare, on-line (l'algoritmo agisce sull'input al momento senza averlo tutto)
    single-pass AEAD, parallelizzabile
    \item OCB 2.0 (Offset Codebook Mode): il più veloce, on-line single-pass AEAD
    \item EAX (Encrypt then Authenticate then X(trans)late): on-line double-pass AEAD, lento ma piccolo
    \item CCM (CTR mode with CBC-MAC): off-line double pass, il più lento
\end{itemize}
\paragraph*{AEAD (Authenticated Encryption with Associated Data)}
Schema di AE che contiene dati associati non confidenziali, la cui integrità è protetta. Usata nell'header dei pacchetti rete.
\subsubsection{Autenticazione tramite digest e cifratura asimmetrica}
Si invia anche un digest, cifrato con la chiave privata del mittente. Solo chi conosce la chiave pubblica può confrontare 
il digest trasmesso con quello calcolato sui dati ricevuti.
\subsubsection{Firme RSA e funzioni hash}
Una funzione di hash da usarsi in uno schema RSA deve essere resistente alle collisioni e difficile da invertire (quindi 
da falsificare).
\subsubsection{PKCS}
Sono un gruppo di standard. PKCS \#1 definisce le primitive per l'uso di RSA: conversione e rappresentazione di grandi 
numeri, algoritmi base di cifratura/decifratura, algoritmi base di firma e verifica. Queste primitive devono essere usate 
per creare uno schema crittografico sicuro.
\begin{itemize}
    \item schemi di cifratura/decifratura: RSAES-OAEP, RSAES-PKCS 
    \item schemi di firma/verifica (con appendice):
\end{itemize}
Gli schemi di firma sono detti con appendice perché non si cifrano i dati ma un loro riassunto (hash), RSA tratta direttamente 
solo dati di dimensione minore del modulo pubblico $n$.
\subsubsection{Autenticazione e integrità: analisi}
Tramite segreto condiviso:
\begin{itemize}
    \item utile solo per il ricevente
    \item non usabile come prova senza rivelare la chiave segreta
    \item non usabile per il non ripudio
\end{itemize}
Tramite crittografia asimmetrica:
\begin{itemize}
    \item essendo lenta la si applica solo al digest 
    \item usabile come prova formale
    \item usabile per il non ripudio
    \item equivale alla firma digitale
\end{itemize}
Con una sola chiave privata è possibile generare infinite firme digitali.
\subsection{Prestazioni crittografiche}
Le \href{https://www.cse.wustl.edu/~jain/cse567-06/ftp/encryption_perf/}{prestazioni} dipendono dalla CPU e dalla sua cache. Non sono un problema sul client, ma potrebbero esserlo sul server,
ovviato con acceleratori crittografici.
L'algoritmo più veloce è RC4, seguito da AES-128-CBC, DES-CBC e DES-EDE3-CBC. RSA 1024 è più lento di RSA 2048.
\begin{enumerate}
    \item lunghezza chiavi crittografia simmetrica: 80, hash: 160, lunghezza chiavi crittografia asimmetrica: 1248, 
    ECC: 160 = attaccabile in poco tempo 
    \item lunghezza chiavi crittografia simmetrica: 96, hash: 192, lunghezza chiavi crittografia asimmetrica: 1776, 
    ECC: 192 = legacy
    \item lunghezza chiavi crittografia simmetrica: 112, hash: 224, lunghezza chiavi crittografia asimmetrica: 2432, 
    ECC: 256 = attaccabile a medio termine 
    \item lunghezza chiavi crittografia simmetrica: 128, hash: 256, lunghezza chiavi crittografia asimmetrica: 3248, 
    ECC: 160 = sicurezza a lungo termine
    \item lunghezza chiavi crittografia simmetrica: 256, hash: 512, lunghezza chiavi crittografia asimmetrica: 15424, 
    ECC: 512 = per ora inattaccabile senza computer quantistici
\end{enumerate}
\section{Certificati}
È una struttura dati per legare in modo sicuro una chiave pubblica ad alcuni attributi. Tipicamente lega chiave a identità,
è firmato in modo elettronico dall'emettitore, detto autorità di certificazione (CA). Ha una scadenza e può essere revocato.

PKI: infrstruttura tecnica e organizzativa preposta alla creazione, distribuzione e revoca dei certificati a chiave pubblica.
È composta da:
\begin{itemize}
    \item end entity: utente finale, umano o non umano 
    \item CA: autorità fidata da uno o più utenti che crea e assegna certificati e CRL, firmandoli digitalmente
    \item RA: componente opzionale che può essere utilizzata per alleggerire il carico di lavoro del CA, tipo verificare 
    l'identità di una end entity 
    \item repository: denota ogni metodo utilizzato per memorizzare i certificati e le CRL 
    \item relying party: ogni entità che dipende dai dati in un certificato per prendere decisioni 
\end{itemize}
\begin{center}
    \includegraphics[width=\textwidth]{PKI.png}
\end{center}
\subsection{Certificati X.509}
Struttura: versione, numero di serie, algoritmo per la firma, richiedente, lasso di validità, soggetto, informazioni chiave
pubblica, firma digitale.

X.509 si basa sull'uso di crittografia a chiave pubblica e firme digitali. Lo standard non detta l'uso d uno specifico 
algotitmo. Le informazioni contenute nel certificato sono firmate calcolando il valore di hash delle informazioni stesse 
e generando una firma digitale utilizzando il valore di hash e la chiave privata della CA. Il certificato può poi essere  
distribuito.

Tutti gli utenti che sono a conoscenza della chiave pubblica del CA possono verificare la chiave pubblica dell'utente certificata.
Nessuno oltre al CA può modificare il certificato senza essere scoperto.
\subsection{Revoca dei certificati}
Un certificato può essere revocato prima della sua scadenza naturale, su richiesta dei titolare o automaticamente dall'emettitore.
Quando si valida una firma si deve verificare che il documento fosse valido all'atto della firma. La varifica è a carico 
del ricevente, il relying party (RA).
Meccanismi di revoca:
\begin{itemize}
    \item CRL (Certificate Revocation List): elenco dei certificati revocati, firmato dalla CA o da un delegato
    \item OCSP (On-line Certificate Status Protocol): risposta puntuale su un singolo certificato, firmato dal server 
\end{itemize}
Struttura CRL X.509: versione, algoritmo di firma, richiedente, data di aggiornamento, elenco date revoche certificati, 
firma digitale del CA.
Le CRL sono firmate dalla CA che ha emesso i certificati e da una revokation authority delegata dalla CA.
\subsubsection{X.509 versione 3}
Raccoglie in un unico documento le modifiche necessarie a estendere le definizioni dei certificati e delle CRL. Esistono 
estensioni di due tipi: pubbliche, ossia definite nello standard e quindi note a tutti, e private, uniche per una certa 
comunità di utenti. Le estensioni sono aggiunte al formato del certificato.

Un'estensione può essere definita critica o non critica. Nel processo di verifica devono essere rifiutati i certificati 
che contengono un'estensione critica non riconosciuta; un'estensione non critica può essere ignorata se sconosciuta. Il
differente trattamento è a carico di chi effettua la verifica: il Relying Party (RP).
X.509v3 definisce 4 categorie di estensioni:
\begin{itemize}
    \item key and policy information
    \item crtificate subject and certificate issuer attributes
    \item certificate path constraints
    \item CRL distribution points 
\end{itemize}
\paragraph*{Key and policy information}
Key usage: identifica lo spazio delle applicazioni pper il quale la chiave pubblica può essere usata, può essere critica 
o non critica, se è critica allora il certificato può essere usato solo per gli scopi la cui corrispondente opzione è definita.
\paragraph*{Certificate subject and certificate issuer attributes}
Subject alternative name: consente di usare diversi formalismo per identificare il possessore del certificato, critica se 
il campo subject-name è vuoto.
\paragraph*{Certificate path constraints}
Basic constraints: indica se il soggetto del certificato può agire da autorità di certificazione, è possibile definire la 
massima profondità dell'albero di certificazione, si consiglia di definirla come critica.

Name constraints: solo per CA, fissa lo spazio dei nomi certificabili da una CA, specifica whitelist/blacklist.
\paragraph*{CRL distribution points}
Identifica il punto di distribuzione della CRL da usare come verifica della validità di un certificato.
\paragraph*{Estensioni private}
È possibile definire estensioni private, comuni a una certa comunità di utenti. Tra queste, PKIX Authotity Information Access:
indica come accedere a informazioni e servizi della CA che ha emesso il certificato.
\subsubsection{OCSP}
Standard IETF-PKIX per verificare in linea se un certificato è valido (non revocato), è un'alternativa alle CRL. 
Le risposte sono firmate dal server (non dalla CA) e il certificato del server non è verificabile con OCSP.
Sono possibili risposte precalcolate, che diminuiscono carico sul server ma rendono possibili attacchi replay.
Funzionamento:
\begin{enumerate}
    \item A e B hanno dei certificati rilasciati dalla CA 
    \item A vuole connettersi con B e gli invia il certificato 
    \item B crea una richiesta OCSP che contiene il numero di serie del certificato e lo invia alla CA 
    \item il responder della CA controlla lo status del certificato nel suo database
    \item se il certificato è valido, conferma la validità inviando una risposta a B 
    \item B verifica la risposta della CA utilizzando la chiave pubblica della CA 
    \item B si connette con A 
\end{enumerate} 
\paragraph*{Trusted Responder}
Il server OCSP firma le risposte con una coppia chiave/certificato indipendente dalla CA per la quale sta rispondendo. Il 
servizio è pagato dall'azienda o dagli utenti.
\paragraph*{Delegated Responder} 
Il server OCSP firma le risposte con una coppia chiave/certificato diversa in base alla CA per la quale sta rispondendo. Il 
servizio è pagato dalla CA.
\subsection{Applicazioni relative all'uso e memorizzazione dei certificati}
\subsubsection{Timestamping}
Prova della creazione dei dati prima di un certo istante di tempo. Regolata dalla TSA (Time-Stamping Authority). RFC-3161
descrive il protocollo di richiesta e il formato della prova.
Funzionamento:
\begin{enumerate}
    \item l'hash dei dati originali viene calcolato, a esso viene aggiunto il timestamp dato dal TSA e il suo risultato 
    (hash A) viene calcolato
    \item la firma del TSA viene verificata decifrandola con la chiave pubblica del TSA, producendo hash B 
    \item hash A e B vengono confrontati 
\end{enumerate}
\subsubsection{PSE (Personal Security Environment)}
Ogni utente dovrebbe proteggere la propria chiave privata e i certificati delle root CA fidate. Può essere sw o hw.
\paragraph*{Smart-card crittografiche}
Carte a chip a memoria e con capacità crittografiche autonome. È semplice realizzare card in grado di svolgere crittografia 
simmetrica e complesso con quella asimmetrica. Dispongono di poca memoria.
\paragraph*{HSM (HW Security Module)}
Acceleratore crittografico per server, ha memoria protetta e capacità crittografiche autonome.
\paragraph*{API di sicurezza}
PKCS \#11 descrive le API per creare e manipolare i token crittografici. Può essere utilizzato in hw e sw.
\subsection{Formati per documenti elettronici sicuri}
\subsubsection{PKCS \#7 e CMS}
PKCS \#7 è lo standard RSA per la busta sicura (posta elettronica), CMS è la sua evoluzione. Permette autenticazione. integrità, riservatezza 
dei dati con algoritmi simmetrici o asimmetrici. Permette più firme su uno stesso oggetto e può includere certificati per 
revocare la firma. È un formato ricorsivo.
Gli algoritmi base sono:
\begin{itemize}
    \item digest: MD5, SHA-1
    \item firma: RSA, DSA 
    \item key management: DH, RSA, 3DES e PBKDF2
    \item cifratura contenuto: 3DES-CBC, RC2-CBC
    \item MAC: HMAC-SHA1
\end{itemize}
\subsubsection{Struttura CMS}
CMS racchiude i due campi content type e content in contentInfo.
\paragraph*{Tipi di contentType}
\begin{itemize}
    \item data: codifica di una generica sequenza di byte 
    \item signedData: dati e firme digitali parallele 
    \item envelopedData: dati cifrati e chiave cifrata per i destinatari 
    \item authenticatedData: dati, MAC e chiave cifrata per i destinatari 
    \item digestedData: dati e digest 
    \item encryptedData: dati cifrati con algoritmo simmetrico
\end{itemize}
\subsubsection{PKCS \#10 (Certificate Signing Request)}
Descrive il formato per la richiesta di un certificato, (CSR). La richiesta contiene: DN, chiave pubblica e attributi.
\subsubsection{PKCS \# 12}
Definisce un formato archivio per memorizzare oggetti crittografici in un solo file, consentendone il trasporto. Trasporta 
una chiave privata e uno o più certificati, quindi l'identità digitale di un utente.
\subsubsection{Documenti firmati}
Un documento firmato può:
\begin{enumerate}
    \item essere avvolto nella firma
    \item avvolgere la firma 
    \item avere una firma separata 
\end{enumerate}
É possibile avere firme multiple. 
\subsubsection{Electronic Signature (ES) Europea}
Dati che sono logicamente associati con altri dati in formato elettronico che ne forniscono un mezzo di autenticazione.
Una firma scannerizzata è una firma elettronica. È un concetto legale distinto dalle firme digitali.
\paragraph*{AES (Advanced Electronic Signature)}
È una ES che soddisfa i seguenti requisiti:
\begin{itemize}
    \item è in relazione univoca con il firmatario
    \item consente di identificare il firmatario 
    \item è creata usando strumenti che il firmatario può mantenere sotto il suo controllo 
    \item è in relazione con i dati ai quali si riferisce in modo che ogni successiva modifica dei dati possa essere individuata 
\end{itemize}
\subsubsection{Qualified Certificate (QC)}
Certificato che garantisce l'identità di una persona. Contiene:
\begin{itemize}
    \item l'indicazione che si tratta di un QC 
    \item l'indicazione del certificatore e dello stato in cui è stato emesso 
    \item indicazioni sulle limitazioni di utilizzo del certificato 
    \item indicazioni sul limite delle transazioni commerciali effettuabili con quel certificato 
\end{itemize}
\paragraph*{Qualified Electronic Signature}
È una AES apposta usando QC e dispositivi di firma sicuri. Ha valore legale equivalente alla firma autografa. 
\subsubsection{Standard ETSI per firma elettronica}
Si chiama CAdES (CMS Advanced Electronic Signature), è un formato di firma grezzo. Esistono anche XAdES e PAdES (per formati XML e PDF).
ASiC (Associated Signature Containers) sono contenitori per associare documenti elettronici con firme detached.
\section{Attacchi alle reti IP}
IP non ha alcuna autenticazione degli indirizzi e i pacchetti non sono protetti. Sono quindi attaccabili tutti i protocolli 
che utilizzano IP come trasporto.
\subsection{Protezione DHCP}
DHCP è un protocollo non autenticato, debole ad attacchi shadow server, che può sferrare DoS e MITM logici.
Alcuni switch offrono DHCP snooping, che consentono risposte solo da porte affidabili, e IP guard che ammetono solo IP 
ottenuti da DHCP. Si può anche usare HMAC-MD5 per autenticare i messaggi, ma è scarsamente adottato.
\subsection{Sicurezza ICMP}
è un protocollo di servizio per reti a pacchetto che si occupa di trasmettere informazioni riguardanti malfunzionamenti, 
informazioni di controllo o messaggi tra i vari componenti di una rete. Privo di autenticazione, soggetto a smurfing e fraggle.
\paragraph*{Smurfing}
Attacco DDoS durante il quale un gran numero di pacchetti ICMP vengono inviati in broadcast su una rete. I dispositivi 
sulla rete rispondono all'IP sorgente, che l'attaccante ha designato come IP della vittima.

Per contrastare l'attacco, basta rifiutare il broadcast IP o identificare il responsabile con strumenti di network management.
\paragraph*{Fraggle}
Come smurfing ma con pacchetti UDP su porte 7 e 19.
\subsection{ARP poisoning}
ARP è utilizzato per scoprire l'indirizzo MAC di un nodo del quale si conosce l'indirizzo IP. Il risultato è memorizzato 
nella tabella ARP. Con ARP poisoning si inseriscono dati falsi nella tabella per inviare dati ad altri dispositivi.
\subsection{TCP SYN flooding}
Vengono inviate multiple richieste con IP spoofing, saturando la tabella delle connessioni di rete fino a quando non vanno
in timeout.

Per difendersi:
\begin{itemize}
    \item abbassare il timeout, rischiando di eliminare client validi ma lenti
    \item aumentare le dimensioni della tabella, aggirabile con più richieste 
    \item usare un router come intercettatore o come controllore che uccide i collegamenti pendenti
    \item SYN cookie, usa il numero di sequenza del pacchetto SYN-ACK per trasmettere cookie al client e riconoscere così 
    i client che hanno già inviato il SYN senza memorizzare nulla sul server
\end{itemize}
\subsection{Sicurezza DNS}
\paragraph*{DNS shadow server}
Fa sniffing per intercettare le query e spoofing per generare risposte false.
\paragraph*{DNS cache poisoning}
La vittima fa query sul NS compromesso, che risponde inviando risposte anche a query non effettuate per sovrascrivere la 
cache della vittima; oppure fornire risposta falsa per inserirla nella cache della vittima.
\paragraph*{DNS flash crowd}
Flooding di richieste su una vittima.
\subsection{DNSsec}
Fornisce protezione end-to-end tramite firme digitali create dagli amministratori dei server DNS e verificate dal software 
di risoluzione del ricevente. Evita di doversi fidare dei NS intermedi.

Le query stesse non sono firmate, e non esiste una root CA, non fornisce siurezza nel dialogo tra DNS client e server.
\subsection{Sicurezza del routing}
Poca sicurezza nell'accesso sistemistico ai router per la gestione e nello scmbio di tabelle di routing. È possibile attivare
protezione con keyed-digest.
\subsection{Protezione IP spoofing}
Per proteggere dagli impostori interni e esterni. Si utilizzano filtri presenti nel router.
\section{VPN}
È una tecnica hw o sw per realizzare una rete privata utilizzando canali e apparati di trasmissione non affidabili. 
\paragraph*{VPN tramite rete nascosta}
Indirizzamento non standard per non essere raggiungibili da altre reti. Aggirabile se qualcuno scopre gli indirizzi usati,
vulnerabile a sniffing.
\paragraph*{VPN mediante tunnel}
I router provvedono a incapsulare i pacchetti di rete all'interno di altri pacchetti. I router controllano l'accesso alle 
reti tramite ACL (Access Control List).
Se il pacchetto da trasmettere supera la massima dimensione consentita, deve essere frammentato.
\paragraph*{VPN tramite tunnel IP sicuro}
Prima di essere incapsulati i pacchetti di rete vengono protetti con MAC (integrità e autenticazione), cifratura (riservatezza)
e numerazione (contro replay). Anche detta Secure VPN.
\subsection{IPsec}
Architettura IETF per fare sicurezza al livello 3 sia in IPv4 che in IPv6. Permette di creare VPN su reti non dfidate e 
di fare sicurezza end-to-end.

Definisce due formati particolari:
\begin{itemize}
    \item AH (Authentication Header): per integrità, autenticazione, protezione da replay
    \item ESP (Encapsulating Security Payload): per riservatezza 
\end{itemize}
Usa il protocollo IKE (Internet Key Exchange) per lo scambio delle chiavi.
\subsubsection{IPsec Security Association}
Connessione logica unidirezionale protetta tra due sistemi IPsec. A ogni SA sono associabili caratteristiche di sicurezza 
diverse, occorrono due SA per avere protezione completa in un canale bidirezionale.
\subsubsection{Database locali IPsec}
\begin{itemize}
    \item SPD (Security Policy Database): contiene le security policy da applicare ai diversi tipi di comunicazione, configurato 
    a priori oppure agganciato a un sistema automatico 
    \item SAD (SA Database): elenco delle SA attive e delle loro caratteristiche
\end{itemize}
\subsubsection{Funzionamento di IPsec}
\begin{enumerate}
    \item il modulo IPsec riceve pacchetto IP e controlla quale policy applicare consultando la SPD 
    \item ottiene le regole di sicurezza e crea/legge SA 
    \item ottiene algoritmi e paramtretri dalla SAD 
    \item il pacchetto IP è ora protetto da IPsec
\end{enumerate}
\paragraph*{IPsec in transport mode}
Usato per fare sicurezza end-to-end, ossia usato dagli host, non dai gateway. È computazionalmente leggero ma non protegge 
i campi variabili.
\paragraph*{IPsec in tunnel mode}
Usato per fare VPN, solitamente dai gateway. Protegge i campi variabili.
\subsubsection{AH}
Fornisce integrità (tramite funzione di hash) e autenticazione dei dati, protezione da replay attack (scartando pacchetti 
vecchi). Utilizza HMAC. Opera sopra IP.
\paragraph*{Processamento di un pacchetto con AH}
\begin{enumerate}
    \item in parallelo, si estrae AH e si normalizza il pacchetto 
    \item si estrae ICV (dati di autenticazione) da AH
    \item si estrae SPI (Security Parameters Index) da AH e lo si cerca nella SAD
    \item si calcola il valore di autenticazione tramite algoritmi e parametri della SAD e pacchetto IP nornalizzato 
    \item si confrontano i valori ricevuti da ICV e quelli calcolati per vedere se corrispondono
\end{enumerate}
\paragraph*{Normalizzazione per AH}
\begin{itemize}
    \item azzerare campo TTL/hop limit 
    \item se il pacchetto contiene Routing Header: fissare il campo di destinazione all'indirizzo del destinatario finale, 
    fissare il contenuto del RH e Address Index al valore che avrà a destinazione
    \item azzerare tutte le operazioni che hanno bit C attivo 
\end{itemize}
\paragraph*{Funzionamento HMAC-SHA1-96}
\begin{enumerate}
    \item dato $M$ normalizzarlo generando $M'$
    \item allineare a 160 bit $M'$ generando $M'p$
    \item calcolare la base di autenticazione B=HMAC-SHA1$(K,M'p)$
    \item ICV: 96 leftmost bit di $B$
\end{enumerate}
\subsubsection{ESP}
Usa come meccanismo base DES-CBC, fornisce integrità e autenticazione, riduce la dimensione del pacchetto e risparmia una 
SA.
\paragraph*{ESP in transport mode}
Usato dagli host, non dai gateway, non nasconde l'header.
\paragraph*{ESP in tunnel mode}
Usato solitamente dai gateway, nasconde anche gli header.
\begin{center}
    \includegraphics[width=\textwidth]{AHvsESP.png}
\end{center}
\subsubsection{Protezione da replay in IPsec}
\begin{enumerate}
    \item quando si crea una SA, il mittente inizializa il sequence number a 0
    \item quando si invia un pacchetto, si incrementa il sequence number 
    \item quando si raggiunge il sequence number $2^{32}-1$ si negozia un nuova SA
\end{enumerate}
Detta anche sliding window tecnique.
\subsubsection{IPsec v3}
AH è opzionale, ESP è obbligatorio. Supporto per multicast da singola sorgente e AEAD. Possiede ESN (Extended Sequence Number).
\subsection{Comparazione metodi di sicurezza}
\paragraph*{End-to-end security}
IPsec è adottato a livello host tramite transport-mode SA. È computazionalmente leggero ma non protegge i campi variabili.
\paragraph*{Basic VPN}
Ipsec adottato a livello gateway, tunnel-mode SA. Protegge i campi variabili.
\paragraph*{End-to-end security con VPN}
Utilizza entrambe le sopra citate.
\paragraph*{Secure gateway}
Il primo host si connette direttamente alla WAN, il gateway che riceve il pacchetto indirizzato al secondo host è protetto 
da IPsec, solo tunnel-mode SA.
\paragraph*{Secure remote access}
Come sopra, ma entrambi gli host utilizzano IPsec, quindi sia transport-mode che tunnel-mode SA.
\subsubsection{IPsec key management}
Componente fondamentale per IPsec, fornisce ai sistemi IPsec comunicanti le chiavi simmetriche necessarie per l'autenticazione 
e/o la cifratura dei pacchetti. Le chiavi vengono distribuite OOB o automaticamente.
\paragraph*{ISAKMP}
Internet Security Association and Key Management Protocol, definisce le procedure necessarie per negoziare, stabilire, modificare 
e cancellare la SA. Non indica il metodo da usare per lo scambio delle chiavi. Lo scambio è realizzato solitamente dal 
protocollo OAKLEY.
\paragraph*{IKE}
Internet Key Exchange, crea una SA per proteggere lo scambio ISAKMP. Con questa SA protegge la negoziazione della SA richiesta 
da IPsec, può essere riutilizzata più volte per negoziare altre SA IPsec.
Modi di funzionamento:
\begin{itemize}
    \item main mode: 6 messaggi, protegge l'identità delle parti 
    \item aggressive mode: 3 messaggi, non protegge identità
    \item quick mode: 3 messaggi, negoziazione solo della SA IPsec 
    \item new group mode: 2 messaggi 
\end{itemize}
Metodi di autenticazione: 
\begin{itemize}
    \item digital signature: non repudiation della negoziazione IKE 
    \item PKE: protezione dell'identità in aggressive mode 
    \item revised PKE: meno costoso, solo 2 operazioni a chiave pubblica 
    \item pre-shared key: l'ID della contropart può essere solo il suo indirizzo IP 
\end{itemize}
\subsubsection{VPN concentrator}
Apparecchiature special-purpose che fungono da terminatori di tunnel IPsec. Per accesso remoto di singoli client oppure 
per creare VPN site-to-site. Prestazioni elevate.
\subsection{Prestazioni IPsec}
\begin{itemize}
    \item router: CPU potente o acceleratore crittografico, non gestito in outsourcing 
    \item firewall: CPU potente 
    \item VPN concentrator: massima indipendenza dalle altre misure di sicurezza 
\end{itemize}
IPsec riduce il throughput di rete, perché i pacchetti hanno dimensione maggiore e si ha un maggior numero di pacchetti.

IPsec è applicabile solo a pacchetti unicast, tra parti che hanno attivato la SA tramite chiavi condivise o certificati.
\section{Firewall e IDS/IPS}
\subsection{Firewall}
\begin{itemize}
    \item ingress firewall: collegamenti incoming, tipicamente per selezionare i servizi offerti dall'esterno
    \item egress firewall: collegamenti outgoing, controllo dell'attività personale
\end{itemize}
Un firewall deve garantire un buon equilibrio tra sicurezza e funzionalità, puntando sulla sicurezza.
Principi inderogabili:
\begin{enumerate}
    \item il FW deve essere l'unico punto di contatto della rete interna con quella esterna 
    \item solo il traffico autorizzato può attraversare il FW 
    \item il FW deve essere un sistema altamente sicuro esso stesso
\end{enumerate}
\paragraph*{Whitelisting}
Tutto ciò che non è espressamente permesso è vietato. Fornisce maggiore sicurezza ma è più dfficile da gestire.
\paragraph*{Blacklisting}
Tutto ciò che non è espressamente vietato è permesso. Minore sicurezza rispetto al whitelisign, più facile da gestire.
\subsection{Tecnologie di FW}
\subsubsection{Packet filter}
Storicamente disponibile sui roouter, effettua controlli sui soongoli pacchetti IP.

Pro:
\begin{itemize}
    \item basso costo 
    \item ottima scalabilità e prestazioni 
\end{itemize} 
Contro:
\begin{itemize}
    \item controlli pocco precisi, quindi più facile da ingannare 
    \item arduo supportare servizi con porte allocate dinamicamente 
    \item configurazione complessa 
    \item difficile fare autenticazione degli utenti 
\end{itemize}
\subsubsection{Stateful packet filter}
Simile al packet filter ma state-aware.  Riceve informazioni di stato dal livello trasporto e quello applicativo, distingue 
le nuove connessioni da quelle già aperte.

Migliori prestazioni rispetto al packet filter, ma comunque limitato.
\subsubsection{Circuit-level gateway}
FW non application-aware. Crea un circuito tra client e server a livello trasporto, ma non comprende i dati in trasporto,
si limita a copiare tra le sue interfacce i segmenti TCP o i datagrammi UDP (se rispettano le regole di controllo accessi),
deve riassemblare pacchetti IP quindi protegge da alcuni attacchi L3/L4.

Rompe il modello client server per una specifica connessione:
\begin{itemize}
    \item i server sono più protetti: isola da attacchi che riguardano handshake TCP e frammentazione pacchetti IP 
    \item può autenticare il client 
\end{itemize}
Rimangono limitazioni proprie al packet filter.
\subsubsection{Application-level gateway}
Composto da una serie di proxy che esaminano il contenuto dei pacchetti a livello applicativo. Può effettuare il mascheramento 
o rinumerazione degli indirizzi IP interni e funzioni di autenticazione. 

Pro:
\begin{itemize}
    \item regole più granulari rispetto a packet filter 
    \item server più protetti 
    \item può autenticare il client 
\end{itemize}
Contro:
\begin{itemize}
    \item ritardo nel supporto per le nuove applicazioni
    \item consuma più risorse ed è più lento 
    \item mancanza di trasparenza per i client 
    \item può esporre il SO del FW ad attacchi 
\end{itemize}
Varianti: 
\begin{itemize}
    \item trasparent proxy: meno intrusivo per i client, più complesso 
    \item strong application proxy: solo comandi/dati sono trasmessi
\end{itemize}
\subsubsection{HTTP proxy}
Server HTTP che fa solo da front-end e poi passa le richieste a un server esterno.
Pro:
\begin{itemize}
    \item cache delle pagine esterne per tutti gli utenti interni 
    \item autenticazione e autorizzazione degli utenti interni 
    \item possibili vari controlli 
\end{itemize}
\subsubsection{HTTP reverse proxy}
Server HTTP che fa solo da front-end e poi passa le richieste a un server interno.
Pro:
\begin{itemize}
    \item obfuscation del server 
    \item acceleratore SSL 
    \item load balancer 
    \item web accelerator, cache di contenuti statici 
    \item compressione 
    \item spoon feeding: riceve dal server tutta una pagina creata dinamicamente e la serve poco alla volta al client 
\end{itemize}
\subsubsection{WAF (Web Application Firewall)}
Modulo installato su proxy per filtrare il traffico applicativo, utilizza HTTP.
\subsubsection{Local FW}
Firewall installato direttamente sul nodo da difendere. È tipicamente un packet filter.

Rispetto a un normale FW in rete può controllare i processi a cui è permesso aprire collegamenti verso altri nodi e ricevere 
richieste di collegamento.
\subsection{Architetture di FW}
\subsubsection{Screening router (choke)}
Router che filtra il traffico di rete, sia su livello IP che superiore. È economico ma insicuro. 
\subsubsection{Application gateway (proxy)}
Servizio che svolge il lavoro per conto di un applicativo, tipicamente con controllo d'accesso.
\subsubsection{Dual-homed gateway}
Sistema con due connessioni di rete e routing disabilitato. Facile da realizzare, è possibile mascherare la reten interna,
ma richiede più hardware rispetto a screening router ed è poco flessibile.
\subsubsection{Screened host gateway}
Il router blocca traffico da interno a esterno e viceversa tranne se arriva dal o al bastion. Il bastion host ospita circuit
o application gateway per controllare i servizi autorizzati.
Pro:
\begin{itemize}
    \item maggiore flessibilità
\end{itemize}
Contro:
\begin{itemize}
    \item si possono mascherare solo gli host/protocolli che passano dal bastion 
    \item è più costoso e complesso da gestire 
    \item ricompare il problema del sigle-point-of-failure
\end{itemize}
\subsubsection{Screened subnet}
Uno o più screening router sono utilizzati come firewall per definire tre diverse sottoreti: un router esterno separa la 
rete esterna dalla rete di perimetro, e un router interno separa la rete di perimetro da quella interna. La rete di perimetro 
è anche detta DMZ, e ospita i server che sono accessibili sia dalla rete interna che da quella esterna.
Si può configurare il roouting in modo che la rete interna sia sconosciuta.

Per motivi di costo e di semplicità di gestione spesso si omettono i router, incorporando le loro funzioni nel gateway.
Anche noto come FW a tre gambe.
\subsection{IDS e IPS}
\subsubsection{IDS (Intrusion Detection System)}
È un sistema per identificare individui che utilizzano un computer o una rete senza autorizzazione, esteso anche all'identificazione 
di utenti autorizzati ma che violano i loro privilegi. Si basa sul fatto che il pattern di comportamento degli utenti non
autorizzati si differenzia da quello degli utenti autorizzati.

Gli IDS passivi utilizzano checksum crittografici e riconosciemnto di pattern, mentre quelli attivi fanno learning, monitoring 
e reaction.

Caratteristiche topologiche:
\begin{itemize}
    \item HIDS (Host-Based IDS): analisi dei log e attivazione di strumenti di monitoraggio interni al SO
    \item NIDS (Network-Based IDS): attivazione di strumenti di monitoraggio del traffico di rete 
\end{itemize}
\subsubsection{SIV E LFM}
\begin{itemize}
    \item SIV: controlla i file di un nodo per rilevarne cambiamenti, ad esempio file di registro
    \item LFM: controlla i file dei log e rileva pattern d'attacco 
\end{itemize}
\subsubsection{NIDS}
Componenti: 
\begin{itemize}
    \item sensor: controlla traffico e log, attiva i security event rilevanti e interagisce con il sistema 
    \item director: coordina i sensor e gestisce il security database 
    \item IDS message system: consente la comunicazione sicura e affidabile tra i componenti dell'IDS
\end{itemize}
\subsubsection{IPS}
Per velocizzare e automatizzare la risposta alle intrusioni, è IDS e FW dinamico distribuito. Non un prodotto ma una tecnologia.
\section{Access control e audit}
\subsection{Access control}
Controllo di accesso: processo di concessione o negazione di richieste specifiche per ottenere e utilizzare informazioni 
e i relativi servizi di elaborazione delle informazioni, e accedere a strutture specifiche.
O anche processo mediante il quale l'uso delle risorse è regolato in base a una politica di sicurezza ed è consentito solo 
da entità autorizzate in base a tale politica. 

Un meccanismo di controllo dell'accesso avviene tra un utente e le risorse di sistema. Il sistema deve prima autenticare 
un'entità in cerca d'accesso.

L'access control dovrebbe permettere di specificare: chi è autorizzato, a che risorse, per quanto tempo, in quali giorni,
con che modalità, eseguendo quale operazione.
\subsubsection{Subjects, objects, actions}
\begin{itemize}
    \item un soggetto è un'entità in grado di accedere agli oggetti 
    \item un oggetto è una risorsa a cui è controllato l'accesso 
    \item un'azione descrive il modo e operazione con cui un soggetto può accedere a un oggetto 
\end{itemize}
\paragraph*{Subjects}
Generalmente, il concetto di soggetto si identifica con quello di processo. Qualsiasi utente o applicazione ottiene effettivamente 
l'accesso a un oggetto tramite un processo che rappresenta tale utente o applicazione. Il processo assume gli attributi 
dell'utente, come i diritti d'accesso. Un soggetto è in genere ritenuto responsabile delle azioni che ha avviato e lascia 
una traccia di controllo.
\paragraph*{Objects}
Entità utilizzata per contenere e/o ricevere informazioni. Ad esempio: record, file, pagine, directory, ma anche bit, porte 
di comunicazione, ecc\dots

Il numero e tipo di oggetti da proteggere da un sistema di controllo degli accessi dipende dall'ambiente in cu opera e dal 
compromesso desiderato tra sicurezza e complessità.
\paragraph*{Azioni}
Azioni tipo: read, write, execute, delete, create, search.
\subsubsection{Strategie di access control}
\paragraph*{DAC (Discretionary Access Control)}
Controlla l'accesso in base all'identità del richiedente e alle regole d'accesso (autorizzazioni) che indicano ciò che i
richiedenti sono o non sono autorizzati a fare. È detta discrezionale perché un'entità potrebbe avere diritti d'accesso 
che consentono all'entità, di sua spontanea volontà, di consentire a un'altra entità di accedere a qualche risorsa.

Il proprietario del sistema può accedere liberamente alle risorse, e stabilisce quali utenti e gruppi possono accedervi.
\begin{center}
    \includegraphics[width=\textwidth]{DAC.png}
\end{center}
\paragraph*{MAC (Mandatory Access Control)}
Controlla l'accesso in base al confronto delle etichette di sicurezza, che indicano quanto sono sensibili o critiche le 
risorse di sistema, con le autorizzazioni di sicurezza, che indicano che le entità di sistema sono idonee ad accedere a 
determinate risorse. È detta obbligatoria perché un'entità che dispone di autorizzazione per accedere a una risorsa non 
può, di sua spontanea volontà, consentire a un'altra entità di accedere a quella risorsa. 
\paragraph*{RBAC (Role-Based Access Control)}
Controlla l'accesso in base aii ruoli che gli utenti hanno all'interno del sistema e sulle regole che stabiliscono a 
quali accessi sono consentiti da utenti in determinati ruoli.

Più efficiente di DAC, la dimensione della access matrix è in base ai ruoli e non agli utenti.
\begin{center}
    \includegraphics[width=\textwidth]{RBAC.png}
\end{center}
\paragraph*{ABAC (Attribute-Based Access Control)}
Controlla l'accesso in base agli attributi dell'utente, la risorsa a cui accedere e le condizioni dell'ambiente attuale.
\subsection{Audit}
Il system audit è un'attività ordinaria per valutare le prestazioni del sistema, i controlli di sicurezza, ecc\dots 

Il monitoring è un'attività ordinaria che tiene traccia di tutte le attività eseguite sul sistema, come il rilevamento delle 
intrusioni e altre.

Auding e monitoring sono le capacità di osservare gli eventi, comprendere le prestazioni e mantenere l'integrità del sistema.
Queste operazioni vengono spesso eseguite registrando gli eventi nel file di registro ed esaminando tali file in un secondo 
momento (logging).
\subsubsection{Internal e external audit}
\paragraph*{Internal audit}
Viene eseguito dai revisiori all'interno dell'organizzazione. Scopi: identificare i rischi ogni volta che si presentano 
in relazione a problemi di prestazioni, sicurezza e conformità. Inoltre tengono d'occhio ciò che viene fatto per mitigare 
questi problemi con l'obiettivo di aiutare le organizzazioni a funzionare meglio.
\paragraph*{External audit}
Svolto da revisori all'esterno dell'organizzazione. Sono enti indipendenti, spesso contabili pubblici certificati.
\begin{center}
    \includegraphics[width=\textwidth]{external-internal audit.png}
\end{center}
\section{Tecniche e sistemi di autenticazione}
\subsection{Autenticazione}
\paragraph*{RFC-4949}
È il processo di verificare l'affermazione che un sistema o una risorsa abbia un determinato valore. Affermazione è coppia 
attrivuto valore, come username + password.
\paragraph*{whatis.com}
Il processo di determinare che qualcuno o qualoca sia chi o cosa dica di essere. Coinvolge sia utenti che processi/dispositivi.
\paragraph*{NIST IR 7298}
Verificare l'identità di un utente, processo, o dispositivo, spesso come prerequisito per accedere a risorse o infomazioni 
in un sistema informativo.

Si autorizzano attori, come esseri umani, software o hardware (authN). Autorizzazione (cosa posso fare) != autenticazione.
\subsubsection{Fattori di autenticazione}
\paragraph*{Conoscenza}
Qualcosa che solo l'utente conosce. Rischio: memorizzazione e dimostrazione/trasmissione.
\paragraph*{Possesso}
Qualcosa che l'utente possiede.  Rischi: l'autenticatore, per furto, clonazione, uso non autorizzato.
\paragraph*{Essenza}
Qualcosa che l'utente è. Rischi: contraffazione e privacy, non può essere rimpiazzato quando compromesso.
\subsubsection{Autenticazione digitale}
Entità logiche:
\begin{itemize}
    \item CSP (Credential Service Provider): emette o registra le credenziali utente, verifica e memorizza attributi associati 
    \item verifier: esegue un protocollo di autenticazione per verificare il possesso di un autenticatore e credenziali 
    \item RP (Relying Party): richiede e riceve un'asserzione di autenticazione dal verifier per certificare l'identità utente 
\end{itemize}
I ruoli possono essere distinti o combinati in una sola entità.
\begin{center}
    \includegraphics[width=\textwidth]{autenticazione digitale.png}
\end{center}
Inizialmente l'attore è un applicant, ossia un'entità che non si è mai registrato al servizio a cui vuole accedere. In questa 
fase, l'applicant interagisce con il CSP, che registra l'applicant. Per registrarsi, l'applicant deve presentare le proprie 
credenziali. Il CSP valida le credenziali e le memorizza. A questo punto, l'applicant diventa claimant presso quel servizio
(servizio detto relying party). L'asserzione di autenticazione non è fatta direttamente al RP, ma al verifier, che dialoga 
con il claimant tramite un protocollo di autenticazione. Quando il verifier riceve informazioni dia utenticazione dal claimant, 
e le utilizza per dialogare con il CSP. Il CSP, avendo memorizzato in fase di registrazione gli attributi dell'applicant, 
passa questi attributi al verifier che li userà per autenticare il claimant al RP. Se l'autenticazione avviene con successo,
il verifier manda un'asserzione al RP e il claimant diventa subscriber.

Passi per autenticazione:
\begin{enumerate}
    \item il verifier chiede autenticazione 
    \item il claimant invia il proprio identificativo utente 
    \item il verifier richiede una prova 
    \item il claimant risponde inviando un segreto che solo lui può conoscere; il segreto non è trasmesso direttamente, 
    il claimant applica una funzione $F$ che restituisce la prova della sua asserzione di autenticazione 
    \item il verifier, che ha salvato il segreto, applica al segreto un'altra funzione $f$ e ottiene un altro risultato, che
    viene confrontato con la prova ricevuta per stabilire se l'autenticazione ha avuto successo 
\end{enumerate}
\subsection{Autenticazione basata su password ripetibili}
Il segreto è la password dell'utente. 

Il client genera e trasmette la prova, il server la memorizza e la valida. Due casi:
\begin{itemize}
    \item il server conserva le password in chiaro, il controllo d'accesso è verificare se ciò che è stato inserito e la 
    password corrispondono 
    \item il server conserva i digest delle password, il controllo d'accesso è verificare se ciò che è stato trasmesso e 
    l'hash corrispondono 
\end{itemize}
Pro: semplice per l'utente, a condizione che ne abbia una sola.

Contro: conservazione delle password, password indovinabili e validazione lato server.

Altri problemi includono: pssword sniffing, attacchi ai database delle password, riuso password, attacchi MITM e cryptography
ageing.
\subsubsection{Attacchi}
\paragraph*{Attacco dizionario}
Se l'algoritmo di hash è noto, e gli hash delle password sono noti, allora è possibile fare un attacco dizionario, comparando
hash noti con quelli delle password cifrate.
\paragraph*{Rainbow table}
Tecnica di trade-off per memorizzare e cercare una tabella. Utilizza una funzione di riduzione ($r:h\Rightarrow p$).
Ogni riga rappresenta più password.

Per generare la tabella, scelgo un set di password iniziali dell'insieme di password $P$, calcolo catene di hashing (alternando 
funzione di riduzione e funzione di hashing) di lunghezza $k$ per ognuna, e conservo solo l'ultima password di ogni catena.
Dato un valore hash $h$ da invertire, calcolo una catena cche inizia da $h$ applicando la funzione di riduzione e poi quella 
di hash alternando. Se a un certo punto un valore corrisponde a uno dei valori finali della tabella, il valore iniziale 
permette di ricreare la catena completa.

Per proteggere le password da questi attacchi si usa il sale, valore diverso per ciascun utente, random e lungo. 
\subsubsection{Autenticazione forte}
\paragraph*{Definizione ECB}
(internet banking) L'autenticazione forte è basata sull'uso di due o più fattori di autenticazione. Gli elementi selezionati devono essere 
indipendenti. Almeno un elemento dovrebbe essere non riutilizzabile e non replicabile. La procedura di autenticazione deve
proteggere la riservatezza dei dati di autenticazione.
\paragraph*{Definizione PCI-DSS}
(carte di credito) Richiesta autenticazione a più fattori (diversi) per accessi al Cardholder Data Environment(CDE). Obbligatorio per accesso 
da reti untrusted o per gli amministratori.
\paragraph*{Altre definizioni}
\begin{itemize}
    \item un protocollo crittografico a sfida (Handbook of Applied Cryptography)
    \item una tecnica resistente a un ben definito insieme di attacchi 
\end{itemize}
\subsection{Sistemi di autenticazione a sfida}
\subsubsection{Challenge-response authentication (CRA)}
Una sfida viene inviata al claimant (utente che si vuole autenticare), che risponde con una soluzione calcolata usando un 
qualche segreto e la sfida. Il verifier confronta la risposta con la soluzione calcolata tramite un segreto correlato al 
claimant.

La sfida deve essere non ripetibile per evitare attacchi replay. Di solito la sfida è un nonce (numero arbitrario utilizzabile 
una volta sola) random. La funzione che calcola la soluzione deve essere non invertibile, altrimenti un ascoltatore sul canala può registrare il traffico e calcolare il 
segreto del claimant. 
\subsubsection{Sistemi a sfida simmetrici}
Una sfida viene inviata al claimant, che risponde con il risultato di un calcolo che coinvolge il segreto condiviso con 
il verifier. Il verifier effettua lo stesso calcolo e confronta il suo risultato con la risposta. Il segreto è condiviso 
tra verifier e claimant. Spesso la funzione usata per il calcolo della risposta è una funzione di hash, perché più veloce.


Il segreto deve essere noto in chiaro al verifier, quindi è possibile fare attacchi contro la tabella degli id del verifier.
SCRAM (Salted CRA Mechanism) risolve questo problema usando hash di password sul verifier.
\subsubsection{Mutua autenticazione con protocolli a sfida simmetrici}
Supponiamo che A e B vogliano autenticarsi mutualmente. L'identità è dichiarata esplicitamente solo da chi inizia lo scambio.
\begin{enumerate}
    \item A si pressenta a B, esplicitando la sua identità 
    \item per autenticare A, B invia ad A una sfida $S_b$ (la sfida è in chiaro)
    \item A utilizza una funzione $f$ di cifratura o hashing per rispondere alla sfida; nel farlo, utilizza un segreto 
    condiviso con B, in questo caso una chiave simmetrica $K_{ab}$; nel frattempo, A invia anche la propria sfida, $S_a$
    \item B verifica localmente, grazie alla risposta ricevuta, se l'autenticazione ha avuto successo 
    \item B risponde alla sfida $S_a$ utilizzando la chiave $K_{ab}$
\end{enumerate} 
Alcuni di questi passaggi possono essere accorpati.
\begin{enumerate}
    \item A invia la richiesta e la sfida allo stesso momento 
    \item B risponde inviando la sua sfida e la risposta 
    \item A risponde con la risposta alla sfida di B
\end{enumerate}
Questo metodi di autenticazione non sono robusti perché vulnerabili ad attacchi. 
\begin{enumerate}
    \item un attaccante M, senza conoscere la chiave simmetrica, può comunque fingere di essere uno delle due entità che 
    si vogliono autenticare 
    \item M si presenta come A, inviando una sfida a B 
    \item B risponde alla sfida cifrando la risposta con la propria chiave 
    \item M riceve le informazioni e scarta la parte cifrata, catturando la sfida $C_b$ (è in chiaro, solo la risposta è 
    cifrata)
    \item M appre una seconda connesione verso B, trasmettendo la sfida ricevuta da B come sfida 
    \item B risponde alla sfida con un'altra sfida, cifrando la risposta con la stessa chiave 
    \item M riceve la risposta e la ritorna a B 
\end{enumerate}
In poche parole, utilizzo l'altra parte per generare la risposta a una sfida anche se non conosco la chiave di cifratura.
\subsubsection{Sistemi a sfida asimmetrici}
Più forte, richiede cifratura asimmetrica (il claimant deve avere coppia chiave pubblica/privata). 
\begin{enumerate}
    \item il claimant invia un certificato contenente id e chiave pubblica 
    \item il verifier invia una sfida calcolata utilizzando la chiave pubblica del certificato del claimant su un nonce 
    random $X$ 
    \item il claimant decifra la sfida utilizzando la propria chiave privata e invia la risposta al verifier 
    \item il verifier verifica che la risposta ottenuta sia equivalente al nonce random inviato  
\end{enumerate}
Il sistema è sicuro perché solo il claimant possiede la giusta chiave privata per decifrare il messaggio inviato dal verifier.

Sistema robusto, non richiede memorizzazione di segreti sul verifier, ma la procedura è lenta e il verifier deve verificare 
che il certificato sia valido.

Firma involontaria da parte del claimant: il verifier potrebbe inviare come sfida non la cifratura di nonce e chiave, ma 
l'hash di un certificato o documento elettronico; il claimant, fidandosi del server, usa la sua chiave privata per cifrare 
la sfida, creando quindi la firma digitale di quel documento.
\subsection{OTP}
In fase di autenticazione, la password utilizzata è sempre diversa.

Pro:
\begin{itemize}
    \item password valida solo una volta per il protocollo di autenticazione 
    \item immune allo sniffing
\end{itemize}

Contro:
\begin{itemize}
    \item soggetto a attacchi MITM, il verifier deve autenticare 
    \item difficile fornire password ai subscriber, le password sono molte e si possono esaurire 
    \item l'inserimento della password è difficile 
\end{itemize}
Per fornire le OTP si possono utilizzare password precalcolate o applicativi software.
\subsubsection{Sistema S/KEY}
Prima implementazione di OTP.
\begin{enumerate}
    \item al momento della sottoscrizione, il claimant fornisce un segreto iniziale $S$, come una passphrase 
    \item applica una funzione di hash sul segreto ottenendo l'OTP $P_1$
    \item applica nuovamente la funzione di hash su $P_1$ ottenendo $P_2$
    \item il processo di hashing viene ripetuto fino a ottenere $P_n$
    \item l'utente ora ha $n$ OTP salvate a lato del claimant
    \item il verifier memorizza solo l'ultima password della sequenza, $P_n$
    \item quando il verifier deve autenticare il claimant, chiederà la password $P_{n-1}$
    \item il claimant invia la password $P_{n-1}$ e il verifier verifica, con funzione di hash, di aver ottenuto $P_n$
    \item $P_n$ viene scartata, e alla prossima autenticazione verrà svolta la stessa operazione ma con la password successiva,
    fino all'esaurimento delle password 
\end{enumerate}
Il sistema richiede che la passphrase (pp) sia lunga almeno 8 caratteri. La pp viene concatenata con un seed inviato dal 
server, il che permette di utilizzare la stessa pp su server diversi (basta cambiare seme).

Le password sono lunghe 64 bit, compromesso tra sicurezza e complessità. Per semplificare l'inserimento di queste password, 
si è creato un vocabolario inglese di 2048 parole dal quale si possono pescare le password.

Contro delle OTP:
\begin{itemize}
    \item scomode da utilizzare 
    \item scomode per accedere ai serivizi con autenticazione ripetuta 
    \item costose se basate su autenticatori hw 
    \item non possono essere utilizzate da processi solo da esseri umani 
    \item è necessario un generatore di buone password random 
    \item problemi legati alla trasmissione all'utente 
\end{itemize}
Con autenticatori hw, problemi legati al DoS e al social engineering.
\subsubsection{TOTP}
Le OTP possono essere generate a partire dal tempo e dal segreto del claimant. 
Quando l'utente deve identificarsi nei confronti del verifier, il claimant non pesca da un insieme di OTP precalcolato, 
ma la calcola sul momento utilizzando:
\begin{itemize}
    \item segreto condiviso con il verificatore, $S_{id}$
    \item l'istante temporale in cui la password viene calcolata 
\end{itemize}
Richiede un calcolo locale al claimant, e sincronizzazione tra clock del claimant e del verifier (altrimenti avrebbero 
un seed diverso). Un attacco efficace è la disincronizzazione dei clock. Viene consentita una minima finestra di disincronizzazione per evitare errori.  

A differenza di OTP, a lato verifier viene memorizzato il segreto condiviso, quindi è attaccabile.

Esempio (RSA SecurID):
\begin{enumerate}
    \item il claimant invia al verifier, in chiaro: user (id), PIN ($S_{id}$), token-code calcolato a partire da seed e 
    dall'istante di tempo 
    \item il verifier, in base a user e PIN verifica contro tre possibili token-code: quello con riferimento temporale 
    corrente, precedente e futuro (finestra temporale d'errore)
\end{enumerate}
È possibile che il claimant invii al verifier un token code speciale, il duress code, che fa scattare un allarme nel caso 
il claimant si trovi sotto minaccia.    

Per comunicare con questo protocollo, claimant e verifier devono installare due componenti ACE (Access Control Engine).
\subsubsection{Event-baesd OTP}
Usa un contatore intero monotonico $C$ come input oltre al seed: $p(ID, C)=h(C,S_{id})$
Il contatore viene incrementato dal claimant ogni volta che è necessario creare una nuova OTP.

Vantaggi rispetto a TOTP:
\begin{itemize}
    \item possibile fare autenticazioni a raffica, con serve aspettare il lasso di tempo della finestra d'errore 
    \item è possibile precalcolare OTP (intercettabile da un attaccante)
\end{itemize}
È possibile che ci siano disincronizzazioni.
\subsubsection{OOB OTP}
L'OTP è generata lato verifier e trasmessa al claimant tramite un canale diverso rispetto a quello con cui stanno comunicando.
\begin{enumerate}
    \item alla richiesta di autenticazione, il claimant invia al verifier l'identificativo utente e un segreto
    \item il verifier utilizza i dati ricevuti per generare dati di autenticazione, ossia l'OTP 
    \item l'OTP viene trasmessa su un canale diverso
    \item il claimant riceve l'OTP e la trasmette sul canale utilizzato per comunicare con il verifier 
\end{enumerate}
L'OTP da claimant a verifier va trasmesso in maniera sicura, ad esempio con cifratura, per evitare MITM
\subsubsection{2/MFA}
Utilizzo di più di un fattore di autenticazione per proteggere il verifier e aumentare la forza dell'autenticazione. Ad 
esempio, PIN+OTP. È rischioso se non ci sono meccanismi di protezione su unlock multipli.
\subsection{Zero-knowledge proof (ZKP)}
Una parte, prover, dimostra all'altra, verifier, che una certa affermazione è vera, senza fornire altre informazioni (ad 
esempio, senza password).

Esempio concreto: caverna di Ali Babà
\begin{enumerate}
    \item A deve identificarsi nei confronti di B 
    \item A entra nella grotta, può prendere due strade 
    \item B entra nella grotta, non sa che strada abbia preso A 
    \item B chiede ad A di uscire da una delle due strade 
    \item se A non conosce la parola magica, non può passare da una strada sbarrata dalla porta 
    \item in ogni caso, A non trasmette mai a B la propria password  
\end{enumerate}
Problema: A è già dalla parte giusta. Soluzione: ripetere più volte la prova.

Nel concreto, conoscena del logaritmo discreto con un certo valore.
\subsubsection{ZKPP}
Molti algoritmi, spesso associati a stabilire una chiave segreta condivisa, in modo da poter fornire mutua autenticazione 
e channel binding. Per stabilire una chiave, si utilizzano gli algoritmi PAKE.
\subsection{Autenticazione biometrica}
Assicurarsi di star interagendo con esseri umani, ad esempio tramite impronte digitali, ecc\dots

Problemi:
\begin{itemize}
    \item FAR (False Acceptance Rate)
    \item FRR (False Rejection Rate)
    \item accettazione psicologica
    \item privacy
    \item insostituibile 
    \item mancanza di standard 
\end{itemize}
FAR e FRR visualizzabile come sovrapposizione di due curve gaussiane.
\subsection{Kerberos}
Realizzato per autenticazione da remoto. Caratterizzato dall'esistenza di una terza parte fidata, (TTP) che autentica un 
claimant nei confronti del servizio. Questa possiede il segreto condiviso con il claimant e crea una struttura dati, detta 
ticket, che il claimant utilizza per autorizzarsi al servizio. 

Il ticket è una struttura dati che il claimant non è in grado di interpretare, generata dalla TTP e che verifica il claimant 
al server.

Termini chiave: 
\begin{itemize}
    \item realm: dominio di Kerberos, insieme di sistemi che utilizzano Kerberos come sistema di autenticazione 
    \item credenziale: user.instance@realm
\end{itemize}

Flusso di autenticazione con Kerberos:
\begin{enumerate}
    \item ogni realm ha un authentication server, che possiede tutti i segreti degli utenti del realm, il suo compito è 
    generare un ticket speciale detto Ticket Granting Ticket (TGT)
    \item il TGT permette al client di autenticarsi a un Ticket Granting Server (TGS)
    \item il TGS consente al client di autenticarsi nei confronti di uno specifico server applicativo 
\end{enumerate}
Il ticket è cifrato con una chiave simmetrica, $K_s$, che deve essere la chiave del server rispetto a cui ci si sta autenticando.
Contiene:
\begin{itemize}
    \item id del server
    \item id del client 
    \item indirizzo del client  
    \item timestamp 
    \item durata 
    \item chiave $K{s,c}$ che permette la comunicazione cifrata tra client e server 
\end{itemize}
L'autenticatore è un'altra struttura dati che viene cifrata con la chiave $K{s,c}$ e contiene solo client id e indirizzo 
e timestamp. Serve per autenticare il client nei confronti di chi dovrà fornire il servizio.

Richiesta di TGT:
\begin{enumerate}
    \item il client interagisce con l'authentication server trasmettendo il suo identificativo e quello del servizio con 
    cui deve interagire (il TGS)
    \item l'AS manda un messaggio, cifrato con la chiave simmetrica del client, contenente
    \begin{itemize}
        \item $K_{c,TGS}$ utilizzata per la comunicazione sicura tra client e TGS
        \item TGT, cifrato con $K_{TGS}$, la chiave del TGS 
    \end{itemize}
\end{enumerate}
Richiesta del ticket:
\begin{enumerate}
    \item il client invia al TGS l'id del server con il quale intende comunicare, il TGT e l'autenticatore, che è cifrato 
    con $K_{c,TGS}$
    \item il TGS estrae il vero ticket, che solo lui può decifrare, e il contenuto dell'autenticatore, controllando che 
    la validità temporale dettata dal timestamp e le corrispondenze tra id siano validi 
    \item il TGS invia al client un messaggio (ticket) cifrato con la chiave simmetrica condivisa tra TGS e client, contenente 
    \begin{itemize}
        \item il ticket per il servizio al quale il client vuole autenticarsi, cifrato con la chiave del servizio
        \item una chiave simmetrica $K_{c,s}$ per la comunicazione sicura tra client e server 
    \end{itemize} 
\end{enumerate}
Uso del ticket:
\begin{enumerate}
    \item il ticket ricevuto dal TGS viene inviato al servizio insieme all'autenticatore, il tutto cifrato con $K_{c,s}$ 
    \item il servizio decifra il ticket, confronta i dati ricevuti e, fidandosi del TGS, autentica il client 
    \item il servizio invia al client il timestamp dell'autenticatore + 1 cifrato con $K_{c,s}$, dando la prova di essere 
    stato in grado di decifrare il messaggio del client 
\end{enumerate}
\subsubsection{Single sign-on}
Fornire all'utente un'unica credenziale con cui autenticarsi per tutte le operazioni su qualunque sistema.
\paragraph*{SSO fittizio}
Password diverse in un file unico.
\paragraph*{SSO vero}
Tipo Kerberos, le applicazioni devono essere modificate per supportarlo.
\subsection{FIDO}
Autenticazione asimmetrica forte, prevede 2FA e autenticazione biometrica.
Flusso di autenticazione:
\begin{enumerate}
    \item in fase di autenticazione è generata una coppia di chiavi pubblica/privata tramite sistema biometrico 
    \item la chiave pubblica è trasmessa al server 
    \item per effettuare il login, l'utente utilizza autenticazione biometrica per sbloccare la chiave privata 
    \item la chiave privata viene utilizzata per la firma digitale del testo che viene poi trasmesso al server 
    \item il server verifica la firma con la chiave pubblica 
\end{enumerate}
Non vi è nessuna terza parte a cui fare affidamento (a differenza di Kerberos), e a lato server non vengono conservati 
segreti. Immune al phishing perché la risposta non può essere riutilizzata. Generando una coppia di chiavi a ogni accesso, 
non esiste tracciabilità tra servizi diversi usati dallo stesso utente.
\section{Sicurezza delle applicazioni di rete}
\subsection{Sicurezza di messaggio}
Le proprietà di sicurezza (autenticazione, integrità, segretezza) sono contenute nel messaggio. 

Pro: garantisce il non ripudio.

Contro: richiede modifica delle applicazioni, protocollo di sicurezza apposito per comunicare con altre applicazioni.

Due applicazioni dovrebbero condividere solo il canale logico (socket).
\subsection{Sicurezza di canale}
I messaggi generati non posseggono già le proprietà di sicurezza, queste sono garantite dal canale di comunicazione. 

Pro: non richiede modifica alle applicazioni.

Contro: non garantisce il non ripudio. 

Il livello sessione è quello ideale per implementare le funzioni di sicurezza, ma non esiste nel modello TCP/IP. È stato 
proposto un nuovo livello di sessione sicura: si svincola lo sviluppo del protocollo dallo sviluppo dell'applicazione e 
garantisce interoperabilità tra applicazioni.
\subsubsection{SSL}
Fornisce:
\begin{itemize}
    \item autenticazione obbligatoria del server e opzionale del client 
    \item riservatezza dei messaggi
    \item integrità e autenticazione dei messaggi 
    \item protezione da replay e filtering 
\end{itemize}
SSL lavora a livello di sessione quindi è applicabile a tutti i protocolli basati su TCP.

La peer authentication viene garantita durante la fase di handshake di SSL svolto all'apertura del canale. Il server si 
autentica presentando la sua chiave pubblica (certificato X.509) e subendo una sfida asimmetrica implicita. L'autenticazione 
del client è opzionale, ma quando avviene è una sfida esplicita.

L'autenticazione e integrità dei dati avviene mediante il calcolo di un MAC (keyed-digest).

La protezione da attacchi replay e filtering avviene tramite il calcolo di un MID, che identifica il numero di sequenza 
dei dati.

La riservatezza è garantita tramite generazione dal client di una session key usata per la cifratura simmetrica dei dati.
La chiave viene comunicata al server o concordata tramite crittografia a chiave pubblica (RSA, DH).

\subsubsection{Flusso di operazione di SSL}
\begin{enumerate}
    \item il browser vuole connettersi a un indirizzo web 
    \item server e browser negoziano le configurazioni di sicurezza (quali chiavi e algoritmi utilizzare, ecc\dots)
    \item peer authentication del server tramite certificato 
    \item opzionalmente, quella del browser 
    \item viene aperto il canale sicuro SSL  
\end{enumerate}
Per evitare di rinegoziare a ogni connessione i parametri crittografici, il server SSL può offrire un session identifier.
Se il client presenta un session id valido, il server apre subito il canale SSL.
\subsubsection{Architettura protocollare con SSL}
Si hanno gli header del livello 3 e 4, e un SSL record protocol che fornisce alcune informazioni sul contenuto del messaggio.
Dopo l'header, 4 alternative:
\begin{itemize}
    \item dati di livello applicativo 
    \item protocollo di handshake SSL che realizza la negoziazione dei parametri 
    \item SSL change cipher spec protocol, da un lato permette di comunicare l'intenzione di voler iniziare un canale 
    sicuro, dall'altro cambiare protocollo di cifratura 
    \item SSL alert protocol, per gli avvisi 
\end{itemize}
Generazione di un messaggio SSL:
\begin{enumerate}
    \item i dati applicativi (es generati da HTTP) vengono frammentati 
    \item i frammenti sono compressi 
    \item su ogni frammento viene calcolato un MAC per integrità e identificazione 
    \item applicazione di padding per successiva cifratura 
    \item aggiunta header di SSL record protocol 
\end{enumerate}
Contenuti dell'header SSL:
\begin{itemize}
    \item intero senza segno che rappresenta il tipo di dati che contiene: 20 (change cipher spec), alert (21), handshake (22), appdata (23)
    \item intero senza segno per la versione 
    \item bit della lunghezza dei dati 
\end{itemize}
\paragraph*{Calcolo MAC}
Si utilizza una funzione di digest (come SHA-1) che richiede:
\begin{itemize}
    \item chiave concordata all'inizio della sessione 
    \item sequence number, intero a 64 bit mai trasmesso ma calcolato implicitamente 
\end{itemize}
Calcolo MAC:
\begin{enumerate}
    \item il MAC viene generato a partire dai dati compressi, chiave e numero di sequenza
    \item la cifratura simmetrica si applica sull'insieme di dati compressi, MAC e padding 
    \item 
\end{enumerate}
È authenticate-then-encrypt.
\subsubsection{Protocollo di handshake}
Serve per:
\begin{itemize}
    \item concordare la coppia di algoritmi per la confidenzialità e integrità 
    \item scambiare numeri casuali tra client e server per la successiva generazione di chiavi 
    \item stabilire chiave simmetrica 
    \item negoziare session-id 
    \item scambiare i certificati  
\end{itemize}
Processo per la generazione di parametri crittografici:
\begin{enumerate}
    \item viene generato un segreto condiviso con algoritmo a chiave pubblica (pre-master secret)
    \item dal pre-master secret, client e server ricavano il master secret, che è comune a tutte le sessioni, e tiene conto del client randome e server random della prima connessione 
    \item per ogni singola connessione, client e server avranno chiavi per MAC, chiavi per cifrare e IV diversi per ogni 
    connessione  
\end{enumerate}
\subsubsection{Meccanismi effimeri per generazione delle chiavi}
Una volta ottenuto un certificato X.509, è possibile generare una chiave simmetrica tramite DH effimero. Questo garantisce 
perfect forward secrecy, ossia la chiave privata del server viene utilizzata solo per firma, per evitare che chi la conosca 
possa decifrare tutte le sessioni. 
\subsubsection{Scambio delle credenziali}
\begin{enumerate}
    \item client hello e server hello, vengono negoziati i parametri crittografici
    \item il server invia il suo certificato, un'eventuale richiesta di certificato del client e eventuale server key exchange 
    \item autenticazione del client con certificato (opzionale), scambio chiave simmetrica usata come parametro per future 
    comunicazioni 
    \item la comunicazione termina con un messaggio di change cipher spec e finished
\end{enumerate}
\paragraph*{Client hello}
Contiene
\begin{itemize}
    \item versione di SSL preferita dal client 
    \item 28
\end{itemize}
\end{document}
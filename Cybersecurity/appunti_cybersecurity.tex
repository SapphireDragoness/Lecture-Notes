\documentclass[11pt]{article}
\usepackage[margin=.8in]{geometry}
\usepackage[italian]{babel}

\title{Appunti Cybersecurity}

\begin{document}
\section{Cybersecurity essentials}
\subsection{Definizioni di sicurezza}
La \textbf{sicurezza informatica} è l'insieme dei servizi, delle regole organizzative e dei comportamenti individuali che proteggono 
i sistemi informatici di un'azienda. Ha il compito di proteggere le risorse da accessi indesiderati, garantire la riservatezza 
delle informazioni, assicurare il funzionamento e la disponibilità dei servizi a fronte di eventi imprevedibili.
\subsection{Proprietà di sicurezza}
\paragraph*{Autenticazione}
Il servizio di \textbf{autenticazione} si preoccupa dell'autenticità della comunicazione. Vengono definiti due tipi di 
autenticazione: \textbf{autenticazione delle controparti} e \textbf{autenticazione dell'origine dei dati}.
\paragraph*{Controllo degli accessi}
Il controllo degli accessi è l'abilità di limitare e controllare l'accesso ai sistemi e alle applicazioni tramite canali
di comunicazione. Ogni entità interessata ad accedere ad un servizio deve prima essere autenticata.
\paragraph*{Confidenzialità}
La confidenzialità è l'atto di proteggere i dati trasmessi dagli attacchi passivi. 
\paragraph*{Integrità dei dati}
Il principio di integrità è applicabile ad un flusso di dati
\end{document}
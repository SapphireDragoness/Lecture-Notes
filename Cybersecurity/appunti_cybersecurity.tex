\documentclass[11pt]{article}
\usepackage[margin=.8in]{geometry}
\usepackage[italian]{babel}

\title{Appunti Cybersecurity}

\begin{document}
\section{Cybersecurity essentials}
\subsection{Definizioni di sicurezza}
La \textbf{sicurezza informatica} è l'insieme dei servizi, delle regole organizzative e dei comportamenti individuali che proteggono 
i sistemi informatici di un'azienda. Ha il compito di proteggere le risorse da accessi indesiderati, garantire la riservatezza 
delle informazioni, assicurare il funzionamento e la disponibilità dei servizi a fronte di eventi imprevedibili.
\subsection{Proprietà di sicurezza}
\paragraph*{Autenticazione}
Il servizio di \textbf{autenticazione} si preoccupa dell'autenticità della comunicazione. Vengono definiti due tipi di 
autenticazione: \textbf{autenticazione delle controparti} e \textbf{autenticazione dell'origine dei dati}.
\paragraph*{Controllo degli accessi}
Il controllo degli accessi è l'abilità di limitare e controllare l'accesso ai sistemi e alle applicazioni tramite canali
di comunicazione. Ogni entità interessata ad accedere ad un servizio deve prima essere autenticata.
\paragraph*{Confidenzialità}
La confidenzialità è l'atto di proteggere i dati trasmessi dagli attacchi passivi. 
\paragraph*{Integrità dei dati}
Il principio di integrità è applicabile ad un flusso di dati

\section{Crittografia}
\subsection{Crittografia simmetrica}
Crittografia con chiave comune e unica, a basso carico di elaborazione.
\subsubsection{Algoritmi di crittografia simmetrica}
\paragraph*{DES}
Chiave a 56 bit + 8 di parità, se applicato 3 volte viene detto 3DES. 2DES è vulnerabile a un attacco di tipo known-plaintext
detto meet in the middle.
\paragraph*{IDEA}
Chiave a 128 bit, blocco dati da 64 bit, utilizza XOR, addizione mod16 e moltiplicazione mod$2^{16}+1$.
\paragraph*{RC2, RC4}
Più veloci di DES, con chiave a lunghezza variabile e blocchi da 64 bit.
\subsubsection{Applicazione algoritmi a blocchi}
\paragraph*{ECB (Electronic Code Book)}
Per cifrare dati in quantità superiore. Ogni blocco viene cifrato con lo stesso algoritmo separatamente. Sconsigliato 
perché cifra allo stesso modo blocchi identici.
\paragraph*{CBC (Cipher Block Chaining)}
Per cifrare dati in quantità superiore. Richiede IV, XOR tra blocco cifrato precedente e blocco da cifrare, poi applicazione 
algoritmo di cifratura.
\paragraph*{Padding}
Per cifrare dati in quantità inferiore. Aggiungo bit per riempire lo spazio vuoto. Alcuni tipi offrono controllo d'integrità,
applicando padding a tutti i blocchi.
\paragraph*{CTS (Cipher Text Stealing)}
Per cifrare dati in quantità inferiore. L'ultimo blocco è riempito con byte del penultimo blocco, questi due blocchi 
vengono scambiati durante la cifratura.
\paragraph*{CTR (Counter mode)}
Per cifrare dati in quantità inferiore. Accesso random al testo cifrato, usa algoritmo a blocchi per cifrare n bit alla 
volta
\subsubsection{Algoritmi stream}
Operano su un flusso di dati senza richiederne la divisione in blocchi, tipicamente su un bit o byte.
\paragraph*{Salsa20 e ChaCha20}
Chiavi da 128 o 256 bit. Operazione base: add-rotate-xor su 32 bit. Effettuano 20 volte mixing su input.
\subsubsection{Distribuzione chiavi}
Per una comunicazione privata tra $n$ persone occorrono $\frac{n(n-1)}{2}$ chiavi. Avviene tramite algoritmi per scambio
chiavi.
\subsection{Crittografia asimmetrica}
Le chiavi sono diverse e hanno funzionalità reciproca. È possibile generare un messaggio segreto per uno specifico destinatario 
conoscendone solo la chiave pubblica.
\subsubsection{Algoritmi a chiave pubblica}
\paragraph*{DSA}
Elevamento a potenza e logaritmo del risultato, utilizzato solo per firma digitale.
\paragraph*{RSA}
Può solo cifrare dati il cui valore sia inferiore al modulo pubblico. Funzionamento:
\begin{enumerate}
    \item modulo pubblico $n=pq$, con $p$ e $q$ primi, grandi e segreti
    \item $\phi=(p-1)(q-1)$
    \item esponente pubblico $e$ tale che $1<e<\phi$, $e$ coprimo $\phi$
    \item esponente privato: $d=e^{-1}\phi$
    \item chiave pubblica: $(n,e)$, chiave privata: $(n,d)$
\end{enumerate}
Solitamente le chiavi pubbliche hanno un $e$ che contiene solo due bit a 1 per ottimizzare le prestazioni.

RSA è debole se vengono utilizzati esponenti piccoli, stesse chiavi per firma e cifratura. Per renderlo più forte, 
aggiungere sempre del padding fresco prima di cifrare il messaggio e non firmare dati grezzi.
\subsubsection{Distribuzione chaivi per crittografia asimmetrica}
\paragraph*{Diffie-Hellman}
Sfrutta la difficoltà di risoluzione del problema dell'algoritmo discreto. 
\paragraph*{Curve ellittiche}
Problema del logaritmo discreto sulla curva, più complesso, permette di avere chiavi più corte. Firma digitale: ECDSA, key
agreement: ECDH, key distribution: ECIES.
\subsection{Funzioni di hash e digest}
\subsubsection{Digest}
È un riassunto a lunghezza fissa del messaggio da proteggere. Deve essere veloce da calcolare, difficile da invertire e 
non generare troppe collisioni (digest uguali). Un algoritmo di digest a $n$ bit è insicuro quando vengono generati più 
di $2^{\frac{n}{2}}$ digest perché si ha una probabilità di collisione pari al 50\%.
\subsubsection{Funzioni di hash}
Dividono il messaggio in blocchi e applicano la funzione base per ottenere il valore di hash.
\end{document}